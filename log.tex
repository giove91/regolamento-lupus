\documentclass[a4paper,10pt]{article}


\usepackage[utf8]{inputenc}
\usepackage[italian]{babel}
\usepackage{amsmath}
\usepackage{amsthm}
\usepackage{fancyhdr}
\usepackage{amsfonts}
\usepackage{amssymb}
\usepackage{makeidx}
\usepackage[parfill]{parskip}
\usepackage[colorlinks]{hyperref}
\usepackage{fullpage}
\usepackage{mathpazo}
%\usepackage{utopia}
\usepackage{graphicx}

% * define a `\twoidxcolumn` based on `\twocolumn`:
\def\twoidxcolumn{%
%\clearpage
\global\columnwidth\textwidth%
\global\advance\columnwidth-\columnsep%
\global\divide\columnwidth\tw@%
\global\hsize\columnwidth%
\global\linewidth\columnwidth%
\global\@twocolumntrue%
\global\@firstcolumntrue%
\col@number \tw@%
%\@ifnextchar [\@topnewpage
\@floatplacement%
}%


\makeatletter%
\def\@wrindex#1{%
   \protected@write\@indexfile{}%
      {\string\indexentry{#1}{\theenumi}}%
      \endgroup%
      \@esphack}%

\makeatletter
\renewenvironment{theindex}
               {\twocolumn[\section*{Indice delle parole chiave presenti nella
Sezione \ref{faq}}]%
                \@mkboth{\MakeUppercase\indexname}%
                        {\MakeUppercase\indexname}%
                \thispagestyle{plain}\parindent\z@%
                \parskip\z@ \@plus .3\p@\relax%
                \columnseprule \z@%
                \columnsep 35\p@%
                \let\item\@idxitem}
               {}
\makeatother

\makeindex

\newcommand{\smallspace}{\vskip0.3cm}

% Title Page
\title{Lupus in tempo reale\\ Cronologia versioni}

\begin{document}
	
\maketitle

\section{Principali cambiamenti nella quarta edizione}
 
Questa sezione è pensata per dare a chi ha giocato alla terza edizione di Lupus un'idea delle modifiche più importanti che sono state apportate. Tuttavia non sostituisce un'attenta lettura del resto del regolamento.
 
\begin{itemize}
	\item La Fazione dei Vampiri è stata sostituita dalla nuova Fazione dei	Negromanti, che ha meccaniche di gioco molto diverse.
	\item È stato eliminato il suicidio.
	\item È stato eliminato il rituale dei fantasmi.
	\item I possibili colori dell'aura sono tornati ad essere solamente il bianco e	il nero.
	\item Molti ruoli sono stati creati, eliminati o modificati. A chi è abituato al regolamento della terza edizione raccomandiamo di controllare con particolare	attenzione i seguenti ruoli, che hanno subito i cambiamenti più rilevanti: Cacciatore, Custode del cimitero, Esorcista, Investigatore, Rinnegato, Profanatore di tombe, Negromante, Fantasma, Ipnotista, Medium, Spettro.
	\item Le comunicazioni di gioco (votazione, utilizzo dei poteri, resoconto dell'alba e del tramonto) avvengono attraverso un'interfaccia web e non più via
	e-mail.
\end{itemize}
 
\section{Principali cambiamenti nella quinta edizione}

Questa sezione è pensata per dare a chi ha giocato alla quarta edizione di Lupus un'idea delle modifiche più importanti che sono state apportate. Tuttavia non sostituisce un'attenta lettura del resto del regolamento.

\begin{itemize}
	\item È stato introdotto lo Scrutatore.
	\item È stato eliminato lo Spettro con il potere della Duplicazione.
	\item L'ultimo Ipnotista, quando muore, diventa uno Spettro con il potere dell'Ipnosi.
	\item Il Profanatore di Tombe è stato trasferito alla fazione dei Popolani ed è stato rinominato in Sciamano. Rimane con aura nera, ma diventa mistico.
	\item Il Necrofilo è stato rinominato in Trasformista.
	\item Il Medium scopre il ruolo del personaggio scelto, e non più l'aura.
\end{itemize}

\section{Principali cambiamenti nella sesta edizione}
 
Questa sezione è pensata per dare a chi ha giocato alla quinta edizione di Lupus un'idea delle modifiche più importanti che sono state apportate. Tuttavia non sostituisce un'attenta lettura del resto del regolamento.

\begin{itemize}
	\item Il Trasformista e il Fantasma hanno ora aura nera.
	\item Nessun personaggio può utilizzare il proprio potere su sé stesso.
	\item Sono state rimosse le restrizioni sull'utilizzo di un potere due notti consecutive sullo stesso personaggio.
	\item Sono stati eliminati il Custode del Cimitero, il Cacciatore, e il potere soprannaturale della Mistificazione.
	\item Sono stati introdotti l'Assassino, lo Stregone, e il potere soprannaturale della Confusione.
	\item I personaggi della fazione dei Lupi non possono diventare Spettri.
	\item Le conoscenze iniziali dei personaggi della fazione dei Lupi sono state modificate.
	\item Il potere soprannaturale dell'Ipnosi può ora essere scelto dai Negromanti; il primo Ipnotista che muore diventa uno Spettro con tale potere (se disponibile).
	\item Il Messia fallisce se utilizza il proprio potere su uno Spettro.
	\item L'Avvocato del Diavolo è stato rinominato in Avvocato.
\end{itemize}

\section{Principali cambiamenti nella settima edizione}
 
Questa sezione è pensata per dare a chi ha giocato alla sesta edizione di Lupus un'idea delle modifiche più importanti che sono state apportate. Tuttavia non sostituisce un'attenta lettura del resto del regolamento.
 
\begin{itemize}
	\item Il poteri speciali del Trasformista, dei Lupi, ed il potere soprannaturale della Morte hanno effetto solo sui componenti della Fazione dei Popolani.
 	\item Il potere speciale del Diavolo non ha effetto sui componenti della Fazione dei Negromanti.
 	\item Il potere speciale dei Negromanti può essere usato ogni notte.
 	\item L'Ipnotista non diviene uno Spettro dopo la sua morte.
 	\item I seguenti poteri soprannaturali sono stati modificati: Confusione, Ipnosi, Morte, Visione. È stato aggiunto un nuovo potere soprannaturale: Corruzione.
 	\item Le restrizioni sulla spettrificazione sono state modificate.
 	\item È stato reintrodotto il ruolo di Custode del cimitero. Oltre al vecchio potere, Custode del cimitero e Guardia del corpo scoprono anche quanti personaggi hanno agito sul loro stesso bersaglio.
 	\item È stato introdotto un nuovo ruolo: il Necrofilo. Appartiene alla Fazione dei Lupi.
 	\item La Fattucchiera ha aura bianca.
 	\item Il potere speciale dello Scrutatore è stato modificato.
\end{itemize}

\end{document}
