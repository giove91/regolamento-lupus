\documentclass[a4paper,10pt]{article}


\usepackage[utf8]{inputenc}
\usepackage[italian]{babel}
\usepackage{amsmath}
\usepackage{amsthm}
\usepackage{fancyhdr}
\usepackage{amsfonts}
\usepackage{amssymb}
\usepackage{makeidx}
\usepackage[parfill]{parskip}
\usepackage[colorlinks]{hyperref}
\usepackage{fullpage}
\usepackage{mathpazo}
%\usepackage{utopia}
\usepackage{graphicx}

% * define a `\twoidxcolumn` based on `\twocolumn`:
\def\twoidxcolumn{%
%\clearpage
\global\columnwidth\textwidth%
\global\advance\columnwidth-\columnsep%
\global\divide\columnwidth\tw@%
\global\hsize\columnwidth%
\global\linewidth\columnwidth%
\global\@twocolumntrue%
\global\@firstcolumntrue%
\col@number \tw@%
%\@ifnextchar [\@topnewpage
\@floatplacement%
}%


\makeatletter%
\def\@wrindex#1{%
   \protected@write\@indexfile{}%
      {\string\indexentry{#1}{\theenumi}}%
      \endgroup%
      \@esphack}%

\makeatletter
\renewenvironment{theindex}
               {\twocolumn[\section*{Indice delle parole chiave}]%
                \@mkboth{\MakeUppercase\indexname}%
                        {\MakeUppercase\indexname}%
                \thispagestyle{plain}\parindent\z@%
                \parskip\z@ \@plus .3\p@\relax%
                \columnseprule \z@%
                \columnsep 35\p@%
                \let\item\@idxitem}
               {}
\makeatother

\makeindex

\newcommand{\smallspace}{\vskip0.3cm}

% Title Page
\title{Lupus in tempo reale\\ Regolamento dell'undicesima edizione}
\author{Lorenzo Bresolin, Alessandro Iraci, Giovanni Italiano, Matteo Migliorini}

\begin{document}

\maketitle

\begin{tabular}{lp{0.6\textwidth}}
	\begin{minipage}{0.22\textwidth}
		\vspace{3mm}
		\href{http://creativecommons.org/licenses/by/4.0/}{\includegraphics{ccby.pdf}}
	\end{minipage}
	 &
	This work is licensed under a \href{http://creativecommons.org/licenses/by/4.0/}{Creative Commons Attribution 4.0 International License}.
\end{tabular}

\section*{Le cose che vorreste sapere ma non avete mai osato chiedere}
\label{faq}

Le regole spiegate nel regolamento sono più che sufficienti per poter giocare. Tuttavia possono accadere eventi piuttosto strani (ad esempio, cosa succede se due Sequestratori si sequestrano a vicenda?). Ai giocatori è sempre permesso chiedere ai GM cosa accade in una di queste combinazioni.

Segue una lista, verosimilmente non esaustiva, di eventualità bizzarre con le corrispettive spiegazioni.
Per rendere questo elenco più facilmente consultabile, più avanti vi è un indice analitico con la lista dei ruoli (o altre parole chiave) e dei punti in cui essi compaiono.

\begin{enumerate}

	\item Con \emph{abilità} si intende la capacità di un personaggio vivo di effettuare un'azione speciale di notte. Con \emph{potere} si intende la capacità di un personaggio morto di effettuare un'azione speciale di notte. Con azione speciale si intendono sia potere che abilità.

	\item L'espressione \emph{Spettro del [Nome]} è sinonimo di \emph{Spettro su cui è stato attivato l'Incantesimo [Nome]}. L'espressione \emph{Spettro vuoto} è sinonimo di \emph{Spettro su cui non è attivo nessun Incantesimo}.
	      \index{Spettro}

	\item L'utilizzo di un potere non genera un movimento.

	\item In caso di contraddizione nell'utilizzo delle azioni speciali, viene scelto casualmente un personaggio (o più di uno, se necessario), che non utilizza la propria azione speciale e riceve la consueta notifica di fallimento. Questo viene detto fallimento ``di terza categoria''.

	      Ad esempio, se un Sequestratore agisce su un Medium che agisce su uno Spettro dell'Occultamento che agisce sul Medium, non c'è alcun modo coerente di risolvere le azioni.

	      \index{Medium}
	      \index{Sequestratore}
	      \index{Spettro!Occultamento}
	      \index{Fallimento delle azioni speciali@\emph{Fallimento delle azioni speciali}}

	\item Può accadere che esistano più modi logicamente accettabili di risolvere le azioni speciali. In questi casi, uno di questi modi viene scelto casualmente.

	      Ad esempio, se uno Spettro dell'Invisibilità agisce su un Sequestratore che agisce su un Esorcista che agisce sul Sequestratore mentre una Guardia agisce sull'Esorcista, è coerente che lo Spettro ed il Sequestratore abbiano successo, l'Esorcista fallisca e la Guardia (pur avendo successo) non blocchi il Sequestratore (perché è invisibile); oppure che l'Esorcista abbia successo, la Guardia blocchi il Sequestratore e Spettro e Sequestratore falliscano.

	      \index{Esorcista}
	      \index{Sequestratore}
	      \index{Spettro!Invisibilità}
	      \index{Guardia}

	\item La morte non è una causa di fallimento. L'abilità di un personaggio che agisce con successo la stessa notte in cui muore si risolve normalmente.
	      \index{Fallimento delle azioni speciali@\emph{Fallimento delle azioni speciali}}

	\item Un personaggio può votare sé stesso per la condanna a morte sul rogo.
	      \index{Votazione per il rogo@\emph{Votazione per il rogo}}

	\item Il modo in cui vengono generate le tre frasi destinate a un Divinatore è a discrezione dei GM, e non è noto ai giocatori.
	      \index{Divinatore}

	\item Un personaggio che fallisce nell'utilizzare la propria abilità genera comunque un movimento.
	      Ad esempio, un Alcolista che agisce su un personaggio genera comunque un movimento verso quel personaggio.
	      \index{Alcolista}

	\item Il modo con cui vengono decisi quali componenti della Fazione dei Villici diventano Spettri è il seguente:

	      All'inizio della partita, i GM scelgono una successione crescente di interi positivi, detta \emph{successione spettrale}. All'$n$-esima morte di un componente della Fazione dei Villici, se $n$ appartiene alla successione spettrale, quel personaggio diventa uno Spettro. Se $n$ non appartiene alla successione spettrale, invece, quel personaggio non diventa uno Spettro.
	      \index{Successione spettrale@\emph{Successione spettrale}}

	\item Se più personaggi muoiono la stessa notte, le loro morti sono ordinate in modo casuale: in particolare, a causa della successione spettrale, questo può influenzare quali personaggi diventeranno Spettri.
	      \index{Morte@\emph{Morte}}
	      \index{Successione spettrale@\emph{Successione spettrale}}

	\item Ogni morte di un personaggio della Fazione dei Villici è valida per la successione spettrale. In particolare, se il personaggio resuscitato dal Messia è un componente della Fazione dei Villici che non è diventato Spettro alla sua morte, e quindi gioca ancora con la Fazione dei Villici, la sua eventuale seconda morte può appartenere alla successione spettrale, dunque il personaggio ha una seconda possibilità di diventare uno Spettro.
	      \index{Messia}
	      \index{Morte@\emph{Morte}}
	      \index{Successione spettrale@\emph{Successione spettrale}}

	\item Le informazioni iniziali dovute all'appartenenza alla Fazione dei Villani comprendono \emph{tutte e sole} le informazioni di cui il giocatore dovrebbe essere a conoscenza. In particolare, se nelle informazioni iniziali non è presente alcuna informazione riguardante un ruolo, vuol dire che quel ruolo non è assunto da nessun personaggio diverso dal giocatore che sta ricevendo le informazioni.

	      %	\item Le informazioni iniziali dovute all'appartenenza ad un ruolo che conosce i nomi di personaggi che assumono determinati ruoli (ad esempio Massoni, componenti della fazione dei Villani, o componenti della Fazione dei Negromanti) comprendono \emph{tutte e sole} le informazioni di cui il giocatore dovrebbe essere a conoscenza. In particolare, se nelle informazioni iniziali non è presente alcuna informazione riguardante un ruolo, vuol dire che quel ruolo non è assunto da nessun personaggio diverso dal giocatore che sta ricevendo le informazioni.

	      %	Ad esempio, se un Massone non è a conoscenza di altri Massoni, vuol dire che nella partita lui è l'unico Massone in gioco.
	      %	\index{Massone}

	\item Cercare di uccidere un personaggio che nella stessa notte muore anche per altri motivi non costituisce di per sé una causa di fallimento.
	      Per esempio, se contemporaneamente un Lupo e lo Spettro della Morte utilizzano la propria azione speciale su di uno stesso personaggio, hanno entrambi successo.
	      \index{Morte@\emph{Morte}}
	      \index{Fallimento delle azioni speciali@\emph{Fallimento delle azioni speciali}}
	      \index{Lupo}
	      \index{Spettro!Morte}

	\item Se almeno due Lupi cercano di utilizzare la loro abilità su personaggi diversi, la loro abilità non ha effetto, anche nel caso in cui qualcuno di essi fallisca per altre ragioni.
	      \index{Lupo}
	      \index{Fallimento delle azioni speciali@\emph{Fallimento delle azioni speciali}}

	\item Se un Lupo agisce sullo stesso personaggio sul quale un Cacciatore agisce con successo, il Lupo muore anche se dovesse fallire per altre ragioni.
	      \index{Lupo}
	      \index{Cacciatore}
	      \index{Fallimento delle azioni speciali@\emph{Fallimento delle azioni speciali}}

	\item Se $n$ Lupi agiscono sullo stesso personaggio sul quale agiscono con successo $m$ Cacciatori, se $n\le m$, tutti questi $n$ Lupi muoiono e ricevono una notifica di fallimento, e la tagliola di $n$ Cacciatori scelti a caso scatta.
	      Se invece $n>m$, la tagliola di tutti questi $m$ Cacciatori scatta, e $m$ Lupi scelti a caso muoiono e ricevono una notifica di fallimento. I rimanenti Lupi hanno successo, a meno che non falliscano per altre ragioni.
	      \index{Lupo}
	      \index{Cacciatore}
	      \index{Fallimento delle azioni speciali@\emph{Fallimento delle azioni speciali}}

	      %	\item Due Negromanti diversi possono attivare la stessa notte due Incantesimi diversi su due Spettri diversi, se ci sono abbastanza Incantesimi non attivati e abbastanza Spettri.
	      %	\index{Negromante}

	      %	\item Se più Negromanti (morti) tentano di utilizzare il loro potere sullo stesso personaggio, tutti tranne uno ricevono una notifica di fallimento.
	      %	\index{Negromante}

	\item Se $n$ fra Negromanti e Spettri tentano di attivare lo stesso Incantesimo presente $k<n$ volte dal Grimorio principale su Spettri diversi, $n-k$ scelti casualmente ricevono una notifica di fallimento.

	      Il numero di successi non è necessariamente uguale al numero di Spettri che ottengono quel potere quella notte: se Negromante A e Spettro X tentano di attivare un Incantesimo presente in $2$ copie sul Grimorio principale sullo Spettro X, e Negromante B tenta di attivare lo stesso Incantesimo sullo Spettro Y, potrebbe succedere che Negromante A e Spettro X abbiano successo, e Negromante B fallisca; in tal caso, entrambe le copie dell'Incantesimo in questione vengono rimosse dal Grimorio principale, e solo lo Spettro X ottiene il potere.
	      \index{Negromante}
	      \index{Spettro}

	\item Se più Negromanti o Spettri cercano di attivare Incantesimi diversi sullo stesso Spettro, tutti hanno successo, e gli Incantesimi vengono attivati in ordine casuale. In particolare lo Spettro riceverà tutte le notifiche, e l'Incantesimo effettivamente attivo sullo Spettro sarà l'ultimo attivato.
	      \index{Negromante}
	      \index{Spettro}

	\item Un Negromante può cambiare l'Incantesimo attivato su di uno Spettro nella stessa notte in cui quest'ultimo agisce. Se lo Spettro ha utilizzato con successo un potere attivabile ogni due notti, non potrà utilizzare il nuovo potere la notte successiva.
	      \index{Negromante}

	\item Un Fantasma morto, pur avendo un potere, non è uno Spettro. In particolare, può essere resuscitato dal Messia.
	      \index{Fantasma}
	      \index{Messia}

	\item Se un Fantasma utilizza un potere utilizzabile una volta ogni due notti, non può utilizzare abilità o poteri la notte successiva.
	      \index{Fantasma}

	\item Il Fantasma ottiene un elenco di poteri, non di Incantesimi. In particolare, tale elenco non può contenere "Vita", siccome tale Incantesimo non assegna alcun potere allo spettro su cui è attivato.
	      \index{Fantasma}
	      %\item Se un Fantasma utilizza la sua abilità, non agisce su alcun personaggio e non genera alcun movimento.
	      %\index{Fantasma}

	\item Se l'Esorcista e lo Spettro dell'Occultamento agiscono sulla stessa persona, lo Spettro fallirà, mentre l'Esorcista avrà successo. Se invece lo Stregone e lo Spettro dell'Occultamento agiscono sulla stessa persona, fallirà uno dei due scelto casualmente.
	      \index{Esorcista}
	      \index{Spettro!Occultamento}
	      \index{Stregone}

	\item Se due o più Fattucchiere agiscono su uno stesso personaggio, i loro effetti vengono applicati in un ordine casuale. L'ultimo effetto applicato sarà quello effettivo. Questo non costituisce una causa di fallimento.
	      \index{Fattucchiera}

	\item Se $n$ Spettri della Confusione agiscono su uno stesso personaggio, il colore dell'aura di tale personaggio apparirà invertito se e solo se $n$ è dispari.
	      \index{Spettro!Confusione}

	\item I ruoli influenzati dalla Fattucchiera sono i seguenti: Divinatore, Espansivo, Mago, Medium, Trasformista, Veggente, Diavolo. In particolare, il Cacciatore, la Guardia, lo Spettro della Morte, lo Spettro della Telepatia, e lo Spettro della Visione non sono influenzati.
	      \index{Divinatore}
	      \index{Medium}
	      \index{Mago}
	      \index{Trasformista}
	      \index{Veggente}
	      \index{Diavolo}
	      \index{Espansivo}
	      \index{Fattucchiera}

	\item I ruoli influenzati dallo Spettro dell Confusione sono i seguenti: Espansivo, Veggente.
	      \index{Veggente}
	      \index{Espansivo}
	      \index{Spettro!Confusione}

	\item Se una Fattucchiera e un Trasformista agiscono contemporaneamente sullo stesso personaggio, il Trasformista assumerà il ruolo scelto dalla Fattucchiera, se questo è un ruolo appartenente alla Fazione dei Villici. In particolare, questo può far assumere al Trasformista un ruolo che non era presente inizialmente nella partita.
	      \index{Fattucchiera}
	      \index{Trasformista}

	\item Il Trasformista ottiene, oltre al ruolo e all'abilità, anche l'aura ed eventualmente la proprietà di essere mistico di un personaggio su cui ha agito con successo. Non ottiene però eventuali effetti assegnati all'inizio della partita, come ad esempio le informazioni del Divinatore o l'elenco di abilità dell'Apprendista.
	      \index{Trasformista}

	\item Se un Trasformista usa la propria abilità su un Trasformista morto che aveva già acquisito un nuovo ruolo, egli acquisisce a sua volta quello stesso ruolo. %Lo stesso vale per il Necrofilo.
	      \index{Trasformista}
	      %	\index{Necrofilo}

	\item Se un Trasformista usa la propria abilità su un personaggio con un'abilità utilizzabile una volta per partita (ad esempio un Messia), potrà utilizzare l'abilità del ruolo in cui si è trasformato, anche se il personaggio su cui ha agito come Trasformista aveva già utilizzato la propria abilità.
	      \index{Trasformista}
	      \index{Messia}

	\item La Guardia protegge il personaggio scelto solamente dalle abilità di Cacciatore, Assassino, Lupo, Sequestratore. Altre abilità o poteri utilizzate sul personaggio scelto non sono influenzate dall'abilità della Guardia.
	      \index{Guardia}
	      \index{Lupo}
	      \index{Assassino}
	      \index{Cacciatore}
	      \index{Sequestratore}

	\item La Guardia può bloccare un Apprendista che utilizza l'abilità di Cacciatore, Assassino o Sequestratore.
	      \index{Guardia}
	      \index{Apprendista}
	      \index{Assassino}
	      \index{Cacciatore}
	      \index{Sequestratore}

	\item Se un Apprendista utilizza con successo l'abilità di un ruolo che può agire ogni due notti, non può agire la notte successiva. Se utilizza con successo l'abilità di un ruolo che può agire una volta a partita, non può più utilizzare alcuna abilità.
	      \index{Apprendista}

	\item Se un giocatore X con il ruolo di Apprendista utilizza con successo l'abilità di Espansivo su un personaggio Y, ad Y verrà noticato che X è un Espansivo, non un Apprendista.
	      \index{Apprendista}
	      \index{Espansivo}

	\item Lo Spettro dell'Invisibilità agisce sul personaggio di cui viene nascosto il movimento. In particolare, se lo Spettro nasconde il movimento di un personaggio su cui ha agito anche un Esorcista, il suo potere non ha effetto. Se invece lo Spettro nasconde il movimento di personaggio diretto su qualcuno su cui ha agito anche un Esorcista, il suo potere funziona normalmente.
	      \index{Esorcista}
	      \index{Spettro!Invisibilità}

	      %	\item Se uno Spettro dell'Illusione fa comparire l'illusione di un Voyeur nella casa di un personaggio su cui tale Voyeur agisce, il Voyeur compare nella lista dei personaggi visti da sé stesso.
	      %	\index{Voyeur}
	      %	\index{Spettro!Illusione}

	      %	\item L'illusione generata da uno Spettro dell'Illusione viene contata fra i personaggi visti da una Guardia, anche se appartiene alla Guardia stessa.
	      %	\index{Guardia}
	      %	\index{Spettro!Illusione}

	\item Se un movimento viene cancellato, è come se il giocatore a cui viene annullato il movimento non si fosse mai mosso. Questo influenza le abilità di Guardia, Stalker, Voyeur, Assassino, Sequestratore, Stregone. In particolare, l'abilità del Cacciatore ed il potere dello Spettro della Visione non sono influenzati.
	      \index{Spettro!Invisibilità}
	      \index{Assassino}
	      \index{Guardia}
	      \index{Sequestratore}
	      \index{Stalker}
	      \index{Stregone}
	      \index{Voyeur}

	\item Se un personaggio X agisce sullo stesso personaggio sul quale agisce con successo uno Stregone, e tale movimento viene cancellato, allora X viene comunque bloccato dallo Stregone, ma lo Stregone non vede X.
	      \index{Spettro!Invisibilità}
	      \index{Stregone}

	\item Se sullo stesso personaggio X agiscono un Sequestratore e uno Spettro dell'Invisibilità, il Sequestratore ha successo (salvo che non fallisca per altre ragioni) ma non blocca X, né ne vede un eventuale movimento.
	      \index{Spettro!Invisibilità}
	      \index{Sequestratore}

	      %	\item Il fatto che il movimento generato dallo Spettro dell'Illusione sia fittizio influenza solamente l'abilità dell'Assassino. Infatti, se l'Assassino agisce su un personaggio verso cui si dirige un movimento fittizio, e questo viene selezionato casualmente dall'abilità, non morirà nessuno. Questo è vero anche se il personaggio di cui viene generata l'illusione agisce effettivamente verso il personaggio su cui agisce l'Assassino.
	      %	\index{Assassino}
	      %	\index{Spettro!Illusione}

	\item Non indovinare il movimento non costituisce causa di fallimento nell'utilizzo dell'abilità dell'Assassino.
	      \index{Assassino}
	      \index{Fallimento delle azioni speciali@\emph{Fallimento delle azioni speciali}}

	\item L'Assassino, per usare la sua abilità, deve scegliere due personaggi. Agisce solo sul primo (quello che intende uccidere), e genera un movimento solo verso tale personaggio. In particolare, il secondo personaggio scelto può essere sè stesso.
	      \index{Assassino}

	      %	\item Se più Assassini agiscono contemporaneamente su uno stesso personaggio, i sorteggi delle vittime avvengono in maniera indipendente.
	      %	In particolare è possibile che più Assassini uccidano il medesimo personaggio.
	      %	\index{Assassino}

	      %	\item Un Assassino può essere ucciso da un altro Assassino, se entrambi agiscono contemporaneamente su uno stesso personaggio. In particolare, due tali Assassini potrebbero uccidersi a vicenda.
	      %	\index{Assassino}

	      %	\item Se nella stessa notte un Messia e un Negromante cercano di agire sullo stesso personaggio morto non Spettro, quel personaggio viene resuscitato dal Messia e non risvegliato come Spettro. In particolare, il Negromante riceve una notifica di fallimento.
	\item Se nella stessa notte $n$ Messia ed $m$ Negromanti cercano di agire sullo stesso Villico morto, allora agiscono in un'ordine casuale. Se il primo ad agire è un Messia, tutti i Negromanti falliscono. Se il primo ad agire è un Negromante, tutti i Messia falliscono.
	      \index{Messia}
	      \index{Negromante}

	      %	\item Se al momento della morte del Fantasma non sono più disponibili Incantesimi attivabili (diversi da Vita), il Fantasma diventa uno Spettro vuoto. Se più Fantasmi muoiono contemporaneamente e non sono disponibili abbastanza Incantesimi non attivati per tutti loro, vengono scelti casualmente i Fantasmi su cui viene attivato un Incantesimo.
	      %	\index{Fantasma}
	      %	
	      %	\item I Negromanti hanno sempre la priorità sui Fantasmi morti di notte: prima avviene l'attivazione di un Incantesimo ad opera dei Negromanti, e poi i Fantasmi appena morti ottengono casualmente uno degli Incantesimi ancora disponibili (se ve ne sono).
	      %	\index{Fantasma}
	      %	\index{Negromante}
	      %
	      %	\item Se uno Spettro smette di essere uno Spettro (ad esempio perché ha utilizzato il potere della Morte), non può più avere incantesimi attivati su di sé, ma può essere resuscitato dal Messia. Continua a giocare per la Fazione dei Negromanti.
	      %	\index{Spettro!Morte}
	      %	\index{Messia}
	      %	
	      %	\item L'effetto dello Spettro della Diffamazione viene applicato \emph{dopo} l'effetto dello Spettro dell'Amnesia e dello Spettro dell'Assoluzione.
	      %	\index{Spettro!Amnesia}
	      %	\index{Spettro!Assoluzione}
	      %	\index{Spettro!Diffamazione}
	      %	
	      %	\item La Spia scopre il voto del personaggio su cui agisce \emph{dopo} che sono stati applicati gli effetti dei poteri degli Spettri dell'Amnesia, dell'Assoluzione e della Diffamazione.
	      %	\index{Spia}
	      %	\index{Spettro!Amnesia}
	      %	\index{Spettro!Assoluzione}
	      %	\index{Spettro!Diffamazione}

	\item Lo Spettro della Telepatia ottiene una copia esatta di tutti i messaggi che riceve il personaggio su cui ha utilizzato il proprio potere. Questi comprendono la persona su cui ha agito quest'ultimo, o un'eventuale notifica di fallimento.
	      \index{Spettro!Telepatia}

	\item Non indovinare il ruolo non costituisce causa di fallimento nell'utilizzo del potere dello Spettro della Morte, né di quello dello Spettro della Telepatia. In particolare, l'Incantesimo della Morte viene disattivato anche se lo Spettro della Morte non ha indovinato il ruolo.
	      \index{Spettro!Morte}
	      \index{Spettro!Telepatia}
	      \index{Fallimento delle azioni speciali@\emph{Fallimento delle azioni speciali}}

	\item Lo Spettro della Vita è considerato un personaggio vivo. In particolare, può utilizzare l'abilità data dal suo ruolo, se ne possiede una. Un Negromante non può utilizzare la sua abilità per cambiare l'Incantesimo attivato su di esso.
	      \index{Spettro!Vita}

	\item Se lo Spettro della Vita o il personaggio resuscitato dal Messia possedevano un'abilità utilizzabile una sola volta per partita, questa può essere utilizzata dal personaggio tornato in vita solo se non è stata utilizzata precedentemente.
	      \index{Spettro!Vita}
	      \index{Messia}

	      %	\item Se un Negromante viene resuscitato dal Messia, e successivamente muore, avrà a disposizione il suo potere solo se non lo ha utilizzato precedentemente.
	      %	\index{Messia}
	      %	\index{Negromante}

	      %	\item Il potere dell'Occultamento ha effetto solo sulle abilità dei personaggi vivi. In particolare, non ha effetto sui poteri degli altri Spettri, né sul potere di un Negromante morto.
	      %	\index{Negromante}
	      %	\index{Spettro}
	      %	\index{Spettro!Occultamento}
	      %	\index{Fallimento delle azioni speciali@\emph{Fallimento delle azioni speciali}}

	\item L'abilità dell'Alcolista non ha mai successo. In particolare, ogni volta che l'Alcolista utilizza il suo potere riceve una notifica di fallimento.
	      \index{Alcolista}

	\item Un giocatore della Fazione dei Villici non può diventare uno Spettro se nella stessa notte la Fazione dei Villani perde.

	\item Se ad un certo punto della partita le condizioni di vittoria di entrambe le fazioni sono verificate, entrambe le fazioni vincono. Questo può succedere se ad esempio l'ultimo Lupo va dall'ultimo Villico, che è un Apprendista, che utilizza l'abilità di Assassino sul Lupo indovinando che agisce sull'Apprendista, nel qual caso tutti i Villici e tutti i Lupi muoiono.

	      %	\index{Elezione del Sindaco@\emph{Elezione del Sindaco}}

	      %	\item Se durante lo stesso giorno un personaggio viene eletto Sindaco e condannato a morte sul rogo, immediatamente dopo viene nominato casualmente un nuovo Sindaco (non vi è stato tempo per la nomina di un successore).
	      %	\index{Elezione del Sindaco@\emph{Elezione del Sindaco}}

	      %	\item Se più personaggi ottengono il massimo numero di voti per il rogo e quello scelto casualmente per essere ucciso è stato protetto dall'Avvocato, nessuno muore e non viene resa pubblica l'identità del personaggio protetto dall'Avvocato.
	      %	\index{Avvocato}
	      %	\index{Votazione per il rogo@\emph{Votazione per il rogo}}

	      %	\item Se più Fattucchiere agiscono sulla stessa persona, ciascuna cambia il colore dell'aura. Pertanto, il colore dell'aura percepito durante quella notte è quello corretto se le Fattucchiere in questione sono in numero pari, mentre è quello sbagliato se le Fattucchiere sono in numero dispari.
	      %	\index{Fattucchiera}

	      %	\item Agli occhi di uno Stalker, è come se il personaggio di cui è stata generata l'illusione avesse agito \emph{solo} sul personaggio scelto. Nel caso in cui il personaggio scelto coincida con quello di cui è stata generata l'illusione, uno Stalker che usa il suo potere su tale personaggio riceve informazioni come se questi non avesse agito.
	      %	\index{Stalker}
	      %	\index{Spettro!Illusione}
	      %	
	      %	\item Agli occhi di Custode del cimitero, Guardia, e Voyeur, è come se il personaggio di cui è stata generata l'illusione avesse agito \emph{anche} sul personaggio scelto. Nel caso in cui il personaggio scelto coincida con quello di cui è stata generata l'illusione, questi non compare nell'elenco (né nel numero) di personaggi visti da eventuali Custodi del cimitero, Guardie del corpo, o Voyeur che usano il proprio potere su tale personaggio.
	      %	\index{Custode del cimitero}
	      %	\index{Guardia}
	      %	\index{Voyeur}
	      %	\index{Spettro!Illusione}

	      %	\item I ruoli influenzati dal potere soprannaturale dell'Illusione sono i seguenti: Custode del cimitero, Guardia, Stalker, Voyeur.
	      %	\index{Custode del cimitero}
	      %	\index{Guardia}
	      %	\index{Stalker}
	      %	\index{Voyeur}
	      %	\index{Spettro!Illusione}

	      %	\item I poteri soprannaturali della Corruzione e della Morte possono essere assegnati ad un Trasformista che abbia in precedenza utilizzato con successo il suo potere su un personaggio mistico con aura bianca. Allo stesso modo, uno Spettro della Corruzione ha successo su un tale Trasformista, a meno che non intervengano altre cause di fallimento.
	      %	\index{Trasformista}
	      %	\index{Spettro!Corruzione}

	      %	\item Se un Necrofilo usa il proprio potere su un Necrofilo morto che non aveva usato con successo il proprio potere (a prescindere dal fatto che l'abbia usato fallendo o che non l'abbia usato affatto), ne assume comunque il ruolo ed ottiene nuovamente la possibilità di usare il proprio potere.
	      %	\index{Necrofilo}

	      %	\item Se più Negromanti cercano di creare degli Spettri, falliscono come succede nel caso dei Lupi, a meno che scelgano tutti lo stesso personaggio e selezionino tutti lo stesso potere soprannaturale.
	      %	\index{Negromante}

	      %	\item Se un Ipnotista morto viene resuscitato dal Messia, tutti coloro che si trovavano sotto il controllo dell'Ipnotista al momento della sua morte sono nuovamente sotto il suo controllo, a meno che nel frattempo la loro mente sia stata controllata da un altro Ipnotista o altri effetti.
	      % 
	      %	Se un personaggio sotto il controllo di un Ipnotista muore e viene successivamente resuscitato, torna ad essere sotto il controllo dell'ultimo Ipnotista ad aver agito su di esso.
	      %	\index{Messia}
	      %	\index{Ipnotista}
	      %	
	      %	\item Se un personaggio smette di essere sotto il controllo di un Ipnotista (ad esempio perché un altro Ipnotista usa il suo potere su di esso), l'Ipnotista non riceve alcuna notifica.
	      %	\index{Ipnotista}
	      % 
	      %	\item Se più Ipnotisti agiscono sullo stesso personaggio durante la stessa notte, la mente di quest'ultimo viene controllata da tutti gli Ipnotisti in un ordine casuale, e perciò risulterà infine essere sotto il controllo soltanto dell'ultimo ad aver agito. Gli altri Ipnotisti non ricevono alcuna notifica di fallimento.
	      %	\index{Ipnotista}
	      %	
	      %	\item I personaggi su cui un Ipnotista agisce con successo non vengono informati di essere sotto il suo controllo. Un personaggio può essere sotto il controllo di un solo Ipnotista per volta, e precisamente l'ultimo ad aver agito su di esso.
	      %	\index{Ipnotista}
	      % 
	      %	\item Il Fantasma non può ottenere i poteri soprannaturali della Corruzione e della Morte (perché non è mistico).
	      %	\index{Fantasma}
	      %	\index{Spettro!Corruzione}
	      %	\index{Spettro!Morte}
	      % 
	      %	\item Il poteri dell'Ipnotista e dello Scrutatore e i poteri soprannaturali dell'Amnesia e dell'Ipnosi non incidono sull'elezione del Sindaco, solo sul voto per il rogo.
	      %	\index{Ipnotista}
	      %	\index{Scrutatore}
	      %	\index{Spettro!Amnesia}
	      %	\index{Spettro!Ipnosi}
	      %	\index{Elezione del Sindaco@\emph{Elezione del Sindaco}}
	      % 
	      %	\item Al fine di determinare la validità della votazione, la percentuale di votanti per il rogo è calcolata dopo aver applicato gli effetti dei poteri che interagiscono con la votazione (Ipnotista, Scrutatore, poteri soprannaturali dell'Amnesia e dell'Ipnosi).
	      %	Per esempio, se su 10 personaggi vivi ve ne sono 5 sotto il controllo dell'Ipnotista, e l'Ipnotista non vota, allora il quorum del 50\% non è raggiunto perché l'Ipnotista e i personaggi sotto il suo controllo non votano.
	      %	\index{Ipnotista}
	      %	\index{Scrutatore}
	      %	\index{Spettro!Amnesia}
	      %	\index{Spettro!Ipnosi}
	      %	\index{Votazione per il rogo@\emph{Votazione per il rogo}}
	      %	

	      %	\item Le interazioni fra Spettro dell'Amnesia, Spettro dell'Ipnosi, Ipnotisti, e Scrutatori funzionano nel modo seguente.
	      %	
	      %	Se lo Spettro dell'Amnesia ha agito con successo su un personaggio, durante la votazione del giorno successivo risulterà come se tale personaggio non avesse votato.
	      %	
	      %	Se lo Spettro dell'Ipnosi ha agito con un successo su un personaggio, durante la votazione del giorno successivo risulterà come se tale personaggio avesse votato come indicato dallo Spettro, a meno che su tale personaggio non abbia agito con successo anche lo Spettro dell'Amnesia.
	      %	
	      %	In seguito, i voti delle persone controllate da un Ipnotista vengono indirizzati sul personaggio al quale è diretto in quel momento il voto dell'Ipnotista (e non necessariamente sul personaggio selezionato nell'interfaccia). Per esempio, se uno Spettro dell'Ipnosi agisce con successo su un Ipnotista, i voti dei personaggi controllati da quest'ultimo saranno indirizzati sul personaggio selezionato dallo Spettro, non su quello scelto dall'Ipnotista. Lo stesso accade se l'Ipnotista in questione è sotto il controllo di un altro Ipnotista. Un personaggio su cui ha agito con successo lo Spettro dell'Amnesia o lo Spettro dell'Ipnosi è temporaneamente considerato non essere sotto il controllo dell'Ipnotista.
	      %	
	      %	Infine, i voti diretti su un personaggio su cui ha agito con successo uno Scrutatore vengono indirizzati sul personaggio al quale è diretto in quel momento il voto dello Scrutatore (e non necessariamente sul personaggio selezionato nell'interfaccia).
	      %	
	      %	Nel caso in cui non ci sia un modo coerente di risolvere le dipendenze, o che ce ne sia più di uno (per esempio perché due Ipnotisti sono vicendevolmente sotto il controllo l'uno dell'altro, o perché tre Scrutatori votano ciclicamente il personaggio su cui ha agito con successo un altro Scrutatore), i poteri vengono risolti analogamente al modo in cui sono gestiti i fallimenti ``di terza categoria'' (vedi Sezione \ref{fallimento}).
	      %	
	      %	\index{Ipnotista}
	      %	\index{Scrutatore}
	      %	\index{Spettro!Amnesia}
	      %	\index{Spettro!Ipnosi}
	      %	\index{Votazione per il rogo@\emph{Votazione per il rogo}}
	      %		
	      %	\item Lo Spettro dell'Ipnosi ha la precedenza sull'Ipnotista. Se lo Spettro con il potere dell'Ipnosi agisce su un personaggio controllato da un Ipnotista, il giorno seguente il personaggio vota come indicato dallo Spettro. Tuttavia il personaggio rimane sotto il controllo dell'Ipnotista.
	      %	\index{Spettro!Ipnosi}
	      %	\index{Ipnotista}

	      %	\item Se un personaggio su cui ha agito lo Spettro con il potere dell'Amnesia vota per un personaggio su cui ha agito uno Scrutatore, nel computo dei voti risulta come se il primo personaggio non avesse votato.
	      %	\index{Scrutatore}
	      %	\index{Spettro!Amnesia}
	      % 
	      %	\item Se più Scrutatori agiscono su uno stesso personaggio, viene scelto casualmente uno di essi, e tutti i voti diretti a quel personaggio vengono spostati sulla persona votata da tale Scrutatore.
	      %	\index{Scrutatore}
	      %
	      %	\item Se più Scrutatori agiscono durante la stessa notte, il responso del rogo successivo viene calcolato applicando gli effetti del potere di ciascuno di essi uno dopo l'altro, in un ordine casuale.
	      %	
	      %	Per esempio, se lo Scrutatore A agisce su un personaggio X, lo Scrutatore B agisce su un personaggio Y, lo Scrutatore C agisce su un personaggio Z, e poi A vota Y, B vota X, C vota X, e viene l'ordine di azione è C-B-A, allora tutti i voti diretti a Z vengono spostati su X, poi tutti i voti diretti a Y vengono spostati su X, poi tutti i voti diretti a X (inclusi quelli appena spostati) vengono spostati su Y. Il risultato netto è che per chiunque abbia votato X, Y, Z, il voto effettivo è diretto a Y.
	      %	\index{Scrutatore}
	      % 
	      %	\item Se uno Scrutatore e/o gli Spettri con il potere dell'Amnesia e/o dell'Ipnosi agiscono su un personaggio che muore durante la stessa notte, l'effetto del loro potere è cancellato. 
	      %	Nel caso dello Spettro dell'Ipnosi, la stessa cosa succede se a morire è il personaggio a cui era diretto il voto. Questa eventualità non è considerata un motivo di fallimento.
	      %	\index{Ipnotista}
	      %	\index{Scrutatore}
	      %	\index{Spettro!Amnesia}
	      %	\index{Spettro!Ipnosi}
	      %	\index{Fallimento delle azioni speciali@\emph{Fallimento delle azioni speciali}}

	      %	\item Se una o più Fattucchiere e lo Spettro con il potere della Confusione agiscono sullo stesso personaggio, chiunque cerchi di ottenere informazioni sul colore dell'aura di tale personaggio ottiene il responso previsto dall'azione dello Spettro con il potere della Confusione se il numero di Fattucchiere è pari, e quello opposto se è dispari.
	      %	\index{Fattucchiera}
	      %	\index{Spettro!Confusione}

	      %	\item Se un Sequestratore agisce su un personaggio su cui agisce anche uno Stalker, quest'ultimo riceve informazioni come se il personaggio sequestrato non avesse agito. Se un personaggio sequestrato ed un Voyeur agiscono su uno stesso personaggio, il Voyeur riceve informazioni come se il personaggio sequestrato non avesse agito. Analogamente, un personaggio sequestrato non viene contato fra le persone viste da una Guardia.
	      %%	\index{Custode del cimitero}
	      %	\index{Guardia}
	      %	\index{Stalker}
	      %	\index{Voyeur}
	      %	\index{Sequestratore}

	      %	\item I personaggi su cui agisce con successo un Sequestratore non compaiono nell'elenco di quelli tra cui viene sorteggiata la vittima dell'Assassino.
	      %	%  Se un personaggio fallisce nell'utilizzo del potere a causa di un Sequestratore, tale personaggio non compare nell'elenco di quelli fra cui viene sorteggiata la vittima di un eventuale Assassino che ha agito sullo stesso personaggio su cui avrebbe agito il personaggio sequestrato.
	      %	\index{Assassino}
	      %	\index{Sequestratore}

	      %	\item Se lo Spettro della Corruzione agisce su un personaggio che muore quella stessa notte (per esempio ucciso da un Lupo o da un Assassino), il personaggio in questione non diventa un Negromante. Lo Spettro della Corruzione riceve una notifica di fallimento.
	      %	\index{Assassino}
	      %	\index{Lupo}
	      %	\index{Spettro!Corruzione}

	      %	\item Se un Trasformista usa il suo potere con sucesso su un Ipnotista, in ogni caso non viene conteggiato fra gli Ipnotisti vivi ai fini della creazione dello Spettro con il Potere dell'Ipnosi. In particolare, se tutti gli Ipnotisti appartenenti alla Fazione dei Negromanti muoiono, l'ultimo di essi a morire viene risvegliato come Spettro con il potere dell'Ipnosi. Analogamente, un Trasformista non potrà in nessun caso essere risvegliato come Spettro con il Potere dell'Ipnosi.
	      %	\index{Trasformista}
	      %	\index{Spettro!Ipnosi}

	      %	\item Il voto eventualmente aggiunto dallo Scrutatore conta ai fini del raggiungimento del quorum.
	      %	\index{Votazione per il rogo@\emph{Votazione per il rogo}}
	      %	\index{Scrutatore}
	      %	
	      %	\item Se il voto aggiunto dallo Scrutatore è identico al voto espresso dal personaggio su cui è stato usato il potere dello Scrutatore, entrambi i voti vengono conteggiati per determinare il condannato a morte.
	      %	Tuttavia, nella lista dei votanti, il personaggio su cui è stato usato il potere dello Scrutatore compare solo una volta.
	      %	\index{Votazione per il rogo@\emph{Votazione per il rogo}}
	      %	\index{Scrutatore}
	      %	
	      %	\item Se più Ipnotisti muoiono la stessa notte, solamente uno di essi (scelto casualmente) viene risvegliato come Spettro con il potere dell'Ipnosi, sempre che tale potere non sia ancora stato assegnato.
	      %	\index{Ipnotista}
	      %	\index{Spettro!Ipnosi}

	      %	\item Se lo Spettro con il potere dell'Amnesia e lo Spettro con il potere della
	      %	Duplicazione agiscono contemporaneamente su di un personaggio, il voto di quel
	      %	personaggio è comunque annullato. In altre parole, il potere dell'Amnesia
	      %	annulla il potere della Duplicazione.
	      %	\index{Spettro!Amnesia}
	      %	\index{Spettro!Duplicazione}
	      %	
	      %	\item Quando lo Spettro con il potere della Duplicazione agisce, il villaggio
	      %	non viene messo al corrente dell'identità del personaggio che ha beneficiato del
	      %	potere. Per esempio, se il personaggio X viene votato da A, B e C, e durante la
	      %	notte precedente lo Spettro aveva usato il suo potere su C, allora il server
	      %	riporterà un messaggio simile a ``X è stato votato da A, B e C, ricevendo un
	      %	totale di 4 voti''.
	      %	\index{Votazione per il rogo@\emph{Votazione per il rogo}}
	      %	\index{Spettro!Duplicazione}
	      %	
	      %	\item Lo Spettro con il potere dell'Illusione che agisce su un personaggio può dirigere l'illusione dal personaggio stesso. In questo caso, un eventuale Stalker che utilizzi a sua volta il suo potere sullo stesso personaggio riceve informazioni come se tale personaggio non avesse agito.
	      %	\index{Spettro!Illusione}
	      %	\index{Stalker}
	      %	
	      %	\item I Negromanti hanno sempre la priorità sugli Ipnotisti morti di notte. Se nella stessa notte muore un Ipnotista, e i Negromanti tentano di risvegliare un morto come Spettro con il potere dell'Ipnosi, a meno dell'intervento di altre cause di fallimento questi hanno successo, e l'Ipnotista non viene risvegliato come Spettro.
	      %	\index{Negromante}
	      %	\index{Ipnotista}
	      %	\index{Spettro!Ipnosi}

	      %	\item Se in notti diverse un personaggio muore, viene risvegliato come Spettro, resuscitato dal Messia e la sua anima viene letta dal Diavolo, il Diavolo scopre che il personaggio è uno Spettro, ma non scopre che potere soprannaturale ha.
	      %	\index{Spettro}
	      %	\index{Messia}
	      %	\index{Diavolo}
	      %	
	      %	\item Se nella stessa notte l'ultimo Ipnotista (appartenente alla Fazione dei Negromanti) muore e i Negromanti risvegliano un personaggio come Spettro con il potere della Morte, allora l'Ipnotista morto viene comunque risvegliato come Spettro con il potere dell'Ipnosi.
	      %	\index{Ipnotista}
	      %	\index{Spettro!Ipnosi}

	      %	\item Se lo Scrutatore e lo Spettro con il potere dell'Amnesia agiscono entrambi su uno stesso personaggio, il voto originariamente espresso da quel personaggio viene cancellato (per effetto del potere dell'Amnesia), ma il voto aggiunto dallo Scrutatore rimane valido. In particolare, dal risultato della votazione non è possibile distinguere il caso in cui questo accade dal caso in cui il personaggio scelto ha votato il personaggio indicato
	      %	dallo Scrutatore (a meno che abbiano agito più Scrutatori).
	      %	\index{Scrutatore}
	      %	\index{Spettro!Amnesia}
	      %	
	      %	\item Se il Trasformista utilizza il proprio potere diventando un Ipnotista, al termine della notte in cui questo accade egli smette di essere affetto dai poteri di Ipnotisti, Spettro con il potere dell'Amnesia e Spettro con il potere dell'Ipnosi. In particolare, se era stato precedentemente ipnotizzato da un qualche Ipnotista, smette di essere ipnotizzato (ma l'Ipnotista non riceve alcuna notifica).
	      %	
	      %	Se, nella notte in cui il Trasformista diventa Ipnotista, sul Trasformista agiscono Ipnotisti, Spettro con il potere dell'Amnesia, oppure Spettro con il potere dell'Ipnosi, questi ultimi non ricevono una notifica di fallimento a causa della trasformazione in atto, sebbene il loro potere non abbia effetto (perché al termine della notte il Trasformista diviene immune a tali poteri).
	      %	\index{Trasformista}
	      %	\index{Ipnotista}
	      %	\index{Spettro!Ipnosi}
	      %	\index{Spettro!Amnesia}
	      %	\index{Fallimento delle azioni speciali@\emph{Fallimento delle azioni speciali}}

	      %	\item Ogni responso casuale dovuto all'azione dello Spettro con il potere della Confusione è generato scegliendo con la stessa probabilità una delle possibili risposte.
	      %	Il responso casuale non è influenzato in alcun modo dalla composizione del villaggio né dalle informazioni su di esso.
	      %	
	      %	Per esempio, il Diavolo e il Medium possono ricevere come risposta un ruolo non rappresentato o un ruolo che non può essere presente (come un Lupo dopo che la sua fazione è già stata esiliata).
	      %	\index{Spettro!Confusione}
	      %	\index{Diavolo}
	      %	\index{Medium}
	      %	
	      %	\item Personaggi diversi che cercano di ottenere informazioni su un personaggio su cui ha agito lo Spettro con il potere della Confusione ricevono informazioni casuali indipendenti.
	      %	Per esempio, se contemporaneamente due Veggenti e tale Spettro agiscono sullo stesso personaggio, i due Veggenti possono ottenere responsi differenti.
	      %	\index{Spettro!Confusione}
	      %	\index{Veggente}

	      %	\item Se uno Stregone utilizza il proprio potere su un membro della Fazione dei Negromanti, tale personaggio riceve la relativa notifica ma non può scegliere di bere la pozione. In particolare, lo Stregone viene a sapere che la pozione non è stata bevuta.
	      %	\index{Stregone}
	      %	
	      %	\item Se più Stregoni utilizzano contemporaneamente il proprio potere su uno stesso personaggio, tale personaggio riceve un'unica notifica recante l'offerta di una pozione. Successivamente la sua scelta viene comunicata a tutti gli Stregoni che hanno utilizzato il proprio potere su di lui.
	      %	\index{Stregone}

	      %	\item Fantasmi ed Ipnotisti possono diventare Spettri anche dopo che il potere soprannaturale della Morte è stato assegnato.
	      %	\index{Fantasma}
	      %	\index{Ipnotista}
	      %	\index{Spettro!Morte}
	      %	
	      %	\item Ai fini del raggiungimento del Quorum, l'effetto di uno Scrutatore consiste nella modifica del numero di voti espressi, ma non del totale voti possibili. In particolare, è possibile avere il 110\% dei votanti ad un elezione al rogo.
	      %	\index{Scrutatore}
	      %	\index{Votazione per il rogo@\emph{Votazione per il rogo}}

\end{enumerate}

\printindex

\end{document}
