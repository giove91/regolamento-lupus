\documentclass[a4paper,10pt]{article}


\usepackage[utf8]{inputenc}
\usepackage[italian]{babel}
\usepackage[T1]{fontenc}
\usepackage[dvips]{graphicx}
\usepackage{amsmath}
\usepackage{amsthm}
\usepackage{fancyhdr}
\usepackage{amsfonts}
\usepackage{amssymb}

\topmargin -1cm
\oddsidemargin -0.5cm
%\evensidemargin	-1cm
\textwidth 17cm


\newcommand{\smallspace}{\vskip0.3cm}

% Title Page
\title{Lupus in tempo reale\\ Regolamento della quarta edizione}
\author{Alessandro Iraci, Giovanni Paolini, Leonardo Tolomeo}

\begin{document}
\maketitle


\section{Introduzione}

\subsection{Cos'è Lupus in tempo reale?}

\emph{Lupus in tempo reale} è una variante dei più tradizionali \emph{Lupus in Tabula} e \emph{Mafia}, caratterizzata da un maggiore coinvolgimento e da interazioni più sofisticate tra i giocatori.
La principale differenza consiste nel ritmo del gioco: la partita si estende infatti su diversi giorni, in modo aderente all'ambientazione.
Sono premiate dalla dinamica di gioco la logica, la capacità di bluffare (e di smascherare i bluff altrui), la capacità di saper coordinare un’azione di gruppo, il carisma e le velleità leaderistiche.

\emph{Lupus in tempo reale} è stato inventato nel 2010, ed è stato giocato una volta all'anno fino al 2012 sotto la supervisione di Francesco Guatieri come principale Game Master.
Questo regolamento coincide in buona parte con quello della terza edizione, scritto da Andrea Caleo.
I cambiamenti sono stati discussi e stabiliti da Alessandro Iraci, Luca Ghidelli, Giovanni Paolini e Leonardo Tolomeo.
I Game Master che gestiranno la partita saranno Giovanni Mascellani e Giovanni Paolini (qualcun altro?).


\subsection{Partecipanti}

Possono partecipare:
\begin{itemize}
 \item gli studenti della Scuola;
 \item altre persone che frequentano spesso gli ambienti della Scuola, in particolare il collegio Carducci e la mensa, dal lunedì al venerdì.
\end{itemize}

Non è necessario saper già giocare a Lupus in Tabula, è sufficiente farsi spiegare le regole di base da qualcuno che sa già giocare.

Tutti i partecipanti si impegnano a farsi vedere un po' in giro negli ambienti della Scuola nei giorni della partita. Nessuno vi chiede di non partire venerdì mattina per tornare a casa se non avete lezione, ma se vivete da reclusi in camera vostra e non mangiate mai in mensa potrebbe essere il caso di non giocare.


\subsection{Quando si giocherà}

...


\subsection{Cambiamenti rispetto alla terza edizione}

...


\pagebreak
\section{Metagioco}

\subsection{Giorni e notti di gioco}

I giorni di gioco si estendono dalle 8:00 (circa) alle 22:00 di lunedì, martedì, mercoledì e giovedì, e dalle 8:00 (circa) di venerdì alle 22:00 di domenica.
Le notti di gioco si estendono dalle 22:00 (circa) alle 8:00 di domenica, lunedì, martedì, mercoledì e giovedì.



\subsection{Comunicazioni GM-giocatori, votazioni, attivazioni dei poteri speciali}

All'inizio della partita, ciascun giocatore riceve le credenziali per accedere all'interfaccia web di Lupus 4, il cui indirizzo è \verb|http://uz.sns.it/lupus/|.
Le votazioni durante il giorno e le attivazioni dei poteri speciali avvengono tutte tramite l'interfaccia web. Le informazioni sulle votazioni del giorno e sugli avvenimenti della notte compaiono a loro volta sull'interfaccia web (le informazioni pubbliche possono essere visualizzate senza bisogno di autenticazione, mentre quelle private sono accessibili solo dopo il login).
In caso di impossibilità di accedere a internet, i giocatori possono eccezionalmente contattare il GM (ad esempio via SMS) per comunicargli le proprie intenzioni di voto e/o il modo in cui desiderano utilizzare il proprio potere speciale.

Le comunicazioni tra GM e giocatori sono inviolabili: è vietato origliare le conversazioni del GM con i giocatori, fare pressione sul GM in qualsiasi modo, inviargli e-mail con intenti truffaldini, rifiutarsi di rispondere sinceramente alle sue domande sui messaggi ricevuti e qualsiasi azione che ricordi anche solo vagamente le precedenti.
È inoltre vietato spiare altri giocatori mentre accedono all'area riservata dell'interfaccia web, rubare o hackare account altrui (anche approfittando di eventuali distrazioni), mostrare o dare accesso al proprio account ad altri giocatori, o cercare di violare la sicurezza dell'interfaccia web.
Le medesime regole si applicano alla lettera personale che viene data a ciascun giocatore all'inizio della partita (contenente il ruolo assegnatogli).
La violazione di queste regole può portare alla squalifica del giocatore o, nei casi più gravi, dell'intera fazione.


\subsection{Comunicazioni giocatori-giocatori}

I giocatori possono comunicare tra di loro con qualsiasi mezzo. Possono dirsi qualsiasi cosa.
L’unica cosa illecita è utilizzare truffaldinamente le e-mail per capire i ruoli delle altre persone: mostrare ad un altro giocatore le informazioni che si sono ricevute dal GM (via interfaccia web, via e-mail o con qualsiasi altro mezzo) è vietato; pressioni come ``fammi vedere la tua casella di posta da lontano, per vedere quante mail ricevi sul gioco'' sono vietate; mandare email che sembrano provenire da un indirizzo di posta diverso dal proprio per imbrogliare il destinatario è vietato; qualunque cosa che assomigli alle precedenti è vietata. La violazione di queste regole può portare alla squalifica del giocatore o, nei casi più gravi, dell'intera fazione.

Anche i morti possono parlare con gli altri membri del villaggio, e continuano a giocare. Si incoraggiano i giocatori a non comunicare solo via e-mail ma anche di persona.


\pagebreak
\section{Gioco}


\subsection{Fazioni e condizioni di vittoria}

I giocatori sono divisi in tre fazioni: la Fazione dei Popolani, la Fazione dei Lupi e la Fazione dei Negromanti.
Una fazione vince se, subito dopo l'alba oppure subito dopo la votazione del tramonto, tutti i personaggi vivi appartengono a quella fazione.
Inoltre, la Fazione dei Lupi perde immediatamente se muoiono tutti i Lupi, e la Fazione dei Negromanti perde immediatamente se muoiono tutti i Negromanti. Quando una di queste due eventualità accade, viene reso pubblico l'elenco dei membri della fazione che ha appena perso (vivi e morti), e questi vengono esiliati dal villaggio: da quel momento in poi, smettono di giocare a tutti gli effetti.

Se una fazione ha chiaramente vinto prima che le condizioni di vittoria precedenti siano rispettate, il GM può porre fine alla partita in anticipo (ma non deve farlo per forza).



\subsection{Svolgimento del giorno e votazioni}

L'inizio del giorno è annunciato sul sito web, insieme a tutte le informazioni ottenute da ciascun giocatore in seguito agli avvenimenti della notte appena trascorsa.
Ogni giorno, il Villaggio convoca un'assemblea per stabilire chi condannare a morte.
I personaggi vivi hanno il diritto di votare entro le ore 22 un abitante del villaggio da condannare a morte. Dopo le ore 22, viene pubblicata la lista dei votanti insieme alla preferenza espressa da ciascuna persona.
Il voto è ritenuto valido se almeno il 50\% dei vivi ha votato; in caso contrario, nessuno viene condannato. Il personaggio che ha ricevuto più voti di tutti muore.
In caso di parità tra due o più personaggi, fra questi viene condannato quello eventualmente votato dal Sindaco; se nessuno di questi è stato votato dal Sindaco, ne muore uno a caso.
%Se due o più persone hanno ricevuto lo stesso numero di voti, che è superiore a quello dei voti ricevuti da chiunque altro, muore quella che è stata votata dal Sindaco (vedi Sezione \ref{sindaco}); se nessuna delle due è stata votata dal Sindaco, ne muore una a caso.

Di giorno è anche possibile votare per eleggere un nuovo Sindaco, come è spiegato più dettagliatamente nella Sezione \ref{sindaco}.


\subsection{Svolgimento della notte e poteri speciali}

La notte inizia nel momento in cui viene pubblicato l'esito della votazione del giorno, e termina alle 8:00 della mattina seguente.
Durante la notte, i personaggi che hanno poteri speciali hanno la facoltà di attivarli.
Alcuni personaggi hanno un potere speciale attivabile ogni due notti: si intende che tale potere speciale può essere usato se e solo se non è stato usato durante la notte precedente: in altre parole, l'unica restrizione è data dal non poterlo usare in due notti consecutive.


\subsection{Suicidio}

Ogni personaggio ha un potere speciale attivabile di notte: il suicidio.
Se un personaggio si suicida, viene trovato morto la mattina seguente. Quando qualcuno muore di notte, il villaggio sa solo che è morto, non come ciò sia accaduto.
Questa regola è stata introdotta principalmente per permettere a chi decide di uscire dal gioco di farlo quando desidera, ma è ovviamente possibile suicidarsi come strategia di gioco, e continuare la partita.


\subsection{Composizione del villaggio}

(DA RISCRIVERE)
La composizione del villaggio è scelta dal GM e non è nota ai partecipanti. I ruoli sono assegnati casualmente ai giocatori.
Tutte e tre le fazioni sono sicuramente presenti. All'inizio della partita, la Fazione dei Lupi e la Fazione dei Negromanti contano in totale circa $1/3$ dei giocatori, ma il loro numero non è noto con precisione ai giocatori.

Il GM è libero, all'inizio della partita, di dare alcune indicazioni sulla composizione (ad esempio, potrebbe dire: ``Su $40$ giocatori presenti, i lupi sono più di $4$, la Fazione dei Lupi non sono più di $11$ e ci sono almeno un Veggente ed un Medium'').


\subsection{Ruoli}

Non è necessario che tutti i giocatori sappiano cosa fa ogni ruolo; se siete particolarmente pigri, è sufficiente che sappiate quale potere speciale avete voi (vi verrà ricordato nella lettera in cui vi si assegnerà il ruolo). Tuttavia, per giocare in modo più efficace, è certamente conveniente sapere anche cosa possono fare gli altri.


I seguenti sono i ruoli che \emph{possono} comparire nel gioco, divisi per fazione.


\subsection*{Fazione dei Popolani}

\begin{itemize}
 \item {\bf Contadino} (aura bianca). Il Contadino non ha alcun potere speciale.

 \item {\bf Cacciatore} (aura bianca). Ogni notte, il Cacciatore può scegliere un personaggio vivo e puntarlo col fucile. Se il Cacciatore viene ucciso durante la notte, la persona scelta muore, uccisa dal Cacciatore.
 Il potere speciale del Cacciatore non si attiva come conseguenza del proprio suicidio.
 
 \item {\bf Custode del cimitero} (aura bianca). Ogni notte, il Custode del cimitero può scegliere una persona morta e custodirne la tomba. Per quella notte, la persona scelta non può essere risvegliata come Spettro.
 
 Questo potere speciale non può essere usato per due notti consecutive sullo stesso personaggio.

 \item {\bf Divinatore} (aura bianca, mistico). Il Divinatore non ha alcun potere speciale. All'inizio della partita, il Divinatore è a conoscenza di quattro proposizioni nella forma ''Il personaggio X ha il ruolo Y''. Esattamente due sono vere ed esattamente due sono false.
 
 Il modo in cui queste quattro frasi sono state generate è a discrezione del GM, e non è noto ai giocatori.

 \item {\bf Esorcista} (aura bianca, mistico). Ogni notte, l'Esorcista può scegliere un personaggio, vivo o morto, e benedire la sua casa.
 Per quella notte, gli Spettri non possono utilizzare il proprio potere speciale sul personaggio scelto dall'Esorcista.
%  Quando uno Spettro tenta di usare un potere speciale su qualcuno schermato dall'Esorcista, viene a sapere che il suo potere speciale non ha funzionato.
 
 \item {\bf Espansivo} (aura bianca). Ogni due notti, l'Espansivo può scegliere un personaggio vivo, e andare a trovarlo. Questo personaggio scopre l'identità dell'Espansivo.

 \item {\bf Guardia del corpo} (aura bianca). Ogni notte, la Guardia del corpo può scegliere un personaggio vivo e proteggerlo. Durante la notte, tale personaggio non potrà morire per effetto dell'attacco dei Lupi.
 
 La Guardia del corpo non può usare il suo potere speciale su sé stessa.
 
 \item {\bf Investigatore} (aura bianca). Ogni notte, l'Investigatore può scegliere un personaggio morto e indagare su di esso. Scopre il colore della sua aura.

 \item {\bf Mago} (aura bianca, mistico). Ogni notte, il Mago può scegliere un personaggio, vivo o morto, e captarne la magia. Scoprire se quel personaggio è un mistico oppure no.
 
 \item {\bf Massone} (aura bianca). Il Massone non ha alcun potere speciale. Il Massone conosce gli altri Massoni.
 
 \item {\bf Messia} (aura bianca, mistico). Una sola volta durante l'arco della partita, il Messia può scegliere un pergonaggio morto e resuscitarlo. Quella persona ritornerà in vita all'inizio del giorno seguente, riacquistando i suoi poteri speciali.

%  \item {\bf Nipote di Mubarak} (aura nera). Ha un potere speciale passivo: se la Nipote di Mubarak viene messa al rogo dal villaggio, non muore.

 \item {\bf Necrofilo} (aura bianca). Una sola volta durante l'arco della partita, il Necrofilo può scegliere un personaggio morto e rubarne il potere. 
 Se il personaggio scelto ha un potere speciale attivabile una volta ogni una o due notti, il Necrofilo lo scopre ed ottiene tale potere speciale.
 Se il personaggio scelto non aveva un potere speciale di questo tipo, il potere speciale del Necrofilo è sprecato.
 Il Necrofilo non può acquisire il potere speciale dei Lupi (uccidere), dei Negromanti (creare Spettri) o del Fantasma.
 
 Nel momento in cui il Necrofilo ottiene il potere speciale di un altro personaggio, acquisisce anche il ruolo, l'aura e la proprietà di essere mistico del personaggio in questione.

 \item {\bf Stalker} (aura bianca). Ogni due notti, lo Stalker può scegliere un personaggio vivo e pedinarlo. Scopre se quel personaggio ha agito durante la notte, e su chi. Non scopre, però, cosa ha fatto.
 
 Lo Stalker non può usare il suo potere speciale su sé stesso.
 
 \item {\bf Veggente} (aura bianca, mistico). Ogni notte, il Veggente può scegliere un personaggio vivo e scrutarlo nella sua sfera di cristallo. Scopre il colore della sua aura.

 \item {\bf Voyeur} (aura bianca). Ogni due notti, il Voyeur può scegliere un personaggio, vivo o morto, e spiarlo. Scopre quali sono le persone che durante la notte hanno agito su quel personaggio.
 
 Il Voyeur non può usare il suo potere speciale su sé stesso.


\end{itemize}


\subsection*{Fazione dei Lupi}

\begin{itemize}
 \item {\bf Lupi} (aura nera). Ogni notte, ciascun Lupo può scegliere un personaggio e tentare di ucciderlo.
 Se tutti i Lupi che decidono di usare il proprio potere speciale scelgono lo stesso personaggio, questi muore.
 Se almeno due Lupi indicano personaggi diversi, il loro potere speciale non ha effetto. Nessuno dei personaggi scelti muore.
 
 I Lupi non possono uccidere i Negromanti.
 
 I Lupi conoscono le Fattucchiere e gli altri Lupi.

 \item {\bf Avvocato del diavolo} (aura nera). Ogni due notti, l'Avvocato del diavolo può scegliere un personaggio vivo e scrivere per lui una missiva di rilascio.
 Durante il giorno successivo, se l'assemblea avrà deciso, tramite la votazione, di uccidere il personaggio scelto dall'Avvocato del diavolo, questi non muore. (CHIARIRE COSA SUCCEDE)
 
 L'Avvocato del diavolo non può usare il suo potere speciale su sé stesso.
 %Se questa persona sarà quella che, il giorno dopo, avrà ricevuto più voti, non morirà (grazie ad una legge ad personam), e morirà invece la persona con il secondo numero più alto di voti.

 \item {\bf Complice} (aura bianca). Il Complice non ha alcun potere speciale. Il Complice conosce i Sequestratori e gli eventuali altri Complici.

 \item {\bf Diavolo} (aura nera, mistico). Ogni notte, il Diavolo può scegliere un personaggio vivo e leggerne l'anima. Scopre il ruolo del personaggio scelto, ed il ruolo che questi aveva all'inizio della partita, se diverso. %Scopre inoltre il colore della sua aura.
 
 \item {\bf Fattucchiera} (aura nera). Ogni notte la Fattucchiera può scegliere un personaggio, vivo o morto, e stregarlo. Per quella notte, il colore dell'aura del personaggio scelto risulta diverso da quello effettivo.
 
 La Fattucchiera conosce i Lupi e le eventuali altre Fattucchiere.
 
 \item {\bf Sequestratore} (aura nera). Ogni notte, il Sequestratore può scegliere un personaggio vivo e rapirlo. Per quella notte, il personaggio scelto non può agire.
 Stalker e Voyeur ricevono informazioni come se quel personaggio non avesse agito.
 %Se questa persona cerca di uscire di casa per utilizzare il proprio potere speciale su qualcuno, fallisce. Se si tratta di un Lupo, ed è l'unico che va ad uccidere, non muore nessuno. Se si tratta di un personaggio che usa un potere speciale (ad esempio l'Espansivo, o il Messia), il potere speciale di quel personaggio è sprecato.
 %Il Sequestratore non sa se la persona ha provato ad uscire di casa oppure no, e la persona non sa chi l'ha sequestrata (ma viene a sapere che il suo potere speciale non ha funzionato).
 %Il Sequestratore non può rapire la stessa persona per due notti consecutive. Se prova a farlo, si sposta come se avesse agito, ma il suo potere speciale non ha effetto.
 
 Questo potere speciale non può essere usato per due notti consecutive sullo stesso personaggio.
 
 Il Sequestratore conosce i Complici e gli eventuali altri Sequestratori.


\end{itemize}

\subsection*{Fazione dei Negromanti}

\begin{itemize}

 \item {\bf Negromanti} (aura bianca, mistici). %Ogni due notti, i Negromanti possono scegliere un personaggio e tentare di risvegliarlo come Spettro.
 Se la notte precedente nessun personaggio è stato risvegliato come Spettro, ciascun Negromante può scegliere un personaggio e selezionare un Potere.
 Se tutti i Negromanti che decidono di usare il proprio potere speciale scelgono lo stesso personaggio e selezionano lo stesso Potere, il personaggio scelto diventa uno Spettro e ottiene il Potere selezionato.
 Da quel momento in poi, egli appartiene alla Fazione dei Negromanti. Gli viene inoltre comunicata l'identità dei Negromanti che lo hanno risvegliato come Spettro.
 Se almeno due Negromanti scelgono personaggi diversi o selezionano Poteri diversi, il loro potere speciale non ha effetto. Nessun personaggio viene risvegliato come Spettro.
 
 Una volta che un Potere viene assegnato ad uno Spettro (compreso il Fantasma), questo non può più essere scelto per i nuovi Spettri. Appena viene creato uno Spettro con il Potere della Morte, tutti i Negromanti perdono il potere speciale di creare Spettri.
 
 I Lupi, le Fattucchiere e tutti i personaggi che appartengono già alla Fazione dei Negromanti non possono essere risvegliati come Spettri. % Se un Negromante cerca di risvegliare come Spettro uno di questi personaggi, fallisce.

 I Negromanti non possono essere uccisi dai Lupi.

 I Negromanti conoscono gli altri Negromanti.
 
 \item {\bf Fantasma} (aura bianca). Il Fantasma non ha alcun potere speciale. Se il Fantasma muore, anche in caso di suicidio, diventa immediatamente uno Spettro ed ottiene uno dei Poteri non ancora assegnati, scelto in modo casuale fra quelli possibili. Gli viene inoltre comunicata l'identità dei Negromanti, e ai Negromanti viene comunicata l'identià del Fantasma.
 %Il Fantasma non conta ai fini del numero di Spettri che possono essere creati dai Negromanti, ma i Negromanti non possono più creare Spettri con il Potere assegnato al Fantasma.
 Al Fantasma non può venire assegnato il Potere della Morte.
 
 \item {\bf Ipnotista} (aura bianca). Ogni due notti, l'Ipnotista può scegliere un personaggio vivo e controllarne la mente.
 Da quel momento in poi il voto di quella persona è considerato uguale a quello dell'Ipnotista, indipendentemente dall'eventuale voto espresso; questo è vero anche se l'Ipnotista non vota.
 La persona scelta non viene informata di essere sotto il controllo dell'Ipnotista.

 Se un Ipnotista è morto, le persone sotto il suo controllo votano secondo il proprio volere.
 Una persona può essere sotto il controllo di un solo Ipnotista per volta, e precisamente l'ultimo ad aver agito su di essa.

 L'Ipnotista conosce i Medium e gli eventuali altri Ipnotisti.

 \item {\bf Medium} (aura bianca, mistico). Ogni notte, il Medium può scegliere un personaggio morto. Scopre il colore della sua aura e se tale personaggio è diventato o meno uno Spettro.

 Il Medium conosce gli Ipnotisti e gli eventuali altri Medium.

 \item {\bf Spettri}. Gli Spettri non sono presenti all'inizio della partita: i personaggi morti possono essere risvegliati come Spettri in seguito all'azione di un Negromante.
 Nel momento in cui un personaggio diviene uno Spettro, mantiene la sua aura ma inizia a giocare per la Fazione dei Negromanti. Gli viene comunicata l'identità del Negromante che lo ha risvegliato, e ottiene un Potere che può iniziare a usare dalla notte successiva. Inoltre, perde per sempre gli eventuali poteri speciali che possedeva.
 
 Uno Spettro rimane morto: in particolare non può essere ucciso, e il villaggio non viene informato del suo risveglio.
 La presenza di uno Spettro che sta usando il suo Potere passa inosservata agli occhi del Voyeur.
 
 Se uno Spettro viene resuscitato dal Messia, continua a giocare per la Fazione dei Negromanti, perde per sempre il suo Potere da Spettro, non riottiene i poteri speciali eventualmente posseduti prima di essere stato risvegliato come Spettro, e mantiene la sua aura.
 
%  Gli Spettri perdono la possibilità di utilizzare il proprio Potere se muoiono tutti i Negromanti.
%  Nel caso in cui un Negromante venga resuscitato dal Messia, gli Spettri riacquistano nuovamente la facoltà di utilizzare il proprio Potere.
\end{itemize}


\paragraph{Poteri degli Spettri}

\begin{itemize}
 \item {\bf Amnesia}. Ogni notte, lo Spettro può scegliere un personaggio vivo e ottenebrarne i ricordi. Il giorno successivo, il suo eventuale voto per l'assemblea viene ignorato: a tutti gli effetti è come se non avesse votato. Ciò accade anche se il personaggio scelto è sotto il controllo dell'Ipnotista: invece di votare come l'Ipnotista, non vota affatto.
 
 Questo Potere non può essere usato per due notti consecutive sullo stesso personaggio.

 \item {\bf Duplicazione}. Ogni notte, lo Spettro può scegliere un personaggio vivo e assumerne l'aspetto. Il giorno successivo, il suo voto per il rogo conta doppio. Non risulta, però, che il personaggio abbia votato due volte; semplicemente, la persona per cui vota riceve un voto in più.

 \item {\bf Illusione}. Ogni notte, lo Spettro può scegliere un personaggio, vivo o morto, quindi generare un'illusione di un personaggio vivo. L'illusione compare nella casa del personaggio scelto.
 Agli occhi del Voyeur, è come se il personaggio di cui è stata generata l'illusione avesse agito sul personaggio scelto.
 Questo Potere non ha effetto sullo Stalker.
 
 L'Esorcista impedisce l'utilizzo di questo Potere agendo sulla casa dove compare l'illusione, ma non agendo sulla casa del personaggio di cui è stata generata l'illusione stessa.

 \item {\bf Morte}. Ogni due notti, lo Spettro può scegliere un personaggio vivo e ucciderlo. Questo Potere non può essere usato sui Lupi.
 
 \item {\bf Nebbia}. Ogni notte, lo Spettro può scegliere un personaggio vivo o morto, e creare attorno alla sua casa una fittissima nebbia magica. Per quella notte, nessun altro, eccetto eventualmente l'Esorcista, può usare il proprio potere speciale su quel personaggio.
 
 \item {\bf Ombra}. Ogni notte, lo Spettro può scegliere un personaggio vivo e crearne un'ombra, quindi dirigerla verso un altro personaggio, vivo o morto.
 Agli occhi dello Stalker, è come se il personaggio scelto abbia agito sul personaggio verso cui si è diretta l'ombra (a prescindere da dove eventualmente si rechi il personaggio scelto).
 Questo Potere non ha effetto sul Voyeur.
 
 L'Esorcista impedisce l'utilizzo di questo Potere agendo sulla casa del personaggio di cui viene creata l'ombra, ma non agendo sulla casa verso cui l'ombra si dirige.
 
 \item {\bf Visione}. Ogni notte, lo Spettro può scegliere un personaggio vivo e vederne l'aura. Scopre il colore della sua aura. Scopre inoltre se quel personaggio appartiene alla Fazione dei Negromanti.
 
%  \item {\bf Confusione}.  Ogni notte, lo Spettro può scegliere un personaggio vivo. Il giorno successivo, il suo voto per il rogo viene sostituito da un voto casuale (per una persona viva diversa da sé stessa). Ciò accade anche se il personaggio scelto è sotto il controllo dell'Ipnotista: invece di votare come l'Ipnotista, vota a caso.
%  
%   Questo Potere non può essere usato per due notti consecutive sullo stesso personaggio.

\end{itemize}


\subsection{Utilizzo dei poteri speciali e fallimento}
\label{fallimento}

\paragraph{Utilizzo dei poteri speciali} I personaggi con poteri speciali si considerano sempre agire sul personaggio scelto. %Se una persona agisce durante la notte, può essere vista dallo Stalker e dal Voyeur, anche se non riesce ad utilizzare il proprio potere speciale.
Se un personaggio con un potere speciale attivabile ogni due notti, oppure una volta a partita, prova ad agire, a prescindere dal fatto che ci riesca, non potrà farlo nella notte successiva, o nel resto della partita.
I personaggi che ottengono informazioni di qualsiasi tipo su altri personaggi, le ottengono riguardo alla condizione in cui questi si trovavano al termine del giorno precedente, anche se tali informazioni cambiano nel corso della notte.
Ad esempio, se un morto viene contemporaneamente risvegliato come Spettro e visitato dal Medium, quest'ultimo scopre che il morto non è uno Spettro (perché al termine del giorno precedente non lo era ancora diventato).

\paragraph{Fallimento} Ci sono alcuni modi di utilizzare il proprio potere speciale che sono a priori proibiti, e pertanto non risultano nemmeno tra le scelte disponibili sull'interfaccia web. Ad esempio, l'Investigatore non può scegliere di agire su un personaggio vivo; oppure, un Negromante non può provare a usare il proprio potere speciale se durante la notte precedente è stato creato uno Spettro (anche se non è stato lui a crearlo).

Anche escludendo i casi menzionati sopra, non sempre un personaggio riesce ad usare il proprio potere speciale con successo.
Ci sono essenzialmente tre possibili ragioni che portano al fallimento nell'utilizzo di un potere speciale.
\begin{itemize}
 \item Blocco da parte di un altro personaggio. Ad esempio, si può essere rapiti da un Sequestratore, bloccati da un Esorcista o dallo Spettro con il Potere della Nebbia.
 \item Restrizioni sull'utilizzo del potere speciale. Ad esempio, un Lupo non riesce a uccidere se nella stessa notte un altro Lupo cerca di assassinare una persona diversa; il Necrofilo non riesce ad utilizzare il proprio potere speciale su un Lupo.
 \item Contraddizioni nell'utilizzo dei poteri speciali. Può succedere che non ci sia alcun modo logico di risolvere le azioni dei personaggi nel corso di una notte (vedi Sezione \ref{faq} per una discussione più approfondita). Questo accade per esempio se tre Sequestratori si sequestrano in modo ciclico.
\end{itemize}

Nel caso in cui un personaggio cerchi di utilizzare il proprio potere speciale e fallisca, riceve una notifica di fallimento. Tale notifica non precisa la ragione del fallimento.

%Se un personaggio viene fermato dal Sequestratore, è come se non avesse agito e in particolare non esce di casa (lo Stalker lo vede fermo, e il Voyeur non lo vede a casa di altri).
Il Sequestratore, impedisce agli altri personaggi di agire, non di usare il loro potere speciale: una persona rapita dal Sequestratore non agisce affatto (e pertanto Stalker e Voyeur non possono vederla).
In tutti gli altri casi, il personaggio che fallisce agisce comunque, e pertanto può essere visto da Stalker e Voyeur, ma il suo potere speciale non ha effetto. 

%Se il potere speciale è unico o attivabile ogni due notti, l'utilizzo conta anche in caso di fallimento. Per esempio, un Necrofilo che usa il suo potere speciale su un Lupo, fallisce e perde il potere speciale per sempre.

\subsection{Lo status di Sindaco}
\label{sindaco}

% Il voto del Sindaco vale 1,5 voti per stabilire chi muore dopo una votazione di giorno (ma non per stabilire se una votazione è valida).
Se durante l'assemblea si scopre che non vi è un personaggio che ha ricevuto strettamente più voti di ogni altro, tra i personaggi che hanno ricevuto il maggior numero di voti, viene condannato quello eventualmente votato dal Sindaco.
Il Sindaco è scelto casualmente all'inizio della partita dal GM (in modo scorrelato dalla Fazione di appartenenza).

In qualsiasi momento il Sindaco può designare un successore: se il Sindaco muore, il successore designato diventa il nuovo Sindaco. Se il Sindaco muore senza aver mai designato un successore, o se il successore designato è morto, il GM determina casualmente chi riceverà la carica.

Ogni giorno, oltre a votare per mettere qualcuno al rogo, ogni abitante del villaggio può votare per l'elezione di un nuovo Sindaco. Se alla fine della giornata un personaggio ha ricevuto il voto di almeno il 50\% dei personaggi vivi, riceve la carica di Sindaco (e il Sindaco precedente la perde).



\pagebreak
\section{L'interfaccia web}
(QUANDO CI SARÀ, METTEREMO QUI UNA BREVE GUIDA)



\section{Le cose che vorreste sapere ma non avete mai osato chiedere}
\label{faq}

Le regole spiegate fino a questo punto sono più che sufficienti per poter giocare. Tuttavia possono accadere eventi piuttosto strani (ad esempio, cosa succede se due Sequestratori si sequestrano a vicenda?). In tutti questi casi, il GM decide in modo inappellabile cosa accade; ai giocatori è sempre permesso chiedergli cosa accade in una di queste combinazioni.

Segue una lista (verosimilmente non esaustiva) di eventualità bizzarre con le corrispettive spiegazioni.

\begin{itemize}
%  \item Un Negromante che utilizza la propria azione notturna per ricercare un nuovo Potere e contemporaneamente viene sequestrato, fallisce. Potrà nuovamente provare a ricercare quel Potere nelle notti successive.
 
 \item In caso di contraddizione nell'utilizzo dei poteri speciali, viene scelto casualmente un personaggio (o più di uno, se necessario), che non utilizza il proprio potere speciale e riceve la consueta notifica di fallimento (questo è il fallimento ``di terza categoria'', vedi Sezione \ref{fallimento}).
 
 Ad esempio, se un Esorcista, un Sequestratore e uno Spettro con il Potere della Nebbia agiscono simultaneamente sull'Esorcista, si verifica facilmente (esercizio) che non c'è alcun modo coerente di risolvere le azioni.
 
 \item Se nella stessa notte un Messia e un Negromante cercano di agire sulla stessa persona (morta), quella persona viene resuscitata dal Messia e non risvegliata come Spettro. In particolare, il Negromante riceve una notifica di fallimento.

 \item Se un Necrofilo usa il proprio potere speciale su uno Spettro, copia il potere speciale che quel personaggio aveva prima di essere risvegliato come Spettro.
 
 \item I Negromanti non possono nemmeno provare a usare il proprio potere speciale se durante la notte precedente è stato creato uno Spettro.
 
 \item Se più Negromanti cercano di creare degli Spettri, falliscono come succede nel caso dei Lupi, a meno che scelgano tutti la stessa persona e selezionino tutti lo stesso Potere.

 \item Se un Ipnotista morto viene resuscitato dal Messia, tutti coloro che si trovavano sotto il controllo dell'Ipnotista al momento della sua morte sono nuovamente sotto il suo controllo, a meno che nel frattempo la loro mente sia stata controllata da un altro Ipnotista.
 
 Se una persona sotto il controllo di un Ipnotista muore e viene successivamente resuscitata, torna ad essere sotto il controllo dell'ultimo Ipnotista ad aver agito su di essa.
 
 \item Se più Ipnotisti agiscono sulla stessa persona durante la stessa notte, la mente di quest'ultima viene controllata da tutti gli Ipnotisti in un ordine casuale, e perciò risulterà infine essere sotto il controllo soltanto dell'ultimo ad aver agito. Gli altri Ipnotisti non ricevono alcuna notifica di fallimento.
 
 \item Se uno Spettro con il Potere dell'Illusione prende l'aspetto di un personaggio recandosi in una casa dove agiscono anche quel personaggio e il Voyeur, il Voyeur vede comunque solo una volta quel personaggio. Se tuttavia il personaggio scelto coincide con il Voyeur, questi compare nella lista dei personaggi visti.

 \item Se l'Esorcista e lo Spettro con il Potere della Nebbia agiscono sulla stessa persona, lo Spettro viene bloccato dall'Esorcista e non viceversa.
 
 \item Se lo Stalker o il Voyeur si recano a casa di un personaggio che sta usando il proprio potere speciale su sé stesso, ricevono informazioni come se quel personaggio non avesse agito.
 
 \item La Guardia del corpo protegge la persona scelta solamente dall'attacco dei Lupi. Il Cacciatore, lo Spettro con il Potere della Morte ed eventuali altri effetti che uccidono la persona scelta non sono influenzati dal potere speciale della Guardia del corpo.
 
 \item Se un personaggio muore, viene risvegliato come Spettro, resuscitato dal Messia e la sua anima viene letta dal Diavolo, il Diavolo scopre che il personaggio è uno Spettro, ma non scopre che Potere ha. 
\end{itemize}


(E QUA CONTO SULL'ABILITÀ DI LEONARDO NEL TROVARE UN PO' DI CASI PATOLOGICI)



\end{document}
