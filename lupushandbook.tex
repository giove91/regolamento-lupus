\documentclass[a4paper,10pt]{article}


\usepackage[utf8]{inputenc}
\usepackage[italian]{babel}
\usepackage{amsmath}
\usepackage{amsthm}
\usepackage{fancyhdr}
\usepackage{amsfonts}
\usepackage{amssymb}
\usepackage{makeidx}
\usepackage[parfill]{parskip}
\usepackage{hyperref}


% * define a `\twoidxcolumn` based on `\twocolumn`:
\def\twoidxcolumn{%
%\clearpage
\global\columnwidth\textwidth
\global\advance\columnwidth-\columnsep
\global\divide\columnwidth\tw@
\global\hsize\columnwidth
\global\linewidth\columnwidth
\global\@twocolumntrue
\global\@firstcolumntrue
\col@number \tw@
%\@ifnextchar [\@topnewpage
\@floatplacement
}


\makeatletter
\def\@wrindex#1{%
   \protected@write\@indexfile{}%
      {\string\indexentry{#1}{\theenumi}}
      \endgroup
      \@esphack}

\makeatletter
\renewenvironment{theindex}
               {\twocolumn[\section*{Indice delle parole chiave presenti nella
Sezione \ref{faq}}]%
                \@mkboth{\MakeUppercase\indexname}%
                        {\MakeUppercase\indexname}%
                \thispagestyle{plain}\parindent\z@
                \parskip\z@ \@plus .3\p@\relax
                \columnseprule \z@
                \columnsep 35\p@
                \let\item\@idxitem}
               {}
\makeatother

\makeindex
% [title=Indice dei ruoli presenti nella Sezione \ref{faq},columns=3]


\usepackage[top=1.15in, bottom=1.15in, left=1.3in, right=1.3in]{geometry}

% \topmargin -1cm
% \oddsidemargin -0.5cm
% %\evensidemargin	-1cm
% \textwidth 17cm


\newcommand{\smallspace}{\vskip0.3cm}

% Title Page
\title{Introduzione a Lupus in tempo reale}
\author{Alessandro Iraci}

\begin{document}
\maketitle

\section{Introduzione}

Quella che segue è una descrizione molto sommaria delle regole di \emph{Lupus in tempo reale}, che serve a chi non ha mai partecipato per capire le meccaniche di questo gioco. Per partecipare, comunque, è molto importante leggere il regolamento completo.

All'inizio del gioco, ad ogni giocatore viene segretamente assegnato un ruolo, che determina anche la fazione di cui farà parte. Ci sono tre fazioni: i Popolani, i Lupi e i Negromanti. La partita termina quando rimane una sola fazione in gioco, che viene dichiarata vincitrice. Un giocatore vince se la sua fazione vince, a prescindere dal fatto che questi sia vivo o morto al termine della partita.
I turni di gioco sono divisi in giorni e notti. Durante il giorno, ciascuno ha il diritto di votare un giocatore da mandare al rogo. Al termine del giorno, se un giocatore ha ottenuto più della metà dei voti totali, questi muore. Durante la notte i giocatori hanno diritto ad usare le loro abilità e poteri, che dipendono dal ruolo. Un elenco dei ruoli e delle rispettive abilità e poteri è disponibile nel regolamento.

Quella che segue è una rapida descrizione delle fazioni.

\begin{itemize}
	
	\item I Popolani sono la fazione più numerosa, ma anche la più disorganizzata. All'inizio della partita non si conoscono fra loro, e molte delle loro abilità servono a ottenere informazioni sugli altri giocatori. Altri personaggi, invece, contribuiscono alla vittoria ostacolando le fazioni avversarie.
	
 	\item I Lupi si accordano fra loro per uccidere un personaggio a loro scelta. I loro obiettivi principali sono i Popolani con poteri che potrebbero farli scoprire, ma qualunque avversario in meno li avvicina alla vittoria. Gli altri giocatori della loro fazione li aiutano ad ottenere informazioni, costruirsi una copertura credibile e rallentare le indagini dei Popolani.
 	
	\item I Negromanti sono molto pochi all'inizio della partita ma, man mano che la partita prosegue, alcuni dei Popolani morti diventeranno Spettri, a cui i Negromanti potranno assegnare un potere. Gli Spettri aiutano i Negromanti a mescolarsi nel villaggio e a manipolare le votazioni del giorno.
\end{itemize}


\section{Consigli generali}

In questo gioco è fondamentale la capacità di bluffare, di crearsi una buona copertura e di coordinare le azioni della propria fazione. L'interazione con gli altri giocatori è importantissima, ed è ciò che rende divertente \emph{Lupus in tempo reale}.

In generale conviene tenere segreto il proprio ruolo. Se si gioca per i Lupi o per i Negromanti, ovviamente, venire scoperti comporta quasi certamente la morte sul rogo; se si parteggia per i Popolani, invece, esporsi significa diventare un bersaglio appetibile per gli avversari. Nonostante ciò, a volte ai Popolani conviene rivelare il proprio ruolo a persone fidate, per comunicare preziose informazioni in proprio possesso o per aiutare la fazione a coordinarsi. Bisogna ricordarsi, però, che mentire è fondamentale in questo gioco, e fidarsi degli altri è sempre una scelta da fare con molta cautela.

\subsection{Account anonimi}

Uno dei mezzi più importanti per comunicare con gli altri giocatori sono gli account anonimi sul forum. Spesso è importante parlarsi per ruolo e non per nome, e quindi occorre creare degli indirizzi associati al ruolo. Può essere utile anche creare account fasulli associati a ruoli diversi dal proprio, per aumentare la confusione fra gli avversari e sperare di ottenere da loro informazioni di qualche tipo. Questa strategia di solito è comune nelle fazioni di Lupi e Negromanti, ma anche i Popolani possono adottarla con successo.

Fidarsi di un account anonimo è una scelta da non fare alla leggera. Molti di essi infatti sono fasulli, creati con il solo scopo di estorcere informazioni agli altri giocatori, soprattutto ai più inesperti. Caratteristiche comuni fra gli account fasulli sono quelle di presentarsi dando informazioni vaghe sul ruolo del destinatario (per esempio il colore dell'aura, o l'appartenenza ad una certa lista di ruoli), sperando che siano sufficienti per ottenere almeno il minimo di fiducia necessario per chiedere in cambio informazioni altrettanto vaghe (per esempio la misticità, o la frequenza di uso dell'abilità).

A volte però un account anonimo può celare un giocatore sincero e bene intenzionato. Questi in genere si distinguono per l'esattezza e la precisione delle informazioni fornite (per esempio la persona su cui si ha agito) ed è bene non ignorarli. D'altro canto, non è da escludere che anche in questi casi si tratti di informazioni false la cui speranza di successo si basa sul gran numero di tentativi. In generale, per essere ragionevolmente certi della veridicità di un account serve che questo fornisca informazioni esatte più volte, rendendo improbabile che stia andando a tentativi.

Riuscire a capire di chi fidarsi è cruciale in \emph{Lupus in tempo reale}: fornire troppe informazioni alle persone sbagliate può essere fatale, ma tenerle tutte per sé rende impossibile coordinarsi e diminuisce le probabilità di vittoria.

%L'uso delle email non è più necessario da quando è stato creato il forum, ma nel caso uno volesse creare degli indirizzi email da usare per la partita, una cosa importante da sapere è che gli account Gmail hanno delle falle di sicurezza. La procedura di recupero della password richiede informazioni come la data di creazione dell'account, gli indirizzi a cui si è scritto, e altre informazioni più o meno pubbliche per un account di recente creazione. Inoltre, Gmail effettua un controllo sull'indirizzo IP, che è comune a molti ambienti della scuola. Questo insieme di cose rende gli account Gmail molto facili da violare, e può succedere che un account violato incida in modo pesante sulla partita.
% In particolare, nella quinta edizione l'account anonimo del Diavolo è stato violato, e questa cosa ha dato al villaggio i nomi del Diavolo stesso, di un Avvocato del Diavolo e di uno Scrutatore. La presenza di un Diavolo morto e noto a tutto il villaggio è stata cruciale, perché i due Trasformisti presenti hanno agito su di lui con successo, ottenendo così informazioni fondamentali per i Popolani. A quel punto, la vittoria dei Popolani era quasi certa, ed infatti è avvenuta con un margine molto ampio.

\section{Strategie per i Popolani}

I Popolani devono cercare di scoprire quante più informazioni possibile, organizzare controlli sugli altri giocatori e uccidere dei nemici con le votazioni ogni qualvolta ne hanno l'occasione. Spesso conviene mandare al rogo qualcuno anche sulla base di semplici sospetti, perché non farlo, di fatto, permette ai Lupi di uccidere un Popolano in più.

Fra i Popolani, sono cruciali i ruoli che permettono di ottenere informazioni, come Divinatore, Mago, Stalker, Veggente, Voyeur. Tali ruoli dovrebbero cercare di non esporsi e di trovare il tempo e il modo giusti di comunicare al villaggio le loro informazioni. Sono le vittime preferite dei Lupi, se noti. È molto importante inoltre riuscire a coordinarsi efficacemente per effettuare controlli incrociati, ed in questo senso sono utili l'Esorcista, l'Espansivo, e l'Investigatore. Anche Stalker e Voyeur possono facilmente ottenere la fiducia di altri giocatori, indovinando i loro spostamenti.

Usare al meglio le abilità di Cacciatore, Messia, e Trasformista può far pendere l'ago della bilancia in una direzione piuttosto che in un'altra, ribaltando le maggioranze per il rogo di giorno oppure recuperando abilità utili di personaggi uccisi. È importante per tali personaggi decidere se utilizzare la propria abilità presto, rischiando di commettere degli errori ma essendo certi di riuscirci, oppure attendere di avere sufficienti informazioni per farlo più efficacemente, ma rischiando di venire uccisi nel frattempo.

Infine, tra i ruoli rimasti la Guardia del corpo deve intuire chi sono gli alleati più preziosi e proteggerli, i Massoni sono utili per creare componenti connesse numerose e ridurre l'elenco dei potenziali avversari per esclusione, lo Sciamano può fornire un cruciale vantaggio spezzando Incantesimi necessari ai Negromanti, e la Spia può certificarsi facilmente, oltre a capire se qualcuno sta mentendo o se una votazione è stata manipolata.

\subsection{Strategie ruolo per ruolo}

\begin{itemize}
	
	\item {\bf Cacciatore}. Avendo aura nera e non essendo mistico, il Cacciatore è difficilmente distinguibile da un Lupo, e pertanto non è da escludere che venga scambiato per uno di essi. L'abilità del Cacciatore può essere utilizzata solo una volta durante la partita, e per minimizzare il rischio che una morte precoce glielo impedisca, ha senso farlo la prima volta in cui si ha la ragionevole certezza che un componente delle altre due Fazioni sia ancora vivo al tramonto, per esempio per il mancato raggiungimento del quorum o perché ve n'era più di uno da condannare.
	
	È bene che un Cacciatore annunci, tramite un account anonimo o di persona a qualcuno di fidato, la sua intenzione di utilizzare l'abilità, così da poter confermare la propria identità. Farlo pubblicamente, però, rischia di causare l'intervento di Sequestratori, Stregoni, o Spettri dell'Occultamento, e pertanto, se si sceglie di seguire questa via, è opportuno farlo durante l'alba o appena prima.
	
	\item {\bf Contadino}. Il Contadino, non avendo alcuna abilità, non ha molti modi di ottenere informazioni. D'altro canto, non essendo un bersaglio interessante per Lupi o Negromanti, potrebbe decidere di dichiarare il suo ruolo, apertamente o ad un numero ristretto di giocatori, e potenzialmente fungere da collegamento fra ruoli con abilità più interessanti. Altrimenti, tenere segreto il proprio ruolo ma rendersi molto partecipe alle assemblee può essere un buon modo per attirare l'interesse dei Lupi, deviando l'attenzione da altri personaggi. In fondo alla fine conta la vittoria della Fazione, non la propria vita.
	
	\item {\bf Divinatore}. Il Divinatore è particolarmente forte in congiunzione con altri ruoli che scoprono informazioni, perché è facilmente in grado di scoprire se qualcuno ha mentito. Inizialmente ha senso usare le frasi per connettersi ad un buono certo con cui scambiare informazioni: indovinare con esattezza il ruolo di una persona, infatti, è spesso una prova sufficiente che si è stati contattati proprio da un Divinatore, soprattutto se questo accade all'inizio della partita quando le informazioni disponibili sono poche.
	
	Più avanti nel gioco, quando cominciano a sorgere dei sospetti, è utile verificare se sono fondati. In mancanza di alternative migliori, un Divinatore può andare alla cieca cercando dei Lupi o dei Negromanti, essendo questi solitamente i ruoli più rappresentati delle fazioni avversarie. Avendo un'alta capacità di confutare le dichiarazioni fasulle, il Divinatore è una delle vittime preferite dei Lupi, e pertanto è bene che tenga segreta la propria identità.
	
	\item {\bf Esorcista}. Il compito dell'Esorcista è quello di limitare i poteri degli Spettri, e pertanto all'inizio della partita non ha molto da fare. Potrebbe essere comunque utile uscire quanto più spesso possibile, anche senza Spettri in giro, per sperare di essere intercettato da uno Stalker o da un Voyeur con cui connettersi. L'Esorcista è infatti maggiormente efficiente in collaborazione con i ruoli che ottengono informazioni, per certificare che tali informazioni siano vere e non compromesse dall'azione di alcuni Spettri.
	
	L'Esorcista è una delle vittime preferite dai Negromanti, ma non è un problema se i Lupi scoprono la sua identità: questi ultimi, infatti, hanno interesse che ci sia un Esorcista nel Villaggio, che usando la sua abilità su un Negromante, può permettere che questi venga ucciso di notte (da un Cacciatore o da un Lupo) nonostante lo Spettro dell'Occultamento, o di giorno nonostante lo Spettro dell'Assoluzione.
	
	\item {\bf Espansivo}. Quello dell'Espansivo è un ruolo estremamente delicato. Il compito principale di un Espansivo è quello di costringere la gente a dare informazioni: una volta certificatosi come Popolano ad un altro abitante del Villaggio, l'Espansivo potrà chiedergli tutte le informazioni che possiede. Se l'Espansivo incontra un Popolano, potrà guidarlo sulle azioni da compiere in base alle informazioni raccolte; se invece incontra un componente di una Fazione avversaria, questi dovrà velocemente inventare un bluff, che l'Espansivo potrà eventualmente confutare tramite i suoi altri contatti.
	
	Il problema dell'Espansivo è che dovrà presto o tardi uscire allo scoperto: è molto probabile che, nell'arco di due o tre uscite, il suo nome sia finito ad un componente delle altre Fazioni, che hanno entrambe interesse ad eliminarlo. Da quel momento in poi, bisogna capire se dichiararsi pubblicamente, per dare a tutti un'informazione già nota a Lupi e Negromanti, oppure cercare di restare nascosti, per non far capire di essere in possesso di informazioni preziose e magari guadagnare qualche giorno di vita.
	
	\item {\bf Guardia del corpo}. La Guardia del corpo ha un compito non facile: deve intuire chi sono i giocatori più a rischio di essere sbranati da un Lupo e proteggerli. Molto utile per aumentare l'aspettativa di vita di ruoli come Divinatore, Espansivo, Stalker, Veggente e Voyeur, la Guardia del corpo ha a sua volta una piccola capacità di ottenere informazioni, che saltuariamente può risultare utile per scoprire se qualcuno sta dicendo la verità oppure no.
	
	Il più grosso pericolo che corre la Guardia del corpo nel proteggere sempre la vittima più appetibile per i Lupi è quello di imbattersi in un Assassino, e pertanto la Guardia deve fare attenzione a come sceglie chi proteggere: agendo l'Assassino ogni due notti, la parità delle stesse può fornire una preziosa linea guida.
		
	\item {\bf Investigatore} (aura bianca). Ogni due notti, l'Investigatore può scegliere un personaggio morto e indagare su di esso. Scopre il ruolo di tale personaggio.
	
	\item {\bf Mago} (aura bianca, mistico). Ogni notte, il Mago può scegliere un personaggio, vivo o morto, e percepirne la magia. Scopre se tale personaggio è un mistico oppure no.
	
	\item {\bf Massone} (aura bianca). Il Massone non ha alcun potere speciale. I Massoni conoscono gli altri Massoni.
	
	\item {\bf Messia} (aura bianca, mistico). Una sola volta durante l'arco della partita, il Messia può scegliere un personaggio morto e resuscitarlo. Tale personaggio ritornerà in vita all'inizio del giorno seguente. Se il personaggio scelto è uno Spettro, il potere del Messia fallisce.
	
	\item{\bf Sciamano} (aura nera, mistico). Ogni due notti, lo Sciamano può scegliere un personaggio morto ed effettuare su di lui un rito arcano.
	
	Se il personaggio scelto è uno Spettro, e vi è un Incantesimo attivo su di esso, l'Incantesimo viene disattivato. Inoltre, se durante la stessa notte un Negromante tenta di attivare un Incantesimo su tale Spettro, l'abilità del Negromante fallisce.
	
	Se invece il personaggio scelto appartiene alla Fazione dei Popolani, e durante la stessa notte un Negromante tenta di utilizzare il suo potere su di lui, il potere del Negromante fallisce.
	
	\item {\bf Spia} (aura bianca). Ogni notte, eccetto la prima, la Spia può scegliere un personaggio vivo e curiosare fra i suoi effetti personali. Scopre a chi era diretto il voto di tale personaggio durante l'ultima votazione.
	
	\item {\bf Stalker} (aura bianca). Ogni due notti, lo Stalker può scegliere un personaggio vivo e pedinarlo. Scopre un eventuale movimento generato da tale personaggio.
	
	\item {\bf Trasformista} (aura nera). Una sola volta durante l'arco della partita, il Trasformista può scegliere un personaggio morto e prenderne il ruolo. Se il personaggio scelto ha un ruolo appartenente alla Fazione dei Popolani, il Trasformista lo scopre ed ottiene tale ruolo (includendo aura e misticità); altrimenti, l'abilità del Trasformista fallisce.
	
	\item {\bf Veggente} (aura bianca, mistico). Ogni notte, il Veggente può scegliere un personaggio vivo e scrutarlo nella sua sfera di cristallo. Scopre il colore dell'aura di tale personaggio.
	
	\item {\bf Voyeur} (aura bianca). Ogni due notti, il Voyeur può scegliere un personaggio vivo e spiarlo. Scopre tutti i movimenti diretti verso la casa di tale personaggio.
	
\end{itemize}

\section{Strategie per i Lupi}

La fazione dei Lupi è meno numerosa e più organizzata di quella dei Popolani. Per raggiungere la vittoria essi devono cercare di ottenere quante più informazioni possibili sui Popolani, per capire quali possono manipolare e quali vadano invece eliminati. È inoltre importante ostacolare per quanto possibile lo scambio di informazioni tra i Popolani, ad esempio creando account falsi sul forum per generare rumore e rendere meno credibili gli account veri, oppure infiltrandosi (a proprio rischio) in una cerchia di Popolani noti diffondendo informazioni false o fuorvianti. Per i Lupi, avere una copertura credibile da Popolano, coadiuvata da un account anonimo di cui il villaggio si fida, può fare davvero la differenza fra vittoria e sconfitta.

Potendo coordinarsi in anticipo, i Lupi possono cercare di manipolare l'assemblea dei Popolani a proprio vantaggio (tante persone che dicono la stessa cosa sono di solito convincenti), ma questo comporta dei seri rischi nel caso alcune posizioni siano difficili da sostenere oppure evidentemente contraddittorie: proporre strategie sbagliate o difendere strenuamente membri della propria fazione su di cui ricadono accuse troppo gravi è fonte di forte sospetto. Può essere invece utile a volte contraddirsi a vicenda, rischiando magari di vedere uno dei propri alleati sul rogo ma allontanando le accuse da sé stessi.

\section{Strategie per i Negromanti}

La fazione dei Negromanti è quella che all'inizio è tendenzialmente meno numerosa delle altre. Si amplierà man mano che verranno creati Spettri. Per i Negromanti è relativamente facile nascondersi e mimetizzarsi nel resto del villaggio, e per fare ciò è fondamentale utilizzare bene gli Spettri.

L'obiettivo finale dei Negromanti è quello di far perdere il quorum al villaggio, dopodiché possono abbastanza facilmente conquistare la vittoria (esercizio). Per fare questo lo Spettro dell'Amnesia è molto utile, ma bisognerà stare attenti a bilanciare le strategie difensive percorribili con Confusione, Illusione e Occultamento, che permettono di allontanare i sospetti dai Negromanti, con le strategie offensive date da Amnesia e Morte, che avvicinano la fine della partita. L'Incantesimo della Telepatia è utile per ottenere informazioni a volte necessarie, mentre quello della Vita può essere adoperato in vari modi, per esempio per tentare un bluff da Messia, per usare un'abilità dei Popolani che può risultare molto utile, o per ottenere una maggioranza inaspettata al momento di votare. Infine, gli Incantesimi dell'Assoluzione e della Diffamazione sono molto importanti verso la fine della partita, per proteggere i Negromanti dal rogo e per liberarsi di qualche nemico.

\section{Ringraziamenti}

Si ringraziano Manuele Cusumano, Luca Ghidelli, Giovanni Italiano, Giovanni Mascellani, Matteo Migliorini, Giovanni Paolini, Leonardo Tolomeo e molte altre persone coinvolte nella stesura del regolamento per il loro contributo.

\end{document}
