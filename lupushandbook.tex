\documentclass[a4paper,10pt]{article}


\usepackage[utf8]{inputenc}
\usepackage[italian]{babel}
\usepackage{amsmath}
\usepackage{amsthm}
\usepackage{fancyhdr}
\usepackage{amsfonts}
\usepackage{amssymb}
\usepackage{makeidx}
\usepackage[parfill]{parskip}
\usepackage{hyperref}


% * define a `\twoidxcolumn` based on `\twocolumn`:
\def\twoidxcolumn{%
%\clearpage
\global\columnwidth\textwidth
\global\advance\columnwidth-\columnsep
\global\divide\columnwidth\tw@
\global\hsize\columnwidth
\global\linewidth\columnwidth
\global\@twocolumntrue
\global\@firstcolumntrue
\col@number \tw@
%\@ifnextchar [\@topnewpage
\@floatplacement
}


\makeatletter
\def\@wrindex#1{%
   \protected@write\@indexfile{}%
      {\string\indexentry{#1}{\theenumi}}
      \endgroup
      \@esphack}

\makeatletter
\renewenvironment{theindex}
               {\twocolumn[\section*{Indice delle parole chiave presenti nella
Sezione \ref{faq}}]%
                \@mkboth{\MakeUppercase\indexname}%
                        {\MakeUppercase\indexname}%
                \thispagestyle{plain}\parindent\z@
                \parskip\z@ \@plus .3\p@\relax
                \columnseprule \z@
                \columnsep 35\p@
                \let\item\@idxitem}
               {}
\makeatother

\makeindex
% [title=Indice dei ruoli presenti nella Sezione \ref{faq},columns=3]


\usepackage[top=1.15in, bottom=1.15in, left=1.3in, right=1.3in]{geometry}

% \topmargin -1cm
% \oddsidemargin -0.5cm
% %\evensidemargin	-1cm
% \textwidth 17cm


\newcommand{\smallspace}{\vskip0.3cm}

% Title Page
\title{Lupus in tempo reale\\ Introduzione al gioco}
\author{Alessandro Iraci, Giovanni Mascellani, Giovanni Paolini, Leonardo Tolomeo}

\begin{document}
\maketitle

\section{Introduzione}

Quella che segue è una descrizione molto sommaria delle regole di \emph{Lupus in tempo reale}, che serve a chi non ha mai partecipato per capire le meccaniche di questo gioco. Per partecipare, comunque, è molto importante leggere il regolamento completo.

All'inizio del gioco, ad ogni giocatore viene segretamente assegnato un ruolo, che determina anche la fazione di cui farà parte. Ci sono tre fazioni: i Popolani, i Lupi e i Negromanti. La partita termina quando rimane una sola fazione in gioco, che viene dichiarata vincitrice. Un giocatore vince se la sua fazione vince, a prescindere dal fatto che questi sia vivo o morto al termine della partita.
I turni di gioco sono divisi in giorni e notti. Durante il giorno, ciascuno ha il diritto di votare un giocatore da mandare al rogo. Al termine del giorno, salvo eccezioni, muore il giocatore con più voti. Durante la notte i giocatori hanno diritto ad usare le loro abilità e poteri, che dipendono dal ruolo. Un elenco dei ruoli e delle rispettive abilità e poteri è disponibile nel regolamento.

Quella che segue è una rapida descrizione delle fazioni.

\begin{itemize}
 \item I Lupi si accordano fra loro per uccidere un personaggio a loro scelta. I loro obiettivi principali sono i Popolani con poteri che potrebbero farli scoprire, ma qualunque avversario in meno li avvicina alla vittoria. Gli altri giocatori della loro fazione li aiutano ad ottenere informazioni, costruirsi una copertura credibile e rallentare le indagini dei Popolani.
 \item I Negromanti sono molto pochi all'inizio della partita ma, man mano che la partita prosegue, alcuni dei Popolani morti diventeranno Spettri, a cui i Negromanti potranno assegnare un potere soprannaturale. Gli Spettri, così come gli altri membri della fazione, aiutano i Negromanti a mescolarsi nel villaggio e a manipolare le votazioni del giorno.
 \item I Popolani sono la fazione più numerosa, ma anche la più disorganizzata. All'inizio della partita non si conoscono fra loro, e molti dei loro poteri servono a ottenere informazioni sugli altri giocatori. Altri personaggi, invece, proteggono il villaggio dalla furia omicida dei Lupi e dai poteri degli Spettri e dei Negromanti.
\end{itemize}

\section{Consigli generali}

In questo gioco è fondamentale la capacità di bluffare, di crearsi una buona copertura e di coordinare le azioni della propria fazione. L'interazione con gli altri giocatori è importantissima, ed è ciò che rende divertente \emph{Lupus in tempo reale}.

In generale conviene tenere segreto il proprio ruolo. Se si gioca per i Lupi o per i Negromanti, ovviamente, venire scoperti comporta quasi certamente la morte sul rogo; se si parteggia per i Popolani, invece, esporsi significa diventare un bersaglio appetibile per gli avversari. Nonostante ciò, a volte ai Popolani conviene rivelare il proprio ruolo a persone fidate, per comunicare preziose informazioni in proprio possesso o per aiutare la fazione a coordinarsi. Bisogna ricordarsi, però, che mentire è fondamentale in questo gioco, e fidarsi degli altri è sempre una scelta da fare con molta cautela.

\subsection{Account anonimi}

Uno dei mezzi più importanti per comunicare con gli altri giocatori sono gli account anonimi sul forum. Spesso è importante parlarsi per ruolo e non per nome, e quindi occorre creare degli indirizzi associati al ruolo. Può essere utile anche creare account fasulli associati a ruoli delle altre fazioni, per aumentare la confusione fra gli avversari e sperare di ottenere da loro informazioni di qualche tipo. Questa strategia di solito è comune nelle fazioni di Lupi e Negromanti, ma anche i Popolani possono adottarla con successo.

L'uso delle email non è più necessario da quando è stato creato il forum, ma nel caso uno volesse creare degli indirizzi email da usare per la partita, una cosa importante da sapere è che gli account Gmail hanno delle falle di sicurezza. La procedura di recupero della password richiede informazioni come la data di creazione dell'account, gli indirizzi a cui si è scritto, e altre informazioni più o meno pubbliche per un account di recente creazione. Inoltre, Gmail effettua un controllo sull'indirizzo IP, che è comune a molti ambienti della scuola. Questo insieme di cose rende gli account Gmail molto facili da violare, e può succedere che un account violato incida in modo pesante sulla partita.
% In particolare, nella quinta edizione l'account anonimo del Diavolo è stato violato, e questa cosa ha dato al villaggio i nomi del Diavolo stesso, di un Avvocato del Diavolo e di uno Scrutatore. La presenza di un Diavolo morto e noto a tutto il villaggio è stata cruciale, perché i due Trasformisti presenti hanno agito su di lui con successo, ottenendo così informazioni fondamentali per i Popolani. A quel punto, la vittoria dei Popolani era quasi certa, ed infatti è avvenuta con un margine molto ampio.

\section{Strategie per i Popolani}

I Popolani devono cercare di scoprire quante più informazioni possibile, organizzare controlli sugli altri giocatori e uccidere dei nemici con le votazioni ogni qualvolta ne hanno l'occasione. Spesso conviene mandare al rogo qualcuno anche sulla base di semplici sospetti, perché non farlo, di fatto, permette ai Lupi di uccidere un Popolano in più.

Fra i Popolani, sono cruciali i ruoli che permettono di ottenere informazioni, come Divinatore, Mago, Stalker, Veggente, Voyeur. Tali ruoli dovrebbero cercare di non esporsi e di trovare il tempo e il modo giusti di comunicare al villaggio le loro informazioni. Sono le vittime preferite dei Lupi, se noti. È molto importante inoltre riuscire a coordinarsi efficacemente per effettuare controlli incrociati, ed in questo senso sono utili l'Esorcista, l'Espansivo, e l'Investigatore. Anche Stalker e Voyeur possono facilmente ottenere la fiducia di altri giocatori, indovinando i loro spostamenti.

Usare al meglio le abilità di Cacciatori, Messia, e Trasformisti può far pendere l'ago della bilancia in una direzione piuttosto che in un'altra, ribaltando le maggioranze per il rogo di giorno oppure recuperando abilità utili di personaggi uccisi. È importante per tali personaggi decidere se utilizzare la propria abilità presto, rischiando di commettere degli errori ma essendo certi di riuscirci, oppure attendere di avere sufficienti informazioni per farlo più efficacemente, ma rischiando di venire uccisi nel frattempo.

Infine, tra i ruoli rimasti la Guardia del corpo deve intuire chi sono gli alleati più preziosi e proteggerli, i Massoni sono utili per creare componenti connesse numerose e ridurre l'elenco dei potenziali avversari per esclusione, lo Sciamano può fornire un cruciale vantaggio spezzando Incantesimi necessari ai Negromanti, e la Spia può capire se qualcuno sta mentendo o se una votazione è stata manipolata.

\section{Strategie per i Lupi}

La fazione dei Lupi è meno numerosa e più organizzata di quella dei Popolani. Per raggiungere la vittoria essi devono cercare di ottenere quante più informazioni possibili sui Popolani, per capire quali possono manipolare e quali vadano invece eliminati. È inoltre importante ostacolare per quanto possibile lo scambio di informazioni tra i Popolani, ad esempio creando account falsi sul forum per generare rumore e rendere meno credibili gli account veri, oppure infiltrandosi (a proprio rischio) in una cerchia di Popolani noti diffondendo informazioni false o fuorvianti. Per i Lupi, avere una copertura credibile da Popolano, coadiuvata da un account anonimo di cui il villaggio si fida, può fare davvero la differenza fra vittoria e sconfitta.

Potendo coordinarsi in anticipo, i Lupi possono cercare di manipolare l'assemblea dei Popolani a proprio vantaggio (tante persone che dicono la stessa cosa sono di solito convincenti), ma questo comporta dei seri rischi nel caso alcune posizioni siano difficili da sostenere oppure evidentemente contraddittorie: proporre strategie sbagliate o difendere strenuamente membri della propria fazione su di cui ricadono accuse troppo gravi è fonte di forte sospetto. Può essere invece utile a volte contraddirsi a vicenda, rischiando magari di vedere uno dei propri alleati sul rogo ma allontanando le accuse da sé stessi.

\section{Strategie per i Negromanti}

La fazione dei Negromanti è quella che all'inizio è tendenzialmente meno numerosa delle altre. Si amplierà man mano che verranno creati Spettri. Per i Negromanti è relativamente facile nascondersi e mimetizzarsi nel resto del villaggio, e per fare ciò è fondamentale utilizzare bene gli Spettri.

L'obiettivo finale dei Negromanti è quello di far perdere il quorum al villaggio, dopodiché possono abbastanza facilmente conquistare la vittoria (esercizio). Per fare questo lo Spettro dell'Amnesia è molto utile, ma bisognerà stare attenti a bilanciare le strategie difensive percorribili con Confusione, Illusione e Occultamento, che permettono di allontanare i sospetti dai Negromanti, con le strategie offensive date da Amnesia e Morte, che avvicinano la fine della partita. L'Incantesimo della Telepatia è utile per ottenere informazioni a volte necessarie, mentre quello della Vita può essere adoperato in vari modi, per esempio per tentare un bluff da Messia, per usare un'abilità dei Popolani che può risultare molto utile, o per ottenere una maggioranza inaspettata al momento di votare. Infine, gli Incantesimi dell'Assoluzione e della Diffamazione sono molto utili verso la fine della partita, per proteggere i Negromanti dal rogo e per liberarsi di qualche nemico.


\end{document}
