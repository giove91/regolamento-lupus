\documentclass[a4paper,10pt]{article}


\usepackage[utf8]{inputenc}
\usepackage[italian]{babel}
\usepackage{amsmath}
\usepackage{amsthm}
\usepackage{fancyhdr}
\usepackage{amsfonts}
\usepackage{amssymb}
\usepackage{makeidx}
\usepackage[parfill]{parskip}
\usepackage{hyperref}


% * define a `\twoidxcolumn` based on `\twocolumn`:
\def\twoidxcolumn{%
%\clearpage
\global\columnwidth\textwidth
\global\advance\columnwidth-\columnsep
\global\divide\columnwidth\tw@
\global\hsize\columnwidth
\global\linewidth\columnwidth
\global\@twocolumntrue
\global\@firstcolumntrue
\col@number \tw@
%\@ifnextchar [\@topnewpage
\@floatplacement
}


\makeatletter
\def\@wrindex#1{%
   \protected@write\@indexfile{}%
      {\string\indexentry{#1}{\theenumi}}
      \endgroup
      \@esphack}

\makeatletter
\renewenvironment{theindex}
               {\twocolumn[\section*{Indice delle parole chiave presenti nella
Sezione \ref{faq}}]%
                \@mkboth{\MakeUppercase\indexname}%
                        {\MakeUppercase\indexname}%
                \thispagestyle{plain}\parindent\z@
                \parskip\z@ \@plus .3\p@\relax
                \columnseprule \z@
                \columnsep 35\p@
                \let\item\@idxitem}
               {}
\makeatother

\makeindex
% [title=Indice dei ruoli presenti nella Sezione \ref{faq},columns=3]


\usepackage[top=1.15in, bottom=1.15in, left=1.3in, right=1.3in]{geometry}

% \topmargin -1cm
% \oddsidemargin -0.5cm
% %\evensidemargin	-1cm
% \textwidth 17cm


\newcommand{\smallspace}{\vskip0.3cm}

% Title Page
\title{Lupus in tempo reale\\ Consigli di gioco}
\author{Alessandro Iraci, Giovanni Mascellani, Giovanni Paolini, Leonardo Tolomeo}

\begin{document}
\maketitle

% \section{Introduzione}
% 
% Quella che segue è una descrizione molto sommaria delle regole di \emph{Lupus in tempo reale}, che serve
% a chi non ha mai partecipato per capire le meccaniche di questo gioco. Per partecipare, comunque, è molto
% importante leggere il regolamento completo.
% 
% All'inizio del gioco, ad ogni giocatore viene segretamente assegnato un ruolo,
% che determina anche la fazione di cui farà parte.
% Ci sono tre fazioni: i Popolani, i Lupi e i Negromanti. La partita termina
% quando rimane una sola fazione in gioco, che viene dichiarata vincitrice.
% Un giocatore vince se la sua fazione vince, a prescindere dal fatto che
% questi sia vivo o morto al termine della partita.
% 
% I turni di gioco sono divisi in giorni e notti. Durante il giorno, ciascuno
% ha il diritto di votare un giocatore da mandare al rogo. 
% Al termine del giorno, salvo eccezioni, muore il giocatore con più voti.
% Durante la notte i giocatori hanno diritto ad usare i loro poteri
% speciali, che dipendono dal ruolo. Un elenco dei ruoli e dei rispettivi
% poteri speciali è disponibile nel regolamento.
% 
% Quella che segue è una descrizione sommaria delle fazioni.
% \begin{itemize}
%  \item I Lupi si accordano fra loro per uccidere un personaggio a loro scelta. 
%     I loro obiettivi principali sono i Popolani con poteri che potrebbero farli scoprire,
%     ma qualunque avversario in meno li avvicina alla vittoria.
%     Gli altri giocatori della loro fazione li aiutano ad
%     ottenere informazioni, costruirsi una copertura credibile e rallentare le indagini dei Popolani.
%  \item I Negromanti possono, ogni due notti, scegliere un personaggio morto e trasformarlo in uno Spettro.
%     Questi si aggiungerà alla loro fazione e otterrà un nuovo potere speciale.
%     Gli Spettri, così come gli altri membri della fazione, aiutano i Negromanti a mescolarsi nel villaggio
%     e a manipolare le votazioni del giorno.
%     I Negromanti non avranno abbastanza tempo per creare tutti gli Spettri disponibili, e devono valutare
%     con attenzione il momento in cui risvegliare l'ultimo Spettro, che ha il potere di uccidere.
%  \item I Popolani sono la fazione più numerosa, ma anche la più disorganizzata. All'inizio della 
%     partita non si conoscono fra loro, e molti dei loro poteri servono a ottenere informazioni sugli 
%     altri giocatori. Altri personaggi, invece, proteggono il villaggio dalla furia omicida dei Lupi
%     e dai poteri degli Spettri e dei Negromanti.
%     I Popolani devono puntare a scoprire quante più informazioni possibile, organizzare
%     controlli a tappeto sugli altri giocatori e uccidere dei nemici con le votazioni ogni
%     qualvolta ne hanno l'occasione. Spesso conviene mandare al rogo qualcuno anche sulla base di
%     semplici sospetti, perché non farlo, di fatto, permette ai Lupi di uccidere un Popolano
%     in più.
% \end{itemize}
% 
% In questo gioco è fondamentale la capacità di mentire, di crearsi una copertura
% perfetta e di coordinare le azioni della propria fazione. L'interazione con gli altri
% giocatori è importantissima, ed è ciò che rende divertente \emph{Lupus in tempo reale}.
% 
% In generale conviene tenere segreto il proprio ruolo. Se si gioca per i Lupi o per i Negromanti,
% ovviamente, venire scoperti comporta quasi certamente la morte sul rogo; se si parteggia per i
% Popolani, invece, esporsi significa diventare un bersaglio appetibile per gli avversari.
% Nonostante ciò, a volte ai Popolani conviene rivelare il proprio ruolo a persone fidate, per
% comunicare preziose informazioni in proprio possesso o per aiutare la fazione a coordinarsi.
% Bisogna ricordarsi, però, che mentire è fondamentale in questo gioco, e fidarsi degli altri è
% sempre una scelta da fare con molta cautela.

\section{Consigli generali}

In questo gioco è fondamentale la capacità di mentire, di crearsi una copertura perfetta e di coordinare le azioni della propria fazione. L'interazione con gli altri
giocatori è importantissima, ed è ciò che rende divertente \emph{Lupus in tempo reale}.

In generale conviene tenere segreto il proprio ruolo. Se si gioca per i Lupi o per i Negromanti, ovviamente, venire scoperti comporta quasi certamente la morte sul rogo; se si parteggia per i Popolani, invece, esporsi significa diventare un bersaglio appetibile per gli avversari.
Nonostante ciò, a volte ai Popolani conviene rivelare il proprio ruolo a persone fidate, per comunicare preziose informazioni in proprio possesso o per aiutare la fazione a coordinarsi. Bisogna ricordarsi, però, che mentire è fondamentale in questo gioco, e fidarsi degli altri è sempre una scelta da fare con molta cautela.

\subsection{Account anonimi}

Uno dei mezzi più importanti per comunicare con gli altri giocatori sono gli indirizzi e-mail anonimi. Spesso è importante parlarsi per ruolo e non per nome, e quindi occorre creare degli indirizzi associati al ruolo. Può essere utile anche creare account fasulli associati a ruoli delle altre fazioni, per aumentare la confusione fra gli avversari e sperare di ottenere da loro informazioni di qualche tipo. Questa strategia di solito è comune nelle fazioni di Lupi e Negromanti, ma anche i Popolani possono adottarla con successo.

Una cosa importante da sapere è che gli account Gmail hanno delle falle di sicurezza. La procedura di recupero della password richiede informazioni come la data di creazione dell'account, gli indirizzi a cui si è scritto, e altre informazioni più o meno pubbliche. Inoltre, Gmail effettua un controllo sull'indirizzo IP, che è comune a molti ambienti della scuola. Questo insieme di cose rende gli account Gmail molto facili da violare, e può succedere che un account violato incida in modo molto pesante sulla partita.
In particolare, nella quinta edizione l'account anonimo del Diavolo è stato violato, e questa cosa ha dato al villaggio i nomi del Diavolo stesso, di un Avvocato del Diavolo e di uno Scrutatore. La presenza di un Diavolo morto e noto a tutto il villaggio è stata cruciale, perché i due Trasformisti presenti hanno agito su di lui con successo, ottenendo così informazioni fondamentali per i Popolani. A quel punto, la vittoria dei Popolani era quasi certa, ed infatti è avvenuta con un margine molto ampio.

Un'altra cosa da osservare è che lo stile di scrittura può essere usato per ottenere informazioni. Non è semplice, ma nella scorsa partita questo ha permesso ad un giocatore (Andrea Bianchi scienze, ndr), applicatosi con pazienza certosina nel confrontare le e-mail arrivate al villaggio, di scoprire il ruolo di quattro persone ad un buon livello di sicurezza.

\section{Strategie per i Popolani}

I Popolani sono la fazione più numerosa, ma anche la più disorganizzata. All'inizio della partita non si conoscono fra loro, e molti dei loro poteri servono a ottenere informazioni sugli altri giocatori. Altri personaggi, invece, proteggono il villaggio dalla furia omicida dei Lupi e dai poteri degli Spettri e dei Negromanti.
I Popolani devono puntare a scoprire quante più informazioni possibile, organizzare controlli a tappeto sugli altri giocatori e uccidere dei nemici con le votazioni ogni qualvolta ne hanno l'occasione. Spesso conviene mandare al rogo qualcuno anche sulla base di semplici sospetti, perché non farlo, di fatto, permette ai Lupi di uccidere un Popolano in più.

I ruoli più importanti per i Popolani sono quelli che permettono di ottenere informazioni, come Stalker, Voyeur e Veggente. Tali ruoli dovrebbero cercare di non esporsi e di trovare il tempo e il modo giusti di comunicare al villaggio le loro informazioni. Sono le vittime preferite dei Lupi, se noti.

Una cosa che può risultare utile in certi momenti è quella di organizzare dei controlli sui voti alla ricerca di Ipnotisti, scegliendo qualche persona a caso e facendola votare in modo diverso da tutti gli altri.

\section{Strategie per i Lupi}

La fazione dei Lupi è meno numerosa e più organizzata di quella dei Popolani. Il loro potere tendenzialmente decresce man mano che va avanti la partita, quindi devono puntare quanto più possibile ad una vittoria rapida.

I Lupi possono autenticare facilmente gli account anonimi che creano, semplicemente indicando la persona che verrà uccisa il giorno successivo. Farlo pubblicamente permette alla Guardia del Corpo di intervenire, perciò i Lupi scriveranno all'ultimo minuto chi uccideranno, eventualmente cifrando il messaggio in qualche modo. In questo modo i loro alleati avranno un contatto certo molto presto. Starà a loro poi riuscire a convincere i Lupi del fatto che stiano davvero dalla stessa parte.

\section{Strategie per i Negromanti}

La fazione dei Negromanti è quella che all'inizio è tendenzialmente meno numerosa delle altre. Si amplierà man mano che verranno creati Spettri. Per i Negromanti è relativamente facile nascondersi e mimetizzarsi nel resto del villaggio.

Di fondamentale importanza è trovare gli Ipnotisti il prima possibile, e magari impedire che vengano uccisi dai Lupi, utilizzando opportunamente lo Spettro con il potere dell'Occultamento. L'Ipnotista è un ruolo chiave per la fazione. Deve scegliere il momento giusto per iniziare ad ipnotizzare. Non essendo un'azione reversibile, è facile organizzare dei controlli e notare che ci sono due persone che hanno votato nello stesso modo per diversi giorni consecutivi. Agire troppo presto fa correre il grosso rischio di esporsi troppo, ma d'altro canto agire troppo tardi è inutile.

\end{document}
