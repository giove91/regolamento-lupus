\documentclass[a4paper,10pt]{article}


\usepackage[utf8]{inputenc}
\usepackage[italian]{babel}
\usepackage{amsmath}
\usepackage{amsthm}
\usepackage{fancyhdr}
\usepackage{amsfonts}
\usepackage{amssymb}
\usepackage{makeidx}
\usepackage[parfill]{parskip}
\usepackage{hyperref}


% * define a `\twoidxcolumn` based on `\twocolumn`:
\def\twoidxcolumn{%
%\clearpage
\global\columnwidth\textwidth
\global\advance\columnwidth-\columnsep
\global\divide\columnwidth\tw@
\global\hsize\columnwidth
\global\linewidth\columnwidth
\global\@twocolumntrue
\global\@firstcolumntrue
\col@number \tw@
%\@ifnextchar [\@topnewpage
\@floatplacement
}


\makeatletter
\def\@wrindex#1{%
   \protected@write\@indexfile{}%
      {\string\indexentry{#1}{\theenumi}}
      \endgroup
      \@esphack}

\makeatletter
\renewenvironment{theindex}
               {\twocolumn[\section*{Indice delle parole chiave presenti nella
Sezione \ref{faq}}]%
                \@mkboth{\MakeUppercase\indexname}%
                        {\MakeUppercase\indexname}%
                \thispagestyle{plain}\parindent\z@
                \parskip\z@ \@plus .3\p@\relax
                \columnseprule \z@
                \columnsep 35\p@
                \let\item\@idxitem}
               {}
\makeatother

\makeindex
% [title=Indice dei ruoli presenti nella Sezione \ref{faq},columns=3]


\usepackage[top=1.15in, bottom=1.15in, left=1.3in, right=1.3in]{geometry}

% \topmargin -1cm
% \oddsidemargin -0.5cm
% %\evensidemargin	-1cm
% \textwidth 17cm


\newcommand{\smallspace}{\vskip0.3cm}

% Title Page
\title{Introduzione a Lupus in tempo reale}
\author{Alessandro Iraci}

\begin{document}
\maketitle

\section{Introduzione}

Quella che segue è una descrizione molto sommaria delle regole di \emph{Lupus in tempo reale}, che serve a chi non ha mai partecipato per capire le meccaniche di questo gioco. Per partecipare, comunque, è molto importante leggere il regolamento completo.

All'inizio del gioco, ad ogni giocatore viene segretamente assegnato un ruolo, che determina anche la fazione di cui farà parte. Ci sono due fazioni: i Villici e i Villani. La partita termina quando rimane una sola fazione in gioco, che viene dichiarata vincitrice. Un giocatore vince se la sua fazione vince, a prescindere dal fatto che questi sia vivo o morto al termine della partita.
I turni di gioco sono divisi in giorni e notti. Durante il giorno, ciascuno ha il diritto di votare un giocatore da mandare al rogo. Al termine del giorno, se un giocatore ha ottenuto più della metà dei voti totali, questi muore. Durante la notte i giocatori hanno diritto ad usare le loro abilità e poteri, che dipendono dal ruolo. Un elenco dei ruoli e delle rispettive abilità e poteri è disponibile nel regolamento.

Quella che segue è una rapida descrizione delle fazioni.

\begin{itemize}
	
	\item I Villici sono la fazione più numerosa, ma anche la più disorganizzata. All'inizio della partita non si conoscono fra loro, e molte delle loro abilità servono a ottenere informazioni sugli altri giocatori. Altri personaggi, invece, contribuiscono alla vittoria ostacolando la fazione avversaria.
	
 	\item I Villani sono molti meno dei Villici all'inizio della partita, ma si conoscono tutti fra di loro e quindi per loro è molto più facile coordinarsi. Ogni notte, i Lupi si accordano fra loro per uccidere un personaggio a loro scelta, tendenzialmente cercando di far fuori i Villici con abilità che potrebbero farli scoprire, anche se in realtà qualunque avversario in meno li avvicina alla vittoria. Gli altri giocatori della loro fazione li aiutano ad ottenere informazioni, costruirsi una copertura credibile e rallentare le indagini dei Villici.
\end{itemize}


\section{Consigli generali}

In questo gioco è fondamentale la capacità di bluffare, di crearsi una buona copertura e di coordinare le azioni della propria fazione. L'interazione con gli altri giocatori è importantissima, ed è ciò che rende divertente \emph{Lupus in tempo reale}.

In generale conviene tenere segreto il proprio ruolo. Se si gioca per i Villani, ovviamente, venire scoperti comporta quasi certamente la morte sul rogo; se si parteggia per i Villici, invece, esporsi significa diventare un bersaglio appetibile per gli avversari. Nonostante ciò, a volte ai Villici conviene rivelare il proprio ruolo a persone fidate, per comunicare preziose informazioni in proprio possesso o per aiutare la fazione a coordinarsi. Bisogna ricordarsi, però, che mentire è fondamentale in questo gioco, e fidarsi degli altri è sempre una scelta da fare con molta cautela.

\subsection{Account anonimi}

Uno dei mezzi più importanti per comunicare con gli altri giocatori sono gli account anonimi sul forum. Spesso è importante parlarsi per ruolo e non per nome, e quindi occorre creare degli indirizzi associati al ruolo. Può essere utile anche creare account fasulli associati a ruoli diversi dal proprio, per aumentare la confusione fra gli avversari e sperare di ottenere da loro informazioni di qualche tipo. Questa strategia di solito è comune nella fazione dei Villani, ma anche i Villici possono adottarla con successo.

Fidarsi di un account anonimo è una scelta da non fare alla leggera. Molti di essi infatti sono fasulli, creati con il solo scopo di estorcere informazioni agli altri giocatori, soprattutto ai più inesperti. Caratteristiche comuni fra gli account fasulli sono quelle di presentarsi dando informazioni vaghe sul ruolo del destinatario (per esempio il colore dell'aura, o l'appartenenza ad una certa lista di ruoli), sperando che siano sufficienti per ottenere almeno il minimo di fiducia necessario per chiedere in cambio informazioni altrettanto vaghe (per esempio la misticità, o la frequenza di uso dell'abilità).

A volte però un account anonimo può celare un giocatore sincero e bene intenzionato. Questi in genere si distinguono per l'esattezza e la precisione delle informazioni fornite (per esempio la persona su cui si ha agito) ed è bene non ignorarli. D'altro canto, non è da escludere che anche in questi casi si tratti di informazioni false la cui speranza di successo si basa sul gran numero di tentativi. In generale, per essere ragionevolmente certi della veridicità di un account serve che questo fornisca informazioni esatte più volte, rendendo improbabile che stia andando a tentativi.

Riuscire a capire di chi fidarsi è cruciale in \emph{Lupus in tempo reale}: fornire troppe informazioni alle persone sbagliate può essere fatale, ma tenerle tutte per sé rende impossibile coordinarsi e diminuisce le probabilità di vittoria.

%L'uso delle email non è più necessario da quando è stato creato il forum, ma nel caso uno volesse creare degli indirizzi email da usare per la partita, una cosa importante da sapere è che gli account Gmail hanno delle falle di sicurezza. La procedura di recupero della password richiede informazioni come la data di creazione dell'account, gli indirizzi a cui si è scritto, e altre informazioni più o meno pubbliche per un account di recente creazione. Inoltre, Gmail effettua un controllo sull'indirizzo IP, che è comune a molti ambienti della scuola. Questo insieme di cose rende gli account Gmail molto facili da violare, e può succedere che un account violato incida in modo pesante sulla partita.
% In particolare, nella quinta edizione l'account anonimo del Diavolo è stato violato, e questa cosa ha dato al villaggio i nomi del Diavolo stesso, di un Avvocato del Diavolo e di uno Scrutatore. La presenza di un Diavolo morto e noto a tutto il villaggio è stata cruciale, perché i due Trasformisti presenti hanno agito su di lui con successo, ottenendo così informazioni fondamentali per i Popolani. A quel punto, la vittoria dei Popolani era quasi certa, ed infatti è avvenuta con un margine molto ampio.

\section{Strategie per i Villici}

I Villici devono cercare di scoprire quante più informazioni possibile, organizzare controlli sugli altri giocatori e uccidere dei nemici con le votazioni ogni qualvolta ne hanno l'occasione. Spesso conviene mandare al rogo qualcuno anche sulla base di semplici sospetti, perché non farlo, di fatto, permette ai Lupi di uccidere un Villico in più.

Fra i Villici, sono cruciali i ruoli che permettono di ottenere informazioni precise, come Divinatore, Stalker, Veggente, Voyeur. Tali ruoli dovrebbero cercare di non esporsi e di trovare il tempo e il modo giusti di comunicare al villaggio le loro informazioni. Sono le vittime preferite dei Lupi, se noti. È molto importante inoltre riuscire a coordinarsi efficacemente per effettuare controlli incrociati, ed in questo senso sono utili l'Esorcista, l'Espansivo, e il Medium. Anche Stalker e Voyeur possono facilmente ottenere la fiducia di altri giocatori, indovinando i loro spostamenti.

Usare al meglio le abilità di Cacciatore, Guardia, Messia, e Trasformista può far pendere l'ago della bilancia in una direzione piuttosto che in un'altra, ribaltando le maggioranze per il rogo di giorno oppure recuperando abilità utili di personaggi uccisi. È importante per Messia e Trasformista decidere se utilizzare la propria abilità presto, rischiando di commettere degli errori ma essendo certi di riuscirci, oppure attendere di avere sufficienti informazioni per farlo più efficacemente, ma rischiando di venire uccisi nel frattempo.

Infine, l'Apprendista può essere un prezioso jolly, utilizzando l'abilità giusta al momento giusto, per far saltare un bluff in modo imprevisto oppure ottenere qualche informazione cruciale che non sarebbe possibile avere altrimenti. 

\subsection{Strategie ruolo per ruolo}

\begin{itemize}
	
	\item {\bf Apprendista}. L'Apprendista non ha ancora scelto cosa fare della sua vita, e per questo prova un po' di tutto. Le strategie di gioco di questo personaggio dipendono molto dall'elenco dei ruoli che gli vengono assegnati, e sicuramente leggere tutti i consigli di gioco dei ruoli del vostro elenco può aiutare a capire a cosa serve una determinata abilità.
	
	Per esempio, se si hanno abilità che permettono di mettersi in contatto con altri Villici (per esempio l'Espansivo, ma anche Divinatore, Voyeur, Diavolo, eccetera) potrebbero essere usate all'inizio per creare da subito una componente connessa di Villici, mentre alla fine potrebbe essere necessario trovare informazioni su uno specifico partecipante. Infine, è cruciale valutare bene quando utilizzare poteri one-shot come quelli di Cacciatore, Messia, e Trasformista: sono molto forti, ma una volta utilizzati con successo precludono l'uso del resto della propria lista, quindi bisogna valutare bene se sia più opportuno utilizzarli presto, quando possono fare la differenza, o più tardi, dopo aver esaurito le altre abilità utili.
	
	In ogni caso, seguite il flusso della partita e lasciatevi ispirare! 

	\item {\bf Cacciatore}. Avendo aura nera e non essendo mistico, il Cacciatore è difficilmente distinguibile da un Lupo, e pertanto non è da escludere che venga scambiato per uno di essi. 
	
	Il Cacciatore ha un compito non facile: deve intuire chi sono i giocatori più a rischio di essere sbranati da un Lupo e proteggerli. Molto utile per aumentare l'aspettativa di vita di ruoli come Divinatore, Espansivo, Stalker, Veggente e Voyeur, il Cacciatore deve cercare di proteggere la vittima più appetibile per i Lupi, ma così facendo rischia di imbattersi in un Assassino, e pertanto deve o stare molto attento a mantenere l'anonimato, oppure dirigersi da qualche bersaglio interessante per i Lupi, ma non completamente ovvio.
	
	\item {\bf Contadino}. Il Contadino, non avendo alcuna abilità, non ha molti modi di ottenere informazioni. D'altro canto, non essendo un bersaglio interessante per i Villani, potrebbe decidere di dichiarare il suo ruolo, apertamente o ad un numero ristretto di giocatori, e potenzialmente fungere da collegamento fra ruoli con abilità più interessanti. Altrimenti, tenere segreto il proprio ruolo ma rendersi molto partecipe alle assemblee può essere un buon modo per attirare l'interesse dei Lupi, deviando l'attenzione da altri personaggi. In fondo alla fine conta la vittoria della Fazione, non la propria vita.
	
	\item {\bf Divinatore}. Il Divinatore è particolarmente forte in congiunzione con altri ruoli che scoprono informazioni, perché è facilmente in grado di scoprire se qualcuno ha mentito. Inizialmente ha senso usare le frasi per connettersi ad un buono certo con cui scambiare informazioni: indovinare con esattezza il ruolo di una persona, infatti, è spesso una prova sufficiente che si è stati contattati proprio da un Divinatore, soprattutto se questo accade all'inizio della partita quando le informazioni disponibili sono poche.
	
	Più avanti nel gioco, quando cominciano a sorgere dei sospetti, è utile verificare se sono fondati. In mancanza di alternative migliori, un Divinatore può andare alla cieca cercando dei Lupi o dei Negromanti, essendo questi solitamente i ruoli più rappresentati delle fazioni avversarie. Avendo un'alta capacità di confutare le dichiarazioni fasulle, il Divinatore è una delle vittime preferite dei Lupi, e pertanto è bene che tenga segreta la propria identità.
	
	\item {\bf Esorcista}. Il compito dell'Esorcista è quello di limitare i poteri degli Spettri, e pertanto all'inizio della partita non ha molto da fare. Potrebbe essere comunque utile uscire quanto più spesso possibile, anche senza Spettri in giro, per sperare di essere intercettato da uno Stalker o da un Voyeur con cui connettersi. L'Esorcista è infatti maggiormente efficiente in collaborazione con i ruoli che ottengono informazioni, per certificare che tali informazioni siano vere e non compromesse dall'azione di alcuni Spettri.
	
	L'Esorcista è relativamente innocuo per i Villani all'inizio della partita, ma man mano che il numero di Spettri aumenta, diventa una vittima via via più appetibile per i Lupi. Non potendo proteggere sé stesso, ancora più che per gli altri ruoli, vale il fatto che dichiararsi pubblicamente rende facili prede dello Spettro della Morte.
	
	\item {\bf Espansivo}. Quello dell'Espansivo è un ruolo molto forte, ma anche estremamente delicato. 
	
	Il suo compito principale è quello di radunare i membri della fazioni dei Villici e guidarli verso la vittoria. Usando la sua abilità, viene certificato dal gioco come membro di questa fazione, pertanto 
	la persona che ottiene questa informazione si trova una delle due seguenti situazioni:
	
	\begin{enumerate}
	    \item È un membro della fazione dei Villici. In questo caso, il villico è molto contento di rivelare a un suo compagno di fazione tutte le informazioni che possiede, ricevendo in cambio indicazioni utili su come procedere e, soprattutto, qualcuno di fidato con cui parlare per il resto della partita.
	    \item È un membro della fazione dei Villani. In particolare, è una Fattucchiera o un Alcolista, oppure un altro ruolo coperto da una Fattucchiera o dallo Spettro della Confusione. Il Villano in questione dovrà inventarsi un alibi per convincere l'Espansivo di rientrare nella categoria precedente.
	\end{enumerate}
	
	Attenzione! L'Espansivo non ha la certezza di interagire con un membro della sua fazione, dovrà quindi filtrare le informazioni che riceve e limitare quelle che restituisce. In particolare, la fazione dei Villani ha un certo interesse ad eliminare l'Espansivo e lasciare i Villici soli e confusi.

    L'Espansivo dovrà inoltre trovare il momento e il modo di uscire allo scoperto, così da poter diffondere a tutto il villaggio le informazioni che la sua rete di conoscenze ha prodotto e guidare la sua fazione a deporre un voto consapevole.
	
	\item {\bf Guardia}. La Guardia è in grado di proteggere un giocatore non solo dai Lupi, ma anche da eventuali Assassini o Sequestratori. Come per il Cacciatore, andare dai personaggi più esposti dei Villici aiuta ad aumentare l'aspettativa di vita di ruoli come Divinatore, Espansivo, Stalker, Veggente e Voyeur, ma può anche essere utilizzata come modo per garantire che il bersaglio non venga bloccato da un Sequestratore bene informato, ed ottenere inoltre preziose informazioni.
	
    Occhio però a nemici come l'Assassino e lo Spettro dell'Invisibilità. Proteggere sempre la vittima più appetibile rende facili vittime del primo, mentre la presenza del secondo rende potenzialmente inutile la Guardia: un nemico invisibile non viene bloccato, pertanto per avere certezza di riuscita è bene essere in contatto con un Esorcista.
	
	Infine, la Guardia ha a sua volta una piccola capacità di ottenere informazioni, che saltuariamente può risultare utile per scoprire se qualcuno sta dicendo la verità oppure no, ed è potenzialmente molto forte se utilizzata in modo coordinato con un Voyeur.
	
	\item {\bf Mago}. Il mago trova ogni notte un'informazione non priva di importanza, però non banale da utilizzare: i ruoli mistici sono circa la metà, e la misticità non è correlata alla fazione (a differenza dell'aura vista dal Veggente) né a qualche altra caratteristica del ruolo, ma solo all'ambientazione: i ruoli mistici sono quelli collegati con la magia e la spiritualità. Il rischio corso dal Mago, e che dovrebbe cercare di evitare, è quello di non riuscire ad usare le informazioni trovate prima della fine della partita, e dunque di sprecare la propria abilità.
	
	Come per molti altri ruoli, il Mago potrebbe trovare utile l'uso di un account anonimo per annunci pubblici e comunicazioni private, però, essendo il suo ruolo non tra i più temibili per i Lupi, nel caso in cui una sua informazione sia davvero importante, per ottenere più credibilità il Mago può rivelarsi di persona (al villaggio o ad un buono certo) senza troppe preoccupazioni. Ad esempio se una sera qualcuno sta per essere messo al rogo come Sequestratore ma il Mago sa che è mistico, oppure se qualcuno sta per salvarsi dal rogo dichiarandosi Esorcista ma il Mago sa che non è mistico, è plausibile che il Mago voglia dichiararsi pubblicamente e rivelare quello che sa.

	\item {\bf Medium}. Il Medium ha due funzioni, blocca gli spettri e contemporaneamente scopre il ruolo dei giocatori morti. Per il Medium è molto utile capire chi sono gli Spettri, quindi aguzzate la vista e provate a capire nelle assemblee quale dei morti ha cambiato fazione! Anche scoprire il ruolo dei morti può essere utile per confermare o smentire delle versioni dei giocatori, o semplicemente per ottenere informazioni sul villaggio.

	\item {\bf Messia}. Dato che può usare la sua abilità solo una volta a partita, il Messia ha interesse a resuscitare un Villico con un ruolo importante, cioè dotato di un’abilità che gli permette di raccogliere più informazioni possibili. 
	
	Ogni volta che un villico muore nella notte, quindi, il Messia potrebbe volergli chiedere, di solito anonimamente, se da vivo avesse un ruolo importante. In questo modo, tra l’altro, il Messia raccoglie informazioni sui morti. Se un morto dichiara di avere un ruolo poco importante, potrebbe essere uno Spettro: in quel caso il Messia può scegliere in generale di non agire (per non correre il rischio di resuscitare Contadini) ma di segnalare la cosa sul forum. Naturalmente, se agisce e fallisce, può a maggior ragione scegliere di comunicarlo a tutti gli altri giocatori.
	
	Se invece il Messia ha successo su un morto, è certo della sua bontà e può decidere di dichiarargli la propria identità reale, con l’idea che, anche da semplice contadino, possa giocare accedendo alle informazioni note al risorto ed eventualmente essere incluso in una componente connessa di Villici.
	
	\item {\bf Stalker}. Uno degli aspetti più importanti dell'informazione ottenuta di notte dallo Stalker è che è difficilmente indovinabile. Questo fa sì che lo Stalker possa confermarsi tale in modo relativamente affidabile, semplicemente comunicando alla persona visitata da chi la si è vista andare la notte. L'unico caveat è che anche Sequestratore e Stregone possono scoprire un movimento, ma fanno anche fallire l'azione che l'ha generato: quindi se un sedicente Stalker prova a confermarsi ad una persona in questo modo, quella persona ha ragionevole motivo di credergli (ovviamente a patto che l'informazione sia corretta), a meno che non abbia fallito quella notte.

    Presentarsi di persona per confermarsi è pericoloso, siccome rischia di comunicare ai Villani l'identità di uno Stalker; a meno di non essere ragionevolmente sicuri che la persona visitata sia un Villico, utilizzare un account anonimo è generalmente più prudente.

    Il movimento in sé può fornire molte informazioni. Ad esempio, vedere qualcuno recarsi da una persona che è stata ritrovata morta all'alba dovrebbe sollevare più di un qualche sospetto; vedere qualcuno andare da un morto restringe di molto suoi possibili ruoli. Naturalmente, se in qualche modo si conosce il ruolo di un personaggio, le sue azioni possono rivelare molto di più; ad esempio un personaggio visitato da un Sequestratore è (salvo casi assolutamente improbabili) un Villico.

    Infine, si noti che non c'è alcun modo in cui uno Stalker possa scoprire un movimento che non è avvenuto, quindi se scopre un movimento lo può considerare un'informazione certamente vera. %Tuttavia, se scopre che qualcuno non si è mosso, è possibile (ma non necessariamente probabile) che quell'informazione sia falsa, se c'è di mezzo lo Spettro dell'Invisibilità.

	\item {\bf Trasformista}. Dato che può cambiare identità solo una volta, il Trasformista ha interesse a diventare in un villico con un ruolo importante, cioè dotato di un’abilità che gli permette di raccogliere più informazioni possibili.
    
    Ogni volta che un villico muore nella notte, quindi,  il Trasformista potrebbe volergli chiedere, di solito anonimamente, che ruolo avesse da vivo, o più in generale se avesse un ruolo importante. In questo modo, tra l’altro, il Trasformista raccoglie informazioni sui morti.
    
    Gli sforzi del Trasformista nella ricerca di un nuovo ruolo rischiano di essere vanificati dall’intervento della Fattucchiera, e dunque, se ritiene probabile che i Villani vogliano farla agire su un morto, può valutare di tardare ad agire o cambiare bersaglio, per minimizzare il rischio di svegliarsi come semplice Contadino.
	
	\item {\bf Veggente}. Il Veggente scopre ogni notte il colore dell'aura di un giocatore a sua scelta.
    Questa è un'informazione molto importante, dato che quasi tutti i Villici hanno aura bianca, mentre quasi tutti i Villani hanno aura nera. Grazie al Veggente quindi i Villici possono individuare chi bruciare al rogo.
    
    Il Veggente deve riuscire a convincere l'assemblea di quello che ha visto, senza però esporsi troppo e rischiare di essere sbranato dai lupi. Una cosa che si potrebbe fare è creare un account anonimo da Veggente sul forum e scrivere cosa si è visto durante la notte. Se all'inizio magari sarà soltanto uno delle decine di account che si fingono Veggenti, potra pian piano assumere credibilità e diventare un faro nella notte per tutti i Villici spaesati!
    
    In alternativa, dato che un responso bianco indica che probabilmente la persona scrutata nella propria sfera di cristallo è dei Villici, il Veggente può decidere di rivelare la propria identità solo alle persone viste, imitando un po' l'Espansivo. Ovviamente probabile non significa certo, e questa strategia fa sì che il Veggente potrebbe incautamente rivelare la propria identità per esempio ad una Fattucchiera, errore che spesso si paga con la vita.
	
	\item {\bf Voyeur}. La forza del Voyeur è la sua capacità di creare connessioni con altri giocatori; infatti può scoprire il movimento di una persona, che è un'informazione difficile da indovinare essendoci un sacco di possibilità. Come conseguenza, può confermarsi a qualcuno in modo relativamente affidabile semplicemente comunicandogli dove l'ha visto andare. Sebbene anche Stregone e Sequestratore possano scoprire un movimento, questi fanno anche fallire l'abilità che lo ha generato, quindi, se un giocatore riceve un tentativo di conferma da parte di un sedicente Voyeur ed ha utilizzato la propria abilità con successo durante la stessa notte, ha generalmente motivo di credergli.
	
	Presentarsi di persona per confermarsi è molto pericoloso, siccome rischia di comunicare ai Villani l'identità di un Voyeur e di invitare spiacevoli visite notturne; generalmente è più prudente utilizzare un account anonimo a meno di non essere ragionevolmente sicuri che la persona con cui si ci sta connettendo sia un Villico. 

    Se la persona visitata muore di notte, al Voyeur dovrebbe nascere il forte sospetto che una delle altre persone che l'ha visitata sia un Lupo, o magari un Assassino, specialmente se c'è stato un solo visitatore; tuttavia non è sempre detto, siccome ad esempio un Lupo può essere reso invisibile. In ogni caso, se ci si convince di aver trovato un Lupo e di averne le prove, potrebbe valere la pena (per quanto sia piuttosto rischioso) di provare a farlo condannare affermando pubblicamente di essere un Voyeur, specialmente se a quel punto della partita si è già riusciti a connettere un buon numero di persone che possano avvalorare tale affermazione.

%    Vale la pena notare che, in aggiunta all'informazione su chi ha visitato una certa casa, il Voyeur ottiene anche l'informazione negativa su chi non l'ha visitata (ma attenzione che è possibile che tale informazione non sia del tutto veritiera, se c'è di mezzo Invisibilità); questa informazione non è di solito troppo utile, ma potrebbe trovare applicazione, ad esempio, per sbugiardare qualche bluff o magari qualche account anonimo che millanti azioni farlocche.

    Infine, come Cacciatore e Guardia, il Voyeur potrebbe voler visitare personaggi che hanno destato particolare interesse nei giorni precedenti, per massimizzare le probabilità di osservare qualcuno. Chiaramente questo significa rischiare di incappare in Assassini o Stregoni, quindi bisogna pesare ogni cosa quando si sceglie su chi agire.
\end{itemize}

\section{Strategie per i Villani}

    La fazione dei Villani è meno numerosa e più organizzata di quella dei Villici. Per raggiungere la vittoria essi devono cercare di ottenere quante più informazioni possibili sugli avversari, per capire quali possono manipolare e quali vadano invece eliminati. È inoltre importante ostacolare per quanto possibile lo scambio di informazioni tra i Villici, ad esempio creando account falsi sul forum per generare rumore e rendere meno credibili gli account veri, oppure infiltrandosi (a proprio rischio) in una cerchia di Villici noti diffondendo informazioni false o fuorvianti. Per i Villani, avere una copertura credibile da Villico, coadiuvata da un account anonimo di cui il villaggio si fida, può fare davvero la differenza fra vittoria e sconfitta.

    Potendo coordinarsi in anticipo, i Villani possono cercare di manipolare l'assemblea del villaggio a proprio vantaggio (tante persone che dicono la stessa cosa sono di solito convincenti), ma questo comporta dei seri rischi nel caso alcune posizioni siano difficili da sostenere oppure evidentemente contraddittorie: proporre strategie sbagliate o difendere strenuamente membri della propria fazione su di cui ricadono accuse troppo gravi è fonte di forte sospetto. Può essere invece utile a volte contraddirsi a vicenda, rischiando magari di vedere uno dei propri alleati sul rogo ma allontanando le accuse da sé stessi.
    
    Infine, acquisendo man mano nuovi componenti quando un Villico morto diventa Spettro, i Villani ottengono nuove informazioni, non solo sui ruoli delle persone ma anche su quali connessioni ci siano fra i Villici. Un uso coordinato degli Spettri può essere molto importante, facendo fuori qualche nemico troppo esposto con lo Spettro della Morte, salvando qualcuno dal rogo con Assoluzione, o semplicemente costruendo coperture solide grazie a Confusione e Invisibilità.
    
\subsection{Strategie ruolo per ruolo}

\begin{itemize}

    \item {\bf Lupo}. I Lupi hanno un compito ovvio: far fuori quanti più Villici possibile. Il grosso delle decisioni si riduce quindi a chi sbranare, e qui ci sono due possibili opzioni: fare fuori i giocatori più attivi o con i ruoli più forti, rischiando però di incappare in qualche Cacciatore, Guardia, o Voyeur; oppure andare a banchettare con qualcuno di meno in vista, per essere sicuri di ridurre il numero di Villici ancora in vita. Tenete infine presente che, se a morire sono solo giocatori attivi, a qualcuno potrebbe venire il legittimo sospetto che i giocatori attivi che non muoiono siano alleati dei Lupi...

	\item {\bf Alcolista}. Privo di qualunque abilità utile, da perfetto scarto della società, l'Alcolista può contribuire alla vittoria dei Villani tramite le assemblee, dove può sfruttare le informazioni della fazione per seminare discordia o fingersi qualcuno di più importante, per distogliere l'attenzione da chi è veramente utile.
	
	Inoltre, non essendo un ruolo centrale per la fazione, avendo aura bianca, e potendo muoversi liberamente, un Alcolista può più facilmente degli altri ruoli uscire allo scoperto, fingendosi un Villico importante grazie alle informazioni fornitegli dai suoi compagni di fazione, infiltrandosi così nelle cerchie di Villici e cercando di sabotarne le strategie. Attenzione però, da ubriaco potrebbe lasciarsi sfuggire qualcosa di troppo!
	
	\item {\bf Assassino}. L'Assassino è il deterrente che hanno i Lupi per impedire controlli coordinati da parte dei Villici: dichiarare che lo Stalker e il Veggente dovrebbero andare a visitare una certa persona aumenta significativamente la probabilità che un Assassino sveglio abbia successo. La sola presenza dell'Assassino, quindi, fa sì che il villaggio debba essere un po' più accorto sulle informazioni da far trapelare.
	
	Sequestratore e Stregone sono grandi alleati dell'Assassino: sapendo chi si muove in quale direzione, raccolgono informazioni molto utili per un'imboscata notturna. 
	
	\item {\bf Diavolo}. Il Diavolo è un ruolo senz'altro importante e delicato, potendo stabilire se il ruolo di un giocatore appartiene a un certo insieme. La sua abilità, in connessione con altri ruoli che raccolgono informazioni, gli permette di identificare i Villici più importanti. È quindi utile che il Diavolo presti attenzione a chi parla durante le assemblee, per riconoscere chi può avere più informazioni del dovuto, per poi confermarne il ruolo la notte seguente. Se necessario, può impiegare le informazioni ricavate tramite i suoi poteri infernali per fingersi Divinatore, Mago, o Veggente, cercando di ottenere la fiducia dei Villici.
	
	\item {\bf Fantasma}. Da vivo, il Fantasma è un Villano senza abilità, e come l'Alcolista può fare di necessità virtù ed utilizzare la propria scarsa centralità per tentare dei bluff rischiosi con dei Villici, cercando di conquistarne la fiducia ma potendo finire sul rogo nel caso venisse scoperto.
	
	Anche in caso venisse arrostito, non tutto il male viene per nuocere! Dopo la morte, il Fantasma può cominciare ad utilizzare una vasta scelta di poteri da utilizzare nel corso della partita. Vista l'importanza della morte per il suo ruolo, il Fantasma si può anche sacrificare per distogliere l'attenzione da personaggi importanti che stanno attirando sospetti.
	
	In seguito alla morte, la strategia cambia. A seconda della situazione, il Fantasma può scegliere il potere più opportuno, per sopperire, se necessario, all'assenza dello Spettro corrispondente, o alla sua forzata inattività. La maggior parte dei poteri diventano utili quando i Villici cominciano a identificare i Villani, per salvare i candidati al rogo, o fermare Villici pericolosi. Ovviamente sta al giocatore stabilire quando è saggio utilizzare un potere, per non sprecarlo inutilmente, o ritrovarsi impossibilitato ad agire il giorno dopo, quando servirebbe.
	
	\item {\bf Fattucchiera}. La Fattucchiera ha una duplice utilità, in quanto può tentare sia di scagionare un Villano che di incastrare un Villico. Se l'apparenza di un personaggio vivo è modificata, normalmente la cosa più importante da scegliere è il colore dell'aura da assegnare, in secondo luogo la misticità (siccome in generale il colore dell'aura è maggiormente identificativo della fazione), e infine il ruolo esatto, per raggirare qualche Divinatore particolarmente zelante. In ogni caso, l'apparenza scelta non conta nulla se nessuno la controlla, quindi agire su di un Villico a caso ha una buona probabilità di non sortire alcun reale effetto; a meno che non ci sia motivo per credere che un Veggente o un Espansivo possano voler andare da qualcuno in particolare, è probabilmente più utile cercare di coprire un Lupo. Attenzione però: se in qualche modo si dovesse scoprire che qualcuno è stato visitato da una Fattucchiera, è naturale che sorga qualche sospetto nei suoi confronti.

    La Fattucchiera ha anche un'altra utilità meno ovvia: siccome può andare dai morti, può influire sull'abilità del Medium e soprattutto del Trasformista. In particolare, con un uso mirato della propria abilità, può far sì che eventuali Trasformisti diventino ruoli poco utili, ad esempio Contadini.

    Infine, è bene notare che la Fattucchiera non può in alcun modo influenzare le trappole poste dal Cacciatore; un Lupo con altre sembianze resta sempre un Lupo, e come tale non può resistere ad una succulenta bistecca usata come esca. Similmente, l'abilità della Guardia non può essere influenzata dall'abilità della Fattucchiera, mentre quella dell'Espansivo può.
	
	\item {\bf Negromante}. Il Negromante è un ruolo che aumenta di molto la flessibilità del pool di Spettri: ottenendo un Grimorio personale ad inizio partita, e potendo cambiare Incantesimi attivi a seconda delle necessità, fa sì che i Villani abbiano quasi sempre lo Spettro giusto per gestire eventuali problemi che possono presentarsi.
	
	Il Negromante deve gestire il proprio Grimorio e quello Principale: tendenzialmente vuole esaurire il proprio Grimorio il prima possibile, ma potrebbe avere senso rimandare se il Grimorio Principale ha degli Incantesimi che sono più utili all'inizio della partita.
	
	Infine, una sola volta a partita, il Negromante può rendere Spettro un qualsiasi Villico morto: ciò è molto forte, sia perché impedisce che un Villico morto possa raccogliere informazioni per la propria fazione, sia perché permette ai Villani di ottenere uno Spettro in più. Questa abilità però ha un costo molto alto: una volta utilizzata, infatti, il Negromante non può più assegnare Incantesimi, e diventa a tutti gli effetti un Contadino, ma nero e mistico.
	
	\item {\bf Sequestratore}. Il Sequestratore è uno dei motivi per cui i ruoli più forti dei Villici non vogliono esporsi troppo: nel caso venire sbranati da un Lupo non fosse un deterrente sufficiente, c'è anche il rischio di finire legati nel cofano di un'auto e svegliarsi la mattina dopo confusi e storditi.
	
	Il Sequestratore scopre inoltre i movimenti delle persone, e pertanto può aiutare i propri compagni di fazione a scoprire le identità degli altri giocatori, e soprattutto l'Assassino ad agire con cognizione di causa.
	
	\item {\bf Stregone}. Lo Stregone è uno dei ruoli più importanti della fazione dei Villani, potendo potenzialmente far fallire controlli incrociati con facilità, e raccogliendo un gran numero di informazioni sui movimenti nel frattempo.
	
	Lo Stregone blocca tutte le abilità e tutti i poteri che vengono utilizzati su una persona, pertanto una possibile strategia è di utilizzare lo Stregone sempre su un Lupo per far fallire tutti i tentativi di controllo su di lui, anche se ciò lo renderebbe alquanto sospetto. In alternativa, lo Stregone può essere utilizzato per fare caos, cercando di far fallire più persone possibile. Infine, potrebbero presentarsi delle situazioni che richiedono che chiunque agisca su una data persona fallisca, per esempio perché è previsto un controllo incrociato.
	
	Chiaramente, per lo Stregone più che per tutti gli altri, è importante coordinarsi con gli altri membri della propria fazione, dato che nessun altro può agire con successo sul bersaglio scelto.
\end{itemize}
    
\section{Ringraziamenti}

Si ringraziano Claudio Afeltra, Lorenzo Bresolin, Mariastella Cascone, Andrea Gallese, Giovanni Interdonato, Edoardo Rizzi, Thomas Spinelli, e molte altre persone coinvolte nella stesura del regolamento per il loro contributo.

\end{document}
