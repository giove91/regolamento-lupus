\documentclass[a4paper,10pt]{article}


\usepackage[utf8]{inputenc}
\usepackage[italian]{babel}
\usepackage{amsmath}
\usepackage{amsthm}
\usepackage{fancyhdr}
\usepackage{amsfonts}
\usepackage{amssymb}
\usepackage{makeidx}
\usepackage[parfill]{parskip}
\usepackage[colorlinks]{hyperref}
\usepackage{fullpage}
\usepackage{mathpazo}
%\usepackage{utopia}


% * define a `\twoidxcolumn` based on `\twocolumn`:
\def\twoidxcolumn{%
%\clearpage
\global\columnwidth\textwidth%
\global\advance\columnwidth-\columnsep%
\global\divide\columnwidth\tw@%
\global\hsize\columnwidth%
\global\linewidth\columnwidth%
\global\@twocolumntrue%
\global\@firstcolumntrue%
\col@number \tw@%
%\@ifnextchar [\@topnewpage
\@floatplacement%
}%


\makeatletter%
\def\@wrindex#1{%
   \protected@write\@indexfile{}%
      {\string\indexentry{#1}{\theenumi}}%
      \endgroup%
      \@esphack}%

\makeatletter
\renewenvironment{theindex}
               {\twocolumn[\section*{Indice delle parole chiave presenti nella
Sezione \ref{faq}}]%
                \@mkboth{\MakeUppercase\indexname}%
                        {\MakeUppercase\indexname}%
                \thispagestyle{plain}\parindent\z@%
                \parskip\z@ \@plus .3\p@\relax%
                \columnseprule \z@%
                \columnsep 35\p@%
                \let\item\@idxitem}
               {}
\makeatother

\makeindex

\newcommand{\smallspace}{\vskip0.3cm}

% Title Page
\title{Lupus in tempo reale\\ Regolamento della sesta edizione}
\author{Alessandro Iraci, Giovanni Mascellani, Giovanni Paolini, Leonardo Tolomeo}

\begin{document}
\maketitle


\section{Introduzione}

\subsection{Cos'è Lupus in tempo reale?}

\emph{Lupus in tempo reale} è una variante dei più tradizionali \emph{Lupus in Tabula} e \emph{Mafia}, caratterizzata da un maggiore coinvolgimento e da interazioni più sofisticate tra i giocatori.
La principale differenza consiste nel ritmo del gioco: la partita si estende infatti su diversi giorni, in modo aderente all'ambientazione.
Sono premiate dalla dinamica di gioco la logica, la capacità di bluffare (e di smascherare i bluff altrui), la capacità di saper coordinare un’azione di gruppo, il carisma e le velleità leaderistiche.

\emph{Lupus in tempo reale} è stato inventato nel 2010, ed è stato giocato una volta all'anno fino al 2012 sotto la supervisione di Francesco Guatieri come principale Game Master. A partire dall'anno 2014 è passato sotto il controllo di Giovanni Mascellani e Giovanni Paolini, che lo gestiscono a tutt'oggi.

Questo regolamento è stato messo per la prima volta per iscritto da Andrea Caleo in occasione della terza edizione del gioco. Da allora in poi è stato di volta in volta aggiornato, con cambiamenti sono stati discussi e stabiliti da Luca Ghidelli, Alessandro Iraci, Giovanni Mascellani, Giovanni Paolini e Leonardo Tolomeo.

Nel seguito di questo regolamento, verrà utilizzata la sigla ``GM'' come abbreviazione di ``Game Master'' (o anche di ``Giovanni Mascellani'', se preferite).

\subsection{Ambientazione}

La storia si svolge in quello che una volta era un tranquillo villaggio di campagna. A turbare la quiete del posto è un evento quantomeno atipico, ossia un'invasione di lupi mannari. Nascosti sotto le spoglie di normalissimi contadini, i licantropi hanno intenzione di prendere il controllo del villaggio e sbranare chiunque non sia disposto ad allearsi con loro.
% ; per portare a termine
% la loro missione, hanno cominciato creando quanto più scompiglio e terrore
% possibile, ovvero divorando lo sfortunato sindaco della piccola cittadina.

Le autorità nazionali, venendo a sapere dell'accaduto e temendo il diffondersi della licantropia in tutto il paese, hanno disposto un blocco forzato delle vie d'accesso al villaggio, rendendo virtualmente impossibile a chiunque entrare o uscire dalla frazione.

Eletto con urgenza un nuovo primo cittadino (del precedente sindaco non vi è più traccia), per tentare di ristabilire la normalità viene imposta una durissima legge marziale: ogni sera, il villaggio convocherà un'assemblea composta da tutti gli abitanti, al termine della quale un sospetto lupo mannaro verrà arso sul rogo. La sfiducia nei confronti degli altri abitanti è totale, chiunque potrebbe essere un licantropo sotto mentite spoglie. La propria vicina di casa, l'insospettabile anziano signore dell'ortofrutta, o anche il nuovo sindaco, così normali di giorno, potrebbero col calar della notte assumere le sembianze di un feroce lupo assetato di sangue.

Come se non bastasse, da qualche giorno corre voce che nel villaggio si nasconda una setta di oscuri negromanti. Persone in grado di adoperare la magia nera, un'abilità tenuta nascosta per paura di essere scoperti. La confusione creatasi nel piccolo borgo, e il numero di morti in probabile aumento, potrebbe fornire loro un'occasione irrinunciabile per cominciare a praticare la propria arte, richiamando alla vita le vittime sotto forma di spettri, e utilizzando i poteri soprannaturali di questi ultimi per rendere il villaggio schiavo della setta.

Fra i cittadini c'è paura, e tanta indecisione su quale sia la cosa giusta da fare. Meglio combattere lupi mannari e negromanti e cercare di liberare il villaggio, oppure allearsi con una di queste due fazioni nella speranza di avere salva la vita, e magari anche di ottenere una posizione di potere? Meglio rischiare la pelle andando ad origliare dietro la porta del proprio vicino, cercando di capire se questi è un lupo, oppure stare tranquilli nel proprio letto sperando di non ricevere visite sgradite? Molti abitanti hanno deciso di attivarsi per riprendere il controllo del villaggio; altri hanno ceduto, per paura della morte o per sete di potere, alle lusinghe dei licantropi o dei negromanti, aiutandoli a portare a termine i loro progetti.

Ciascuno ha tutto l'interesse a tenere nascosta la propria scelta: per i traditori, venire scoperti potrebbe significare morte sul rogo; per i villici che hanno deciso di combattere l'invasione, esporsi troppo potrebbe voler dire essere la prossima vittima dei lupi mannari, oppure ricevere terrificanti visite da parte di uno spettro. Ma non sempre tacere per la propria incolumità è la scelta giusta. In determinate circostanze rivelare preziose informazioni, anche a costo di rischiare la vita, potrebbe essere fondamentale; veder trionfare un'altra fazione, in ogni caso significa morte.

Qualunque sia stata la propria decisione, tuttavia, la maggior parte dei cittadini si attiva durante la notte per raccogliere informazioni o infastidire i nemici: per esempio appostandosi fuori da una casa per cercare di capire cosa succede all'interno, utilizzando le proprie capacità divinatorie per scoprire l'identità degli altri, girando armati per proteggere i concittadini dall'attacco dei licantropi, o compiendo rituali per tenere lontani gli spettri. Ciascuno fa il possibile, ma anche gli alleati di lupi mannari e negromanti si mobilitano per intralciare le indagini, diffondere false informazioni e manipolare le votazioni durante l'assemblea.

Nel villaggio, ormai, la vita è una questione di doppio gioco; ogni passo falso rischia di costare la pelle non solamente a sé stessi, ma anche a tutti i propri alleati. Un delicatissimo equilibrio coinvolge le tre parti, cercare di spezzarlo in proprio favore è l'obiettivo di tutti. Solo i più abili manipolatori riusciranno a prendere in mano la situazione e condurre la propria squadra alla vittoria. Lo scopo da raggiungere è chiaro, ma quale sia la fazione che riuscirà a trionfare dipende da voi...


\subsection{Partecipanti}
Possono partecipare:
\begin{itemize}
 \item gli studenti della Scuola;
 \item altre persone che frequentano spesso gli ambienti della Scuola, in particolare il collegio Carducci e la mensa, dal lunedì al venerdì.
\end{itemize}
Tutti i partecipanti si impegnano a farsi vedere un po' in giro negli ambienti della Scuola nei giorni della partita. Nessuno vi chiede di non partire venerdì mattina per tornare a casa se non avete lezione, ma se vivete da reclusi in camera vostra e non mangiate mai in mensa potrebbe essere il caso di non giocare.


\subsection{Quando si giocherà}

La partita inizierà la sera di martedì 18 novembre 2014, con ritrovo al collegio Carducci. % TODO


% \subsection{Principali cambiamenti nella quarta edizione}
% 
% Questa sezione è pensata per dare a chi ha giocato alla terza edizione di Lupus
% un'idea delle modifiche più importanti che sono state apportate. Tuttavia non
% sostituisce un'attenta lettura del resto del regolamento.
% 
% \begin{itemize}
%  \item La Fazione dei Vampiri è stata sostituita dalla nuova Fazione dei
% Negromanti, che ha meccaniche di gioco molto diverse.
%  \item È stato eliminato il suicidio.
%  \item È stato eliminato il rituale dei fantasmi.
%  \item I possibili colori dell'aura sono tornati ad essere solamente il bianco e
% il nero.
%  \item Molti ruoli sono stati creati, eliminati o modificati. A chi è abituato
% al regolamento della terza edizione raccomandiamo di controllare con particolare
% attenzione i seguenti ruoli, che hanno subito i cambiamenti più rilevanti:
% Cacciatore, Custode del cimitero, Esorcista, Investigatore, Rinnegato, Profanatore di tombe, Negromante, Fantasma, Ipnotista, Medium, Spettro.
%  \item Le comunicazioni di gioco (votazione, utilizzo dei poteri, resoconto
% dell'alba e del tramonto) avvengono attraverso un'interfaccia web e non più via
% e-mail.
% \end{itemize}
% 
% \subsection{Principali cambiamenti nella quinta edizione}
% 
% Le seguenti sono le modifiche apportate rispetto alla quarta edizione di Lupus in tempo reale.
% 
% \begin{itemize}
% \item È stato introdotto lo Scrutatore.
% \item È stato eliminato lo Spettro con il potere della Duplicazione.
% \item L'ultimo Ipnotista, quando muore, diventa uno Spettro con il potere dell'Ipnosi.
% \item Il Profanatore di Tombe è stato trasferito alla fazione dei
%   Popolani ed è stato rinominato in Sciamano. 
%   Rimane con aura nera, ma diventa mistico.
% \item Il Necrofilo è stato rinominato in Trasformista.
% \item Il Medium scopre il ruolo del personaggio scelto, e non più l'aura.
% \end{itemize}



\subsection{Principali cambiamenti nella sesta edizione}

Questa sezione è pensata per dare a chi ha giocato alla quinta edizione di Lupus un'idea delle modifiche più importanti che sono state apportate. Tuttavia non sostituisce un'attenta lettura del resto del regolamento.

\begin{itemize}
  \item Il Trasformista e il Fantasma hanno ora aura nera.
  \item Nessun personaggio può utilizzare il proprio potere su sé stesso.
  \item Sono state rimosse le restrizioni sull'utilizzo di un potere due notti consecutive sullo stesso personaggio.
\end{itemize}




\pagebreak
\section{Metagioco}

\subsection{Giorni e notti di gioco}

I giorni di gioco si estendono dalle 8:00 (circa) alle 22:00 di lunedì, martedì, mercoledì e giovedì, e dalle 8:00 (circa) di venerdì alle 22:00 di domenica. Le notti di gioco si estendono dalle 22:00 (circa) di domenica, lunedì, martedì, mercoledì e giovedì, alle 8:00 del giorno successivo.


\subsection{Comunicazioni tra GM e giocatori, votazioni, poteri}

All'inizio della partita, ciascun giocatore riceve le credenziali per accedere all'interfaccia web di Lupus, il cui indirizzo è \verb|http://uz.sns.it/lupus/|.
Le votazioni durante il giorno e le attivazioni dei poteri avvengono tutte tramite l'interfaccia web. Le informazioni sulle votazioni del giorno e sugli avvenimenti della notte compaiono a loro volta sull'interfaccia web (le informazioni pubbliche possono essere visualizzate senza bisogno di autenticazione, mentre quelle private sono accessibili solo dopo il login). In caso di impossibilità di accedere a internet, i giocatori possono eccezionalmente contattare i GM (ad esempio via SMS) per comunicare loro le proprie intenzioni di voto e/o il modo in cui desiderano utilizzare il proprio potere speciale.

Le comunicazioni tra GM e giocatori sono inviolabili. È vietato origliare le conversazioni dei GM con i giocatori, fare pressione sui GM in qualsiasi modo, cercare truffaldinamente di ottenere informazioni sulla partita in corso ed altre azioni di questo genere.

È inoltre vietato spiare altri giocatori mentre accedono all'area riservata dell'interfaccia web, rubare o hackare account altrui (anche approfittando di eventuali distrazioni), mostrare o dare accesso al proprio account ad altri giocatori, o cercare di violare la sicurezza dell'interfaccia web. Le medesime regole si applicano alla lettera personale che viene data a ciascun giocatore all'inizio della partita (contenente il ruolo assegnatogli). La violazione di queste regole può portare alla squalifica del giocatore o, nei casi più gravi, dell'intera fazione.

Ci possono essere altri comportamenti che non ricadono in quelli descritti in precedenza ma che evidentemente sono contrari allo spirito del gioco. Anche in questi casi i GM hanno il diritto di squalificare un giocatore o un'intera fazione, e il loro giudizio è inappellabile.


\subsection{Comunicazioni tra i giocatori}

I giocatori possono comunicare tra di loro con qualsiasi mezzo. Possono dirsi qualsiasi cosa.

È vietato mostrare ad altri giocatori le proprie informazioni riservate sull'interfaccia web, o qualsiasi tipo di informazione ricevuta direttamente dai GM, a prescindere dal mezzo con cui è stata comunicata: tramite interfaccia web, e-mail, SMS, lettera cartacea o in qualsiasi altro modo.

All'inizio della partita, verrà pubblicata una lista degli indirizzi e-mail dei partecipanti. Ai giocatori è vietato usare un indirizzo e-mail di quella lista diverso dal proprio: in particolare, è vietato mandare e-mail da quell'account se ne si ha accesso in qualche modo; è vietato usare qualsiasi tipo di abilità informatica per avere accesso a quella casella e-mail; è vietato inviare e-mail usando quell'indirizzo pur non avendone accesso.

La violazione di queste regole può portare alla squalifica del giocatore o, nei casi più gravi, dell'intera fazione. La violazione delle leggi dello Stato in cui vi trovate è sconsigliata dai GM, ma non è proibita a priori dal regolamento. Questo non vi darà comunque alcun tipo di immunità, civile o penale.

È lecito (anzi, spesso consigliato) imbrogliare gli altri giocatori, mentire, usare indirizzi e-mail fasulli o anonimi per comunicare con gli altri giocatori o per cercare di ingannarli, e qualsiasi altro mezzo vi venga in mente per vincere la partita, purché sia nel rispetto delle regole di cui sopra.

È anche lecito spiare il comportamento di altri giocatori, sia di persona che con mezzi informatici, con la seguente eccezione: è vietato spiare un giocatore all'interno della sua camera o abitazione personale. In particolare è vietato eseguire monitoraggio delle reti dei collegi, allo scopo per esempio di determinare in che istanti i giocatori sono connessi ad Internet. Le reti delle aule computer dei collegi invece non godono di questo tipo di immunità e possono essere
oggetto di spia (ovviamente fatte salve tutte le regole riportate sopra, come l'inviolabilità delle comunicazioni tra un giocatore ed il server). Potrebbero essere fornite dai GM postazioni per accedere all'interfaccia web di gioco in maniera sicura.

Anche i morti possono parlare con gli altri membri del villaggio, e continuano a giocare. Si incoraggiano i giocatori a non comunicare solo via e-mail ma anche di persona.

È ovvio che tutte le regole sopra indicate si possono prestare a diverse interpretazioni. I giocatori sono invitati a chiedere ai GM conferma della liceità di comportamenti che potrebbero essere valutati \emph{border line}.


\subsection{Ulteriori restrizioni}

È vietato utilizzare i privilegi di amministrazione di UZ per ottenere informazioni altrimenti inaccessibili (ad esempio leggendo la posta di altri giocatori o monitorando gli accessi alle aule computer in modi altrimenti impossibili). Altre indagini di natura informatica sono consentite (ed incoraggiate).

Se un amministratore di UZ è costretto, nell'esercizio dei suoi compiti, a violare questa regola (per esempio per attività di manutenzione del sistema informatico), egli deve riportare immediatamente questo fatto ai GM ed astenersi dall'usare queste informazioni ai sensi del gioco. I GM valuteranno se è possibile che egli continui il gioco senza pregiudizio per gli altri giocatori e prenderanno decisione di conseguenza.


\pagebreak
\section{Gioco}

\subsection{Fazioni e condizioni di vittoria}

I giocatori sono divisi in tre fazioni: la Fazione dei Popolani, la Fazione dei Lupi e la Fazione dei Negromanti. Una fazione vince se, subito dopo l'alba oppure subito dopo la votazione del tramonto, tutti i personaggi vivi appartengono a quella fazione. Inoltre, la Fazione dei Lupi perde immediatamente se muoiono tutti i Lupi, e la Fazione dei Negromanti perde immediatamente se muoiono tutti i Negromanti. Quando una di queste due eventualità accade, viene reso pubblico l'elenco dei membri della fazione che ha appena perso (vivi e morti), e questi vengono esiliati dal villaggio: da quel momento in poi, smettono di giocare a tutti gli effetti.

Se una fazione ha chiaramente vinto prima che le condizioni di vittoria precedenti siano rispettate, i GM possono porre fine alla partita in anticipo (ma non devono farlo per forza).

\subsection{Inizio del gioco}

All'inizio della partita ciascun giocatore riceve le informazioni necessarie per partecipare, tra cui il ruolo ed eventualmente le identità di alcuni altri giocatori (per esempio, ai Lupi viene comunicato chi sono gli altri Lupi e le Fattucchiere). La partita comincia con la notte.
% Ai Lupi non è concesso
% di uccidere, in modo che a tutti siano garantiti almeno una notte ed un giorno di
% gioco. Agli altri personaggi è invece permesso usare i propri poteri, purché vi siano
% bersagli validi (per esempio, non essendoci alcun morto, un eventuale Medium non
% potrà agire).

\subsection{Svolgimento del giorno e votazioni}

L'inizio del giorno è annunciato sul sito web, insieme a tutte le informazioni ottenute da ciascun giocatore in seguito agli avvenimenti della notte appena trascorsa. Ogni giorno, il Villaggio convoca un'assemblea per stabilire chi condannare a morte sul rogo. I personaggi vivi hanno il diritto di votare entro le ore 22:00 un abitante del villaggio da condannare a morte. Dopo le ore 22:00, viene pubblicata la lista dei votanti insieme alla preferenza espressa da ciascuna persona.
Il voto è ritenuto valido se almeno il 50\% dei vivi ha votato; in caso contrario, nessuno viene condannato. Il personaggio che ha ricevuto più voti di tutti muore.
In caso di parità tra due o più personaggi, fra questi viene condannato quello eventualmente votato dal Sindaco; se nessuno di questi è stato votato dal Sindaco, ne muore uno a caso.

Di giorno è anche possibile votare per eleggere un nuovo Sindaco, come è spiegato più dettagliatamente nella Sezione \ref{sindaco}.


\subsection{Svolgimento della notte e uso dei poteri}

La notte inizia nel momento in cui viene pubblicato l'esito della votazione del giorno, e termina alle 8:00 della mattina seguente. Durante la notte, i personaggi vivi che hanno poteri speciali hanno la facoltà di attivarli. Inoltre, gli Spettri hanno la facoltà di attivare i propri poteri soprannaturali.

Alcuni personaggi hanno un potere attivabile ogni due notti: si intende che tale potere può essere usato se e solo se non è stato usato durante la notte precedente: in altre parole, l'unica restrizione è data dal non poterlo usare in due notti consecutive.

Nessun personaggio può utilizzare il proprio potere speciale su sé stesso.


\subsection{Morte}

Molti dei personaggi non resteranno vivi sino alla fine della partita. Un personaggio può morire in diversi modi: condannato al rogo, sbranato dai Lupi, oppure ucciso da uno Spettro o da un Assassino.

I personaggi morti possono comunque continuare a giocare e aiutare la propria fazione. Se una fazione vince, tutti i componenti di quella fazione vincono, indipendentemente dal fatto che siano vivi o morti.

Un personaggio morto non può votare, né usare il proprio potere speciale, se ne ha uno. Può interagire con gli altri personaggi in tutti i modi non vietati dal regolamento, per esempio rivelando informazioni in suo possesso. Può inoltre essere riportato in vita dal Messia o risvegliato come Spettro, se si verificano le opportune condizioni.

Se un personaggio morto viene risvegliato come Spettro, resta morto, ma può usare il suo nuovo potere soprannaturale. Il funzionamento degli Spettri è spiegato nella Sezione \ref{negromanti}.


\subsection{Composizione del villaggio}
 
La composizione del villaggio è scelta dai GM e non è nota ai partecipanti. I ruoli sono assegnati casualmente ai giocatori.

I GM sono liberi, all'inizio della partita, di dare alcune indicazioni sulla composizione (ad esempio, potrebbero dire: ``Su $40$ giocatori presenti, i Lupi sono più di $4$, i componenti della Fazione dei Lupi non sono più di $11$ e ci sono almeno un Veggente ed un Investigatore'').


\subsection{Ruoli}
\label{ruoli}

% Non è necessario che tutti i giocatori sappiano cosa fa ogni ruolo; se siete
% particolarmente pigri, è sufficiente che sappiate quale potere speciale avete
% voi (vi verrà ricordato nella lettera in cui vi si assegnerà il ruolo).
% Tuttavia, per giocare in modo più efficace, è certamente conveniente sapere
% anche cosa possono fare gli altri.

I seguenti sono i ruoli che \emph{possono} comparire nel gioco, divisi per fazione. Per ciascuno di essi è indicata tra parentesi l'aura, che può essere bianca oppure nera, ed è eventualmente indicato l'attributo ``mistico''.


\subsection*{Fazione dei Popolani}

\begin{itemize}
 \item {\bf Contadino} (aura bianca). Il Contadino non ha alcun potere speciale.

%  \item {\bf Cacciatore} (aura nera). Una sola volta durante l'arco della
% partita, in una notte dopo la prima, il Cacciatore può scegliere un personaggio vivo e puntarlo col fucile.
%  Il Cacciatore uccide il personaggio scelto.
 
%  \item {\bf Custode del cimitero} (aura bianca). Ogni notte, il Custode del
% cimitero può scegliere un personaggio morto e custodirne la tomba. Per quella
% notte, se uno o più Negromanti tentano di risvegliare come Spettro il
% personaggio scelto dal Custode del cimitero, il loro potere non ha effetto.
%  
% %  Questo potere speciale non può essere usato per due notti consecutive sullo stesso personaggio.

 \item {\bf Divinatore} (aura bianca, mistico). Il Divinatore non ha alcun potere speciale. All'inizio della partita, il Divinatore è a conoscenza di quattro proposizioni nella forma ``Il personaggio X ha il ruolo Y''. Esattamente due sono vere ed esattamente due sono false.
 
 Il modo in cui queste quattro frasi sono state generate è a discrezione dei GM, e non è noto ai giocatori.

 \item {\bf Esorcista} (aura bianca, mistico). Ogni due notti, l'Esorcista può scegliere un personaggio, vivo o morto, e benedire la sua casa. Per quella notte, se uno Spettro tenta di utilizzare il proprio potere sul personaggio scelto dall'Esorcista, il suo potere non ha effetto.
 
%  L'Esorcista può usare il suo potere speciale su sé stesso.
 
 \item {\bf Espansivo} (aura bianca). Ogni due notti, l'Espansivo può scegliere un personaggio vivo, e andare a trovarlo. Questo personaggio scopre l'identità dell'Espansivo.

 \item {\bf Guardia del corpo} (aura bianca). Ogni notte, la Guardia del corpo può scegliere un personaggio vivo e proteggerlo. Per quella notte, se uno o più Lupi tentano di uccidere il personaggio scelto dalla Guardia, il loro potere non ha effetto.
 
 \item {\bf Investigatore} (aura bianca). Ogni notte, l'Investigatore può scegliere un personaggio morto e indagare su di esso. Scopre il colore della sua aura.

 \item {\bf Mago} (aura bianca, mistico). Ogni notte, il Mago può scegliere un personaggio, vivo o morto, e percepirne la magia. Scopre se quel personaggio è un mistico oppure no.
 
 \item {\bf Massone} (aura bianca). Il Massone non ha alcun potere speciale. I Massoni conoscono gli altri Massoni.
 
 \item {\bf Messia} (aura bianca, mistico). Una sola volta durante l'arco della partita, il Messia può scegliere un personaggio morto e resuscitarlo. Quel personaggio ritornerà in vita all'inizio del giorno seguente, riacquistando i suoi poteri speciali.
 
 Se il personaggio scelto è uno Spettro, il potere del Messia non ha effetto.

 \item{\bf Sciamano} (aura nera, mistico). Ogni due notti, lo Sciamano può scegliere un personaggio morto e tormentarne lo spirito. Se il personaggio scelto è uno Spettro, e questi agisce, il suo potere non ha effetto.

 \item {\bf Stalker} (aura bianca). Ogni due notti, lo Stalker può scegliere un personaggio vivo e pedinarlo. Scopre se quel personaggio ha agito su altri personaggi durante la notte, ed in tal caso su chi. Non scopre, però, cosa ha fatto.
 
%  Se il personaggio pedinato dallo Stalker utilizza il proprio potere su sé stesso, lo Stalker riceve informazioni come se tale personaggio non avesse agito.

 \item {\bf Trasformista} (aura nera). Una sola volta durante l'arco della partita, il Trasformista può scegliere un personaggio morto e rubarne il potere. Se il personaggio scelto ha un potere speciale attivabile una volta ogni una o due notti, il Trasformista lo scopre ed ottiene tale potere; altrimenti, il potere del Trasformista non ha effetto.
 
 Se il personaggio scelto è un Lupo o un Negromante, il potere del Trasformista non ha effetto. Se il personaggio scelto è uno Spettro, il Trasformista ottiene il potere che quel personaggio aveva prima di diventare Spettro.
 
 Nel momento in cui il Trasformista ottiene il potere speciale di un altro personaggio, acquisisce anche il ruolo, l'aura e la proprietà di essere mistico del personaggio in questione, ma continua a giocare con la Fazione dei Popolani.

 \item {\bf Veggente} (aura bianca, mistico). Ogni notte, il Veggente può scegliere un personaggio vivo e scrutarlo nella sua sfera di cristallo. Scopre il colore della sua aura.

 \item {\bf Voyeur} (aura bianca). Ogni due notti, il Voyeur può scegliere un personaggio vivo e spiarlo. Scopre quali sono gli altri personaggi vivi che durante la notte hanno agito sul personaggio scelto.
 
%  Se il personaggio spiato dal Voyeur utilizza il proprio potere su sé stesso, il Voyeur non riceve tale informazione.

\end{itemize}


\subsection*{Fazione dei Lupi}

\begin{itemize}
 \item {\bf Lupi} (aura nera). Ogni notte dopo la prima, ciascun Lupo può scegliere un personaggio vivo e tentare di ucciderlo. Se tutti i Lupi che decidono di usare il proprio potere speciale scelgono lo stesso personaggio, questi muore. Se almeno due Lupi indicano personaggi diversi, il loro potere speciale non ha effetto.
 
 I Lupi non possono uccidere i Negromanti. Se i Lupi tentano di uccidere un Negromante, il loro potere non ha effetto.
 
 I Lupi conoscono gli altri Lupi.
 
 \item {\bf Assassino} (aura nera). Ogni due notti, l'Assassino può scegliere un personaggio vivo e puntare il suo fucile di precisione verso casa sua. Uccide uno degli altri personaggi che hanno usato il proprio potere sul personaggio scelto dall'Assassino, scelto in modo casuale.
 
 L'Assasino conosce i Sequestratori, i Rinnegati e gli eventuali altri Assassini.

 \item {\bf Avvocato del diavolo} (aura nera). Ogni due notti, l'Avvocato del diavolo può scegliere un personaggio vivo e scrivere per lui una missiva di rilascio.
 Durante il giorno successivo, se l'assemblea decide, tramite la votazione, di uccidere il personaggio scelto dall'Avvocato del diavolo, questi non muore.
 
 L'Avvocato del Diavolo conosce i Diavoli, le Fattucchiere e gli eventuali altri Avvocati del Diavolo.

 \item {\bf Diavolo} (aura nera, mistico). Ogni notte, il Diavolo può scegliere un personaggio vivo e leggere nella sua anima. Scopre il ruolo del personaggio scelto.
 
 Il Diavolo conosce gli Avvocati del Diavolo, le Fattucchiere e gli eventuali altri Diavoli.
 
 \item {\bf Fattucchiera} (aura nera, mistico). Ogni notte la Fattucchiera può scegliere un personaggio, vivo o morto, e stregarlo. Per quella notte, il colore dell'aura del personaggio scelto risulta diverso da quello effettivo.
 
%  La Fattucchiera può usare il suo potere speciale su sé stessa.
 
 La Fattucchiera conosce gli Avvocati del Diavolo, i Diavoli e le eventuali altre Fattucchiere.
  
 \item {\bf Rinnegato} (aura bianca). Il Rinnegato non ha alcun potere speciale. Il Rinnegato conosce gli Assassini, i Sequestratori e gli eventuali altri Rinnegati.

 \item {\bf Sequestratore} (aura nera). Ogni notte, il Sequestratore può scegliere un personaggio vivo e rapirlo. Per quella notte, se il personaggio scelto agisce, il suo potere non ha effetto. Stalker e Voyeur ricevono informazioni come se quel personaggio non avesse agito.
 
%  Questo potere speciale non può essere usato per due notti consecutive sullo stesso personaggio.
 
 Il Sequestratore conosce gli Assassini, i Rinnegati e gli eventuali altri Sequestratori.


\end{itemize}

\subsection*{Fazione dei Negromanti}
\label{negromanti}
\begin{itemize}

 \item {\bf Negromanti} (aura bianca, mistici). Se la notte precedente nessun personaggio è stato risvegliato come Spettro, ciascun Negromante può scegliere un personaggio morto e selezionare un potere soprannaturale (per la lista dei possibili poteri, vedere oltre). Se tutti i Negromanti che decidono di usare il proprio potere speciale scelgono lo stesso personaggio e selezionano lo stesso potere soprannaturale, il personaggio scelto diventa uno Spettro e ottiene il potere selezionato. Da quel momento in poi, egli appartiene alla Fazione dei Negromanti. Gli viene inoltre comunicata l'identità dei Negromanti che lo hanno risvegliato come Spettro.
 Se almeno due Negromanti scelgono personaggi diversi o selezionano poteri soprannaturali diversi, il loro potere speciale non ha effetto e nessun personaggio viene risvegliato come Spettro.
 
 Una volta che un potere soprannaturale viene assegnato ad uno Spettro (compreso il Fantasma), questo non può più essere scelto per i nuovi Spettri. Appena viene creato uno Spettro con il potere della Morte, tutti i Negromanti perdono il potere di creare Spettri.
 
 I Lupi, i Diavoli, gli Assassini e tutti i personaggi che appartengono già alla Fazione dei Negromanti non possono essere risvegliati come Spettri.

 I Negromanti non possono essere uccisi dai Lupi. Se i Lupi tentano di uccidere un Negromante, il loro potere non ha effetto.

 I Negromanti conoscono gli altri Negromanti.
 
 \item {\bf Fantasma} (aura nera). Il Fantasma non ha alcun potere speciale. Se il Fantasma muore, diventa immediatamente uno Spettro ed ottiene uno dei poteri soprannaturali non ancora assegnati, scelto in modo casuale fra quelli possibili. Gli viene inoltre comunicata l'identità dei Negromanti, e ai Negromanti viene comunicata l'identità del Fantasma.
 
 Al Fantasma non possono venire assegnati i poteri soprannaturali dell'Ipnosi e della Morte.
 
 \item {\bf Ipnotista} (aura bianca). Ogni due notti, l'Ipnotista può scegliere un personaggio vivo e controllarne la mente. Da quel momento in poi il voto per il rogo di quel personaggio è considerato uguale a quello espresso dall'Ipnotista; questo è vero anche se l'Ipnotista non vota.
 Il personaggio scelto non viene informato di essere sotto il controllo dell'Ipnotista.
 
 L'Ipnotista è immune al potere degli altri Ipnotisti. Se un Ipnotista tenta di controllare la mente di un altro Ipnotista, il suo potere non ha effetto.
 L'Ipnotista è immune ai potere dell'Amnesia e dell'Ipnosi. Se uno di tali Spettri tenta di modificare il voto dell'Ipnotista, il suo potere non ha effetto.

 Se un Ipnotista è morto, i personaggi sotto il suo controllo votano secondo il proprio volere.
 Un personaggio può essere sotto il controllo di un solo Ipnotista per volta, e precisamente l'ultimo ad aver agito su di esso.

 Nel momento in cui un Ipnotista %vivo (e appartenente alla Fazione dei Negromanti) 
 muore, costui diventa automaticamente uno Spettro con il potere dell'Ipnosi. Questo non avviene se il potere dell'Ipnosi è già stato assegnato.

 L'Ipnotista conosce i Medium, gli Scrutatori e gli eventuali altri Ipnotisti.

 \item {\bf Medium} (aura bianca, mistico). Ogni notte, il Medium può scegliere un personaggio morto e contattarne lo spirito. Scopre il ruolo del personaggio.

 Il Medium conosce gli Ipnotisti, gli Scrutatori e gli eventuali altri Medium.
 
 \item {\bf Scrutatore} (aura bianca). Ogni notte, lo Scrutatore può scegliere  un personaggio vivo e aggiugere segretamente un suo voto nell'urna della votazione per il rogo  del giorno seguente. Seleziona anche un personaggio vivo a cui questo voto sarà diretto. Il giorno seguente, al momento della pubblicazione dei voti, risulta che il personaggio scelto ha votato \emph{anche} per il personaggio selezionato.
 
 Lo Scrutatore conosce gli Ipnotisti, i Medium e gli eventuali altri Scrutatori.

 \item {\bf Spettri}. Gli Spettri non sono presenti all'inizio della partita: i personaggi morti possono essere risvegliati come Spettri in seguito all'azione di un Negromante.
 Nel momento in cui un personaggio diviene uno Spettro, mantiene la sua aura ed eventualmente la proprietà di essere mistico, ma inizia a giocare per la Fazione dei Negromanti. Gli viene comunicata l'identità dei Negromanti che lo hanno risvegliato, e ottiene un potere soprannaturale che può iniziare a usare dalla notte successiva. Inoltre, perde per sempre gli eventuali poteri speciali che possedeva.
 
 Uno Spettro rimane morto: in particolare non può essere ucciso, e il villaggio non viene informato del suo risveglio.
 La presenza di uno Spettro che sta usando il suo potere passa inosservata agli occhi del Voyeur.
 
%  Se uno Spettro viene resuscitato dal Messia, continua a giocare per la Fazione ei Negromanti, ma perde per sempre il suo potere soprannaturale. 
%  Non riottiene i poteri speciali eventualmente posseduti prima di essere stato risvegliato come Spettro.
 
\end{itemize}


\paragraph{Poteri soprannaturali degli Spettri}

\begin{itemize}
 \item {\bf Amnesia}. Ogni notte, lo Spettro può scegliere un personaggio vivo e ottenebrarne i ricordi. Il giorno successivo, il suo eventuale voto per il rogo viene ignorato: a tutti gli effetti è come se non avesse votato. Ciò accade anche se il personaggio scelto è sotto il controllo dell'Ipnotista o dello Spettro con il potere dell'Ipnosi: invece di votare come l'Ipnotista, o come stabilito dallo Spettro con il potere dell'Ipnosi, non vota affatto.
 
 L'Ipnotista è immune al potere dell'Amnesia. Se lo Spettro tenta di ottenebrare i ricordi dell'Ipnotista, il suo potere non ha effetto.
 
%  Questo potere non può essere usato per due notti consecutive sullo stesso personaggio.
 
 \item {\bf Confusione}. Ogni notte, lo Spettro può scegliere un personaggio, vivo o morto, e generare confusione attorno a lui.
 Per quella notte, tutti personaggi che cercano di ottenere informazioni (riguardanti aura, misticità, ruolo o fazione) sul personaggio scelto, ricevono un responso casuale.
 
 \item{\bf Ipnosi} Ogni notte, lo spettro può scegliere un personaggio vivo e controllarne la mente. Seleziona un secondo personaggio vivo, e il giorno seguente il voto per il rogo del personaggio scelto sarà espresso contro il secondo personaggio.

 L'Ipnotista è immune al potere dell'Ipnosi. Se lo Spettro tenta di controllare la mente dell'Ipnotista, il suo potere non ha effetto.

%  \item {\bf Duplicazione}. Ogni notte, lo Spettro può scegliere un personaggio
% vivo e assumerne l'aspetto. Il giorno successivo, il suo voto per il rogo conta
% doppio. Non risulta, però, che il personaggio abbia votato due volte;
% semplicemente, il personaggio per cui vota riceve un voto in più.

 \item {\bf Illusione}. Ogni due notti, lo Spettro può scegliere un personaggio, vivo o morto, quindi generare un'illusione di un personaggio vivo. L'illusione compare nella casa del personaggio scelto.
 Agli occhi di uno Stalker, è come se il personaggio di cui è stata generata l'illusione avesse agito \emph{solo} sul personaggio scelto. Nel caso in cui il personaggio scelto coincida con quello di cui è stata generata l'illusione, uno Stalker che usa il suo potere su tale personaggio riceve informazioni come se questi non avesse agito.
 Agli occhi di un Voyeur, è come se il personaggio di cui è stata generata l'illusione avesse agito \emph{anche} sul personaggio scelto. Nel caso in cui il personaggio scelto coincida con quello di cui è stata generata l'illusione, un Voyeur che usa il suo potere su tale personaggio non lo vede.

%  \item {\bf Mistificazione}. Ogni notte, lo Spettro può scegliere un personaggio, vivo o morto, e creare un'aura magica nella sua casa. Per quella notte, il personaggio scelto è considerato mistico. 
 
 \item {\bf Morte}. Ogni due notti, lo Spettro può scegliere un personaggio vivo e ucciderlo.
 Lo Spettro non può uccidere i Lupi. Se lo Spettro tenta di uccidere un Lupo, il suo potere non ha effetto.
 
 \item {\bf Occultamento}. Ogni notte, lo Spettro può scegliere un personaggio vivo o morto, e creare attorno alla sua casa una fittissima nebbia magica. Per quella notte, se un qualsiasi altro personaggio, eccetto l'Esorcista, agisce sul personaggio scelto dallo Spettro, il suo potere non ha effetto.
 
 \item {\bf Visione}. Ogni notte, lo Spettro può scegliere un personaggio vivo e scrutare nei suoi pensieri. Ne scopre la fazione di appartenenza.
 
\end{itemize}


\subsection{Utilizzo dei poteri e fallimento}
\label{fallimento}

\paragraph{Utilizzo dei poteri} 
Se un personaggio con un potere attivabile ogni due notti (oppure una volta a partita) agisce, a prescindere dal fatto che agisca con successo, non potrà farlo nella notte successiva (o nel resto della partita).

I personaggi che ottengono informazioni di qualsiasi tipo su altri personaggi, le ottengono riguardo alla condizione in cui questi si trovavano al termine del giorno precedente, anche se tali informazioni cambiano nel corso della notte.
Ad esempio, se un morto viene contemporaneamente risvegliato come Spettro e visitato dal Medium, quest'ultimo scopre il ruolo che il personaggio aveva prima di diventare uno Spettro (perché al termine del giorno precedente non lo era ancora diventato).

% Se non è diversamente specificato, i personaggi non possono utilizzare i propri poteri su loro stessi.

\paragraph{Fallimento} Ci sono alcuni modi di utilizzare il proprio potere che sono a priori non consentiti, e pertanto non risultano nemmeno tra le scelte disponibili sull'interfaccia web. Ad esempio, l'Investigatore non può scegliere di agire su un personaggio vivo; un Negromante non può provare ad attivare il proprio potere se durante la notte precedente è stato creato uno Spettro (anche se non è stato lui a crearlo).

Anche escludendo i casi menzionati sopra, non sempre un personaggio riesce ad usare il proprio potere con successo. Ci sono essenzialmente tre possibili ragioni per cui il potere può non avere effetto.
\begin{itemize}
 \item Blocco da parte di un altro personaggio. Ad esempio, si può essere rapiti da un Sequestratore, bloccati da un Esorcista o dallo Spettro con il potere dell'Occultamento.
 \item Restrizioni all'utilizzo del potere. Ad esempio, un Lupo non riesce a uccidere se nella stessa notte un altro Lupo cerca di assassinare un personaggio diverso; il potere del Trasformista non ha alcun effetto su un Lupo.
 \item Contraddizioni nell'utilizzo dei poteri. Può succedere che non ci sia alcun modo logico di risolvere le azioni dei personaggi nel corso di una notte (vedi Sezione \ref{faq} per una discussione più approfondita). Questo accade per esempio se tre Sequestratori si sequestrano in modo ciclico.
\end{itemize}
Nel caso in cui un personaggio cerchi di utilizzare il proprio potere e fallisca, riceve una notifica di fallimento. Tale notifica non precisa la ragione del fallimento.

Un personaggio che cerchi di utilizzare il proprio potere e fallisca, viene comunque visto da eventuali Stalker e Voyeur a meno che il fallimento sia dovuto a un Sequestratore (come riportato nella descrizione del Sequestratore, Sezione \ref{ruoli}).

\subsection{Lo status di Sindaco}
\label{sindaco}

Se durante l'assemblea si scopre che non vi è un personaggio che ha ricevuto strettamente più voti di ogni altro, tra i personaggi che hanno ricevuto il maggior numero di voti, viene condannato quello eventualmente votato dal
Sindaco.
Il Sindaco è scelto casualmente all'inizio della partita dai GM (in modo scorrelato dalla fazione di appartenenza).

In qualsiasi momento il Sindaco può designare un successore: se il Sindaco muore, il successore designato diventa il nuovo Sindaco. Se il Sindaco muore senza aver mai designato un successore, o se il successore designato è morto, i GM determinano casualmente chi riceverà la carica.

Ogni giorno, oltre a votare per mandare qualcuno al rogo, ogni abitante del villaggio può votare per l'elezione di un nuovo Sindaco. Se alla fine della giornata un personaggio ha ricevuto il voto di più del 50\% dei personaggi vivi, riceve la carica di Sindaco (e il Sindaco precedente la perde).

L'eventuale elezione di un nuovo Sindaco ha effetto prima della condanna a morte sul rogo: in caso di parità conta il voto del nuovo Sindaco, non di quello vecchio.


\pagebreak
\section{Le cose che vorreste sapere ma non avete mai osato chiedere}
\label{faq}

Le regole spiegate fino a questo punto sono più che sufficienti per poter giocare. Tuttavia possono accadere eventi piuttosto strani (ad esempio, cosa succede se due Sequestratori si sequestrano a vicenda?). Ai giocatori è sempre permesso chiedere ai GM cosa accade in una di queste combinazioni.

Segue una lista, verosimilmente non esaustiva, di eventualità bizzarre con le corrispettive spiegazioni.
Per rendere questo elenco più facilmente consultabile, più avanti vi è un indice analitico con la lista dei ruoli (o altre parole chiave) e dei punti in cui essi compaiono.

\begin{enumerate}
 
 \item In caso di contraddizione nell'utilizzo dei poteri, viene scelto casualmente un personaggio (o più di uno, se necessario), che non utilizza il proprio potere e riceve la consueta notifica di fallimento (questo è il fallimento ``di terza categoria'', vedi Sezione \ref{fallimento}).
 
%  Ad esempio, se un Esorcista, un Sequestratore e uno Spettro con il potere dell'Occultamento agiscono simultaneamente su un secondo Sequestratore che agisce sull'Esorcista, si verifica facilmente (esercizio) che non c'è alcun modo coerente di risolvere le azioni.
%  \index{Esorcista}
%  \index{Sequestratore}
%  \index{Spettro!Occultamento}
 \index{Fallimento dei poteri@\emph{Fallimento dei poteri}}
  
 \item Se nella stessa notte un Messia e un Negromante cercano di agire sullo stesso personaggio (morto), quel personaggio viene resuscitato dal Messia e non risvegliato come Spettro. In particolare, il Negromante riceve una notifica di fallimento.
 \index{Messia}
 \index{Negromante}

 \item Se un Trasformista usa il proprio potere su un Trasformista morto che aveva già acquisito un nuovo potere, egli acquisisce a sua volta quello stesso potere.
 \index{Trasformista}
 
 \item I Negromanti non possono nemmeno provare a usare il proprio potere se durante la notte precedente è stato creato uno Spettro.
 \index{Negromante}
 
 \item Se più Negromanti cercano di creare degli Spettri, falliscono come succede nel caso dei Lupi, a meno che scelgano tutti lo stesso personaggio e selezionino tutti lo stesso potere soprannaturale.
 \index{Negromante}

 \item Se un Ipnotista morto (che non è divenuto uno Spettro) viene resuscitato dal Messia, tutti coloro che si trovavano sotto il controllo dell'Ipnotista al momento della sua morte sono nuovamente sotto il suo controllo, a meno che nel frattempo la loro mente sia stata controllata da un altro Ipnotista.
 
 Se un personaggio sotto il controllo di un Ipnotista muore e viene successivamente resuscitato, torna ad essere sotto il controllo dell'ultimo Ipnotista ad aver agito su di esso.
 \index{Ipnotista}
 \index{Messia}
 
 \item Se un personaggio smette di essere sotto il controllo di un Ipnotista (ad esempio perché un altro Ipnotista usa il suo potere su di esso), l'Ipnotista non riceve alcuna notifica.
 \index{Ipnotista}
 
 \item Se più Ipnotisti agiscono sullo stesso personaggio durante la stessa notte, la mente di quest'ultimo viene controllata da tutti gli Ipnotisti in un ordine casuale, e perciò risulterà infine essere sotto il controllo soltanto dell'ultimo ad aver agito. Gli altri Ipnotisti non ricevono alcuna notifica di fallimento.
 \index{Ipnotista}
 
 \item Se uno Spettro con il potere dell'Illusione fa comparire l'illusione di un personaggio in una casa dove agiscono anche quel personaggio e il Voyeur, il Voyeur vede comunque solo una volta quel personaggio. Se tuttavia il personaggio scelto coincide con il Voyeur, questi compare nella lista dei personaggi visti.
 \index{Spettro!Illusione}
 \index{Voyeur}

 \item Se l'Esorcista e lo Spettro con il potere dell'Occultamento agiscono sulla stessa persona, è il potere dello Spettro a non avere effetto, mentre quello dell'Esorcista funziona regolarmente.
 \index{Esorcista}
 \index{Spettro!Occultamento}
 
 \item La Guardia del corpo protegge il personaggio scelto solamente dall'attacco dei Lupi. L'Assassino, lo Spettro con il potere della Morte ed eventuali altri effetti che uccidono il personaggio scelto non sono influenzati dal potere della Guardia del corpo.
 \index{Guardia del corpo}
 \index{Lupo}
 \index{Assassino}
 \index{Spettro!Morte}
 
%  \item Se in notti diverse un personaggio muore, viene risvegliato come Spettro, resuscitato dal Messia e la sua anima viene letta dal Diavolo, il Diavolo scopre che il personaggio è uno Spettro, ma non scopre che potere soprannaturale ha.
%  \index{Spettro}
%  \index{Messia}
%  \index{Diavolo}
 
 \item Se più Fattucchiere agiscono sulla stessa persona, ciascuna cambia il colore dell'aura. Pertanto, il colore dell'aura percepito durante quella notte è quello corretto se le Fattucchiere in questione sono in numero pari, mentre è quello sbagliato se le Fattucchiere sono in numero dispari.
 \index{Fattucchiera}
 
 \item Se al momento della morte del Fantasma non sono più disponibili poteri soprannaturali, il Fantasma non diventa uno Spettro.
 I Negromanti hanno sempre la priorità sui Fantasmi morti di notte: prima avviene l'eventuale risveglio di un morto come Spettro ad opera dei Negromanti, e poi i Fantasmi appena morti ottengono casualmente uno dei poteri soprannaturali ancora disponibili (se ve ne sono).
 Se più Fantasmi muoiono contemporaneamente e non sono disponibili abbastanza poteri soprannaturali per assegnarne a ciascuno di essi, vengono scelti casualmente i Fantasmi che diventano effettivamente degli Spettri e ottengono un potere soprannaturale.
 \index{Fantasma}
 \index{Negromante}
 
 \item Un personaggio può votare sé stesso, sia per la condanna a morte sul rogo che per l'elezione del Sindaco.
 \index{Votazione per il rogo@\emph{Votazione per il rogo}}
 \index{Elezione del Sindaco@\emph{Elezione del Sindaco}}
 
 \item Se più personaggi ottengono il massimo numero di voti per il rogo e quello scelto casualmente per essere ucciso è stato protetto dall'Avvocato del Diavolo, nessuno muore e non viene resa pubblica l'identità del personaggio protetto dall'Avvocato.
 \index{Avvocato del Diavolo}
 \index{Votazione per il rogo@\emph{Votazione per il rogo}}
 
 \item Se durante lo stesso giorno un personaggio viene eletto Sindaco e condannato a morte sul rogo, immediatamente dopo viene nominato casualmente un nuovo Sindaco (non vi è stato tempo per la nomina di un successore).
 \index{Elezione del Sindaco@\emph{Elezione del Sindaco}}
 
 \item Il poteri dell'Ipnotista e dello Scrutatore e i poteri soprannaturali dell'Amnesia e dell'Ipnosi non incidono sull'elezione del Sindaco, solo sul voto per il rogo.
 \index{Ipnotista}
 \index{Spettro!Amnesia}
 \index{Spettro!Ipnosi}
 \index{Scrutatore}
% \index{Spettro!Duplicazione}
 \index{Elezione del Sindaco@\emph{Elezione del Sindaco}}
 
%  \item Se lo Spettro con il potere dell'Amnesia e lo Spettro con il potere della
% Duplicazione agiscono contemporaneamente su di un personaggio, il voto di quel
% personaggio è comunque annullato. In altre parole, il potere dell'Amnesia
% annulla il potere della Duplicazione.
%  \index{Spettro!Amnesia}
%  \index{Spettro!Duplicazione}
 
%  \item Quando lo Spettro con il potere della Duplicazione agisce, il villaggio
% non viene messo al corrente dell'identità del personaggio che ha beneficiato del
% potere. Per esempio, se il personaggio X viene votato da A, B e C, e durante la
% notte precedente lo Spettro aveva usato il suo potere su C, allora il server
% riporterà un messaggio simile a ``X è stato votato da A, B e C, ricevendo un
% totale di 4 voti''.
%  \index{Votazione per il rogo@\emph{Votazione per il rogo}}
%  \index{Spettro!Duplicazione}
 
%  \item Lo Spettro con il potere dell'Illusione che agisce su un personaggio può dirigere l'illusione dal personaggio stesso. In questo caso, un eventuale Stalker che utilizzi a sua volta il suo potere sullo stesso personaggio riceve informazioni come se tale personaggio non avesse agito.
%  \index{Spettro!Illusione}
%  \index{Stalker}
 
 \item  Se lo Spettro con il potere dell'Illusione decide di dirigere l'illusione nella casa del personaggio su cui ha agito anche un Esorcista, il suo potere non ha effetto.
 Se invece lo Spettro decide di generare l'illusione di un personaggio su cui ha agito anche un Esorcista, il suo potere funzi: prima avviene l'eventuale risveglio di un morto come Spettro ad opera dei Negromanti, e poi i Fantasmi appena morti ottengono casualmente uno dei poteri soprannaturali ancora disponibili (se ve ne sono).
 ona normalmente.
 \index{Esorcista}
 \index{Spettro!Illusione}
 
 \item Cercare di uccidere un personaggio che nella stessa notte muore anche per altri motivi non costituisce di per sé una causa di fallimento.
 Per esempio, se contemporaneamente un Lupo e lo Spettro con il potere della Morte utilizzano il proprio potere su di uno stesso personaggio, hanno entrambi successo (a meno che non intervengano cause di fallimento, come spiegato nella Sezione \ref{fallimento}).
 \index{Morte@\emph{Morte}}
 \index{Fallimento dei poteri@\emph{Fallimento dei poteri}}
 \index{Lupo}
 \index{Spettro!Morte}
 
 \item Al fine di determinare la validità della votazione, la percentuale di votanti per il rogo è calcolata dopo aver applicato gli effetti dell'Ipnotista e dello Spettro con il potere dell'Amnesia.
 Per esempio, se su 10 personaggi vivi ve ne sono 5 sotto il controllo dell'Ipnotista, e l'Ipnotista non vota, allora il quorum del 50\% non è raggiunto perché l'Ipnotista e i personaggi sotto il suo controllo non votano.
 \index{Votazione per il rogo@\emph{Votazione per il rogo}}
 \index{Ipnotista}
 \index{Spettro!Amnesia}
 %\index{Spettro!Duplicazione}

 \item Il potere soprannaturale dell'Occultamento ha effetto anche sugli altri Spettri.
 \index{Spettro}
 \index{Spettro!Occultamento}
 \index{Fallimento dei poteri@\emph{Fallimento dei poteri}}
 
 \item Se lo Scrutatore e lo Spettro con il potere dell'Amnesia agiscono entrambi su uno stesso personaggio, il voto originariamente espresso da quel personaggio viene cancellato (per effetto del potere dell'Amnesia), ma il voto aggiunto dallo Scrutatore rimane valido. In particolare, dal risultato della votazione non è possibile distinguere il caso in cui questo accade dal caso in cui il personaggio scelto ha votato il personaggio indicato
 dallo Scrutatore (a meno che abbiano agito più Scrutatori).
 \index{Scrutatore}
 \index{Spettro!Amnesia}
 
 \item Se più Scrutatori agiscono su uno stesso personaggio, ciascuno di essi aggiunge un voto (anche se tali voti sono indirizzati al medesimo personaggio).
 \index{Scrutatore}
 
 \item Se lo Scrutatore e/o lo Spettro con il potere dell'Ipnosi agiscono su un personaggio che muore durante la stessa notte, l'effetto del loro potere è cancellato. 
 La stessa cosa succede se a morire è il personaggio a cui era diretto il voto. Questa eventualità non è considerata un motivo di fallimento.
 \index{Scrutatore}
 \index{Spettro!Ipnosi}
 \index{Fallimento dei poteri@\emph{Fallimento dei poteri}}
 
%  \item Se un Trasformista usa il suo potere con sucesso su un Ipnotista, in ogni caso non viene conteggiato fra gli Ipnotisti vivi ai fini della creazione dello Spettro con il Potere dell'Ipnosi. In particolare, se tutti gli Ipnotisti appartenenti alla Fazione dei Negromanti muoiono, l'ultimo di essi a morire viene risvegliato come Spettro con il potere dell'Ipnosi. Analogamente, un Trasformista non potrà in nessun caso essere risvegliato come Spettro con il Potere dell'Ipnosi.
%  \index{Trasformista}
%  \index{Spettro!Ipnosi}
 
 \item Se più Ipnotisti muoiono la stessa notte, solamente uno di essi (scelto casualmente) viene risvegliato come Spettro con il potere dell'Ipnosi, sempre che tale potere non sia ancora stato assegnato.
 \index{Ipnotista}
 \index{Spettro!Ipnosi}
 
 \item I Negromanti hanno sempre la priorità sugli Ipnotisti morti di notte. Se nella stessa notte muore un Ipnotista, e i Negromanti tentano di risvegliare un morto come Spettro con il potere dell'Ipnosi, a meno dell'intervento di altre cause di fallimento questi hanno successo, e l'Ipnotista non viene risvegliato come Spettro.
 \index{Negromante}
 \index{Ipnotista}
 \index{Spettro!Ipnosi}
 
%  \item Se nella stessa notte l'ultimo Ipnotista (appartenente alla Fazione dei Negromanti) muore e i Negromanti risvegliano un personaggio come Spettro con il potere della Morte, allora l'Ipnotista morto viene comunque risvegliato come Spettro con il potere dell'Ipnosi.
%  \index{Ipnotista}
%  \index{Spettro!Ipnosi}
 
 \item Il voto eventualmente aggiunto dallo Scrutatore conta ai fini del raggiungimento del quorum.
 \index{Votazione per il rogo@\emph{Votazione per il rogo}}
 \index{Scrutatore}
 
 \item Se il voto aggiunto dallo Scrutatore è identico al voto espresso dal personaggio su cui è stato usato il potere dello Scrutatore, entrambi i voti vengono conteggiati per determinare il condannato a morte.
 Tuttavia, nella lista dei votanti, il personaggio su cui è stato usato il potere dello Scrutatore compare solo una volta.
 \index{Votazione per il rogo@\emph{Votazione per il rogo}}
 \index{Scrutatore}
 
 \item Se il Trasformista utilizza il proprio potere diventando un Ipnotista, al termine della notte in cui questo accade egli smette di essere affetto dai poteri di Ipnotisti, Spettro con il potere dell'Amnesia e Spettro con il potere dell'Ipnosi. In particolare, se era stato precedentemente ipnotizzato da un qualche Ipnotista, smette di essere ipnotizzato (ma l'Ipnotista non riceve alcuna notifica).
 
 Se, nella notte in cui il Trasformista diventa Ipnotista, sul Trasformista agiscono Ipnotisti, Spettro con il potere dell'Amnesia, oppure Spettro con il potere dell'Ipnosi, questi ultimi non ricevono una notifica di fallimento a causa della trasformazione in atto, sebbene il loro potere non abbia effetto (perché al termine della notte il Trasformista diviene immune a tali poteri).
 \index{Trasformista}
 \index{Ipnotista}
 \index{Spettro!Ipnosi}
 \index{Spettro!Amnesia}
 \index{Fallimento dei poteri@\emph{Fallimento dei poteri}}
 
 \item Lo Spettro dell'Ipnosi ha la precedenza sull'Ipnotista. Se lo Spettro con il potere dell'Ipnosi agisce su un personaggio controllato da un Ipnotista, il giorno seguente il personaggio vota come indicato dallo Spettro. Tuttavia il personaggio rimane sotto il controllo dell'Ipnotista.
 \index{Spettro!Ipnosi}
 \index{Ipnotista}
 
 \item L'assenza di possibili bersagli non costituisce causa di fallimento nell'utilizzo del potere dell'Assassino.
 \index{Assassino}
 \index{Fallimento dei poteri@\emph{Fallimento dei poteri}}
 
 \item I ruoli influenzati dal potere soprannaturale della Confusione sono i seguenti: Investigatore, Mago, Veggente, Diavolo, Medium, Spettro con il potere della Visione.
 \index{Spettro!Confusione}
 \index{Investigatore}
 \index{Mago}
 \index{Veggente}
 \index{Diavolo}
 \index{Medium}
 \index{Spettro!Visione}
 
 \item Ogni responso casuale dovuto all'azione dello Spettro con il potere della Confusione è generato scegliendo con la stessa probabilità una delle possibili risposte.
 Il responso casuale non è influenzato in alcun modo dalla composizione del Villaggio né dalle informazioni su di esso.
 
 Per esempio, il Diavolo e il Medium possono ricevere come risposta un ruolo non rappresentato o un ruolo che non può essere presente (come un Lupo dopo che la sua fazione è già stata esiliata).
 \index{Spettro!Confusione}
 \index{Diavolo}
 \index{Medium}
 
 \item Personaggi diversi che cercano di ottenere informazioni su un personaggio su cui ha agito lo Spettro con il potere della Confusione ricevono informazioni casuali indipendenti.
 Per esempio, se contemporaneamente due Veggenti e tale Spettro agiscono su un medesimo personaggio, i due Veggenti possono ottenere responsi differenti.
 \index{Spettro!Confusione}
 \index{Veggente}
 
 \item Se una o più Fattucchiere e lo Spettro con il potere della Confusione agiscono sullo stesso personaggio, chiunque cerchi di ottenere informazioni sul colore dell'aura di tale personaggio ottiene un responso casuale.
 \index{Fattucchiera}
 \index{Spettro!Confusione}
\end{enumerate}



\printindex




\end{document}
