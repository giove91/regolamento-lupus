\documentclass[a4paper,10pt]{article}


\usepackage[utf8]{inputenc}
\usepackage[italian]{babel}
\usepackage[T1]{fontenc}
\usepackage[dvips]{graphicx}
\usepackage{amsmath}
\usepackage{amsthm}
\usepackage{fancyhdr}
\usepackage{amsfonts}
\usepackage{amssymb}

\topmargin -1cm
\oddsidemargin -0.5cm
%\evensidemargin	-1cm
\textwidth 17cm


\newcommand{\smallspace}{\vskip0.3cm}

% Title Page
\title{Lupus in tempo reale\\ Regolamento della quarta edizione}
\author{Giove}

\begin{document}
\maketitle


\section{Introduzione}

...

\section{Partecipanti}

Possono partecipare:
\begin{itemize}
 \item gli studenti della Scuola;
 \item altre persone che frequentano spesso gli ambienti della Scuola, in particolare il collegio Carducci e la mensa, dal lunedì al venerdì.
\end{itemize}

Non è necessario saper già giocare a Lupus in Tabula, è sufficiente farsi spiegare le regole di base da qualcuno che sa già giocare.

Tutti i partecipanti si impegnano a farsi vedere un po' in giro negli ambienti della Scuola nei giorni della partita. Nessuno vi chiede di non partire venerdì mattina per tornare a casa se non avete lezione, ma se vivete da reclusi in camera vostra e non mangiate mai in mensa potrebbe essere il caso di non giocare.


\section{Quando si giocherà}

...


\section{Modifiche rispetto al regolamento della terza edizione}

...


\section{Giorni e notti di gioco}

I giorni di gioco si estendono dalle 8:00 (circa) alle 22:00 di lunedì, martedì, mercoledì e giovedì, e dalle 8:00 (circa) di venerdì alle 22:00 di domenica.
Le notti di gioco si estendono dalle 22:00 (circa) alle 8:00 di domenica, lunedì, martedì, mercoledì e giovedì.


\section{Svolgimento del giorno e votazioni}

L'inizio del giorno è annunciato sul sito web, insieme a tutte le informazioni ottenute da ciascun giocatore in seguito agli avvenimenti della notte appena trascorsa.
I personaggi vivi hanno il diritto di votare entro le ore 22 una persona del villaggio da uccidere. Dopo le ore 22, viene pubblicata la lista dei votanti insieme alla preferenza espressa da ciascuna persona. Il voto è ritenuto valido se almeno il 50\% dei vivi ha votato; in caso contrario, nessuno muore. La persona che ha ricevuto più voti di tutti muore se ha ricevuto almeno il 30\% dei voti totali espressi; in caso contrario, nessuno muore. Se due o più persone hanno ricevuto lo stesso numero di voti, che è superiore a quello dei voti ricevuti da chiunque altro, ed ognuna ha ricevuto almeno il 30\% dei voti totali espressi, muore quella che è stata votata dal Sindaco (vedi Sezione XYZ); se nessuna delle due è stata votata dal sindaco, ne muore una a caso.

TODO: elezione del Sindaco (?)


\section{Svolgimento della notte e poteri speciali}

La notte inizia nel momento in cui viene pubblicato l'esito della votazione del giorno, e termina alle 8:00 della mattina seguente.
Durante la notte, i personaggi che hanno poteri speciali hanno la facoltà di attivarli.
Alcuni personaggi hanno un potere speciale attivabile ``ogni due notti'': si intende che tale potere può essere usato se e solo se non è stato usato durante la notte precedente (in altre parole, l'unica restrizione è data dal non poterlo usare in due notti consecutive).


\section{Comunicazioni GM-giocatori, votazioni, attivazioni dei poteri}

All'inizio della partita, ciascun giocatore riceve le credenziali per accedere all'interfaccia web XYZ (verosimilmente sarà http://uz.sns.it/lupus/).
Le votazioni durante il giorno e le attivazioni dei poteri notturni avvengono tutte tramite l'interfaccia web. Le informazioni sulle votazioni del giorno e sugli avvenimenti della notte compaiono a loro volta sull'interfaccia web (le informazioni pubbliche possono essere visualizzate senza bisogno di autenticazione, mentre quelle private sono accessibili solo dopo il login).
In caso di impossibilità di accedere a internet, i giocatori possono eccezionalmente contattare il GM (ad esempio via SMS) per comunicargli le proprie intenzioni di voto e/o il modo in cui desiderano utilizzare il proprio potere notturno.

Le comunicazioni tra GM e giocatori sono inviolabili: è vietato origliare le conversazioni del GM con i giocatori, fare pressione sul GM in qualsiasi modo, inviargli e-mail con intenti truffaldini, rifiutarsi di rispondere sinceramente alle sue domande sui messaggi ricevuti e qualsiasi azione che ricordi anche solo vagamente le precedenti.
È inoltre vietato spiare altri giocatori mentre accedono all'area riservata dell'interfaccia web, rubare o hackare account altrui (anche approfittando di eventuali distrazioni), mostrare o dare accesso al proprio account ad altri giocatori, o cercare di violare la sicurezza dell'interfaccia web.
Le medesime regole si applicano alla lettera personale che viene data a ciascun giocatore all'inizio della partita (contenente il ruolo assegnatogli).
La violazione di queste regole può portare alla squalifica del giocatore o, nei casi più gravi, dell'intera fazione.


\section{Comunicazioni giocatori-giocatori}

I giocatori possono comunicare tra di loro con qualsiasi mezzo. Possono dirsi qualsiasi cosa.
L’unica cosa illecita è utilizzare truffaldinamente le e-mail per capire i ruoli delle altre persone: mostrare ad un altro giocatore le informazioni che si sono ricevute dal GM (via interfaccia web, via e-mail o con qualsiasi altro mezzo) è vietato; pressioni come ``fammi vedere la tua casella di posta da lontano, per vedere quante mail ricevi sul gioco'' sono vietate; mandare email che sembrano provenire da un indirizzo di posta diverso dal proprio per imbrogliare il destinatario è vietato; qualunque cosa che assomigli alle precedenti è vietata. La violazione di queste regole può portare alla squalifica del giocatore o, nei casi più gravi, dell'intera fazione.

Anche i morti possono parlare con gli altri membri del villaggio, e continuano a giocare. Si incoraggiano i giocatori a non comunicare solo via e-mail ma anche di persona.



\section{Fazioni e condizioni di vittoria}

I giocatori sono divisi in tre fazioni: i Buoni, i Cattivi e i Non-morti.
Una fazione vince se, subito dopo l'alba oppure subito dopo la votazione del tramonto, tutti i personaggi vivi appartengono a quella fazione.

Se una fazione ha chiaramente vinto prima che le condizioni di vittoria precedenti siano rispettate (ad esempio, se sono rimasti 5 Lupi, altri 5 Cattivi e 7 Contadini, ed i Lupi ed i Cattivi si conoscono), il GM può porre fine alla partita in anticipo (ma non deve farlo per forza).


\section{Giocare dopo la disfatta}

...


\section{Suicidio}

Ogni personaggio ha un potere attivabile di notte: il suicidio.
Se un personaggio si suicida, viene trovato morto la mattina seguente. Quando qualcuno muore di notte, il villaggio sa solo che è morto, non come ciò sia accaduto.
Questa regola è stata introdotta principalmente per permettere a chi decide di uscire dal gioco di farlo quando desidera (ma, ovviamente, ci si può suicidare per fare un bluff di qualche tipo, se si vuole).


\section{Composizione del villaggio}

La composizione del villaggio è scelta dal GM e non è nota ai partecipanti. I ruoli sono assegnati casualmente ai giocatori.
Tutte e tre le fazioni sono sicuramente presenti. All'inizio della partita, i Cattivi e i Non-morti sono in totale circa $1/4$ dei giocatori, ma il loro numero non è noto con precisione ai giocatori.

Il GM è libero, all'inizio della partita, di dare alcune indicazioni sulla composizione (ad esempio, potrebbe dire: Su $40$ giocatori presenti, i lupi sono più di $5$, i Cattivi non sono più di $11$ e ci sono almeno un Veggente ed un Medium). La composizione della fazione del vampiro viene sicuramente comunicata a tutti (?).


\section{Ruoli}

Non è necessario che tutti i giocatori sappiano cosa fa ogni ruolo; se siete particolarmente pigri, è sufficiente che sappiate quale potere avete voi (vi verrà ricordato nella lettera in cui vi si assegnerà il ruolo). Tuttavia, per giocare in modo più efficace, è certamente conveniente sapere anche cosa possono fare gli altri.


I seguenti sono i ruoli che \emph{possono} comparire nel gioco.


\subsection*{Personaggi Buoni}

\begin{itemize}
 \item {\bf Contadino} (aura bianca). Il Contadino non ha alcun potere particolare.
 
 \item {\bf Veggente} (aura bianca, mistico). Ogni notte, il Veggente può scegliere un personaggio vivo per scoprirne il colore dell'aura.

 \item {\bf Guardia del corpo} (aura bianca). Ogni notte, la Guardia del corpo può scegliere un personaggio vivo (ma non se stessa) e proteggerlo. Durante la notte, tale personaggio non potrà morire per effetto dell'attacco dei Lupi.
 
 \item Massoni (aura bianca). I Massoni si conoscono tra loro.
 
 \item (DA MODIFICARE) Medium (aura bianca, mistico). Ogni notte, il Medium può scegliere un personaggio morto per scoprirne il colore dell'aura. Scopre inoltre se tale personaggio è diventato o meno uno Spettro.

 \item Messia (aura bianca, mistico). Una sola volta in tutta la partita, il Messia può scegliere di resuscitare una persona morta. Quella persona ritornerà in vita il giorno seguente, riacquistando i suoi poteri speciali (ma non la carica di sindaco, qualora l'avesse avuta).

 \item Voyeur (aura bianca). Ogni due notti, il Voyeur può scegliere una persona, viva o morta, di cui essere morbosamente infatuato, e può andare a spiare la sua camera. Il Voyeur scopre quali sono le persone che durante la notte sono entrate nella sua casa (ad esempio, il Veggente entra nella casa di qualcuno per scrutarne l'aura).
 Il Voyeur non può usare il suo potere su se stesso.

 \item Stalker (aura bianca). Ogni due notti, lo Stalker può indicare una persona viva. Durante la notte seguirà questa persona, scoprendo se è uscita di casa e dove è andata (ma non cosa è andata a fare).
 Ad esempio, se lo Stalker segue un Lupo e si tratta del Lupo che va ad uccidere, lo vedrà andare a casa della persona che i Lupi hanno scelto.
 Lo Stalker non può usare il suo potere su sè stesso.

 
\end{itemize}





\end{document}
