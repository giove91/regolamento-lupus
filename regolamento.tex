\documentclass[a4paper,10pt]{article}


\usepackage[utf8]{inputenc}
\usepackage[italian]{babel}
\usepackage{amsmath}
\usepackage{amsthm}
\usepackage{fancyhdr}
\usepackage{amsfonts}
\usepackage{amssymb}
\usepackage{makeidx}
\usepackage[parfill]{parskip}
\usepackage[colorlinks]{hyperref}
\usepackage{fullpage}
\usepackage{mathpazo}
%\usepackage{utopia}
\usepackage{graphicx}

% * define a `\twoidxcolumn` based on `\twocolumn`:
\def\twoidxcolumn{%
%\clearpage
\global\columnwidth\textwidth%
\global\advance\columnwidth-\columnsep%
\global\divide\columnwidth\tw@%
\global\hsize\columnwidth%
\global\linewidth\columnwidth%
\global\@twocolumntrue%
\global\@firstcolumntrue%
\col@number \tw@%
%\@ifnextchar [\@topnewpage
\@floatplacement%
}%


\newcommand{\smallspace}{\vskip0.3cm}

% Title Page
\title{Lupus in tempo reale\\ Regolamento della decima edizione}
\author{Alessandro Iraci, Giovanni Italiano, Giovanni Mascellani,\\ Matteo Migliorini, Giovanni Paolini, Leonardo Tolomeo}

\begin{document}
	
\maketitle

\begin{tabular}{lp{0.6\textwidth}}
	\begin{minipage}{0.22\textwidth}
		\vspace{3mm}
		\href{http://creativecommons.org/licenses/by/4.0/}{\includegraphics{ccby.pdf}}
	\end{minipage}
	&
	This work is licensed under a \href{http://creativecommons.org/licenses/by/4.0/}{Creative Commons Attribution 4.0 International License}.
\end{tabular}


\section{Introduzione}

\subsection{Cos'è Lupus in tempo reale?}

\emph{Lupus in tempo reale} è una variante dei più tradizionali \emph{Lupus in Tabula} e \emph{Mafia}, caratterizzata da un maggiore coinvolgimento e da interazioni più sofisticate tra i giocatori.
La principale differenza consiste nel ritmo del gioco: la partita si estende infatti su diversi giorni, in modo aderente all'ambientazione.
Sono premiate dalla dinamica di gioco la logica, la capacità di bluffare (e di smascherare i bluff altrui), la capacità di saper coordinare un’azione di gruppo, il carisma e le velleità leaderistiche.

\emph{Lupus in tempo reale} è stato inventato nel 2010, ed è stato giocato una volta all'anno fino al 2012 sotto la supervisione di Francesco Guatieri come principale Game Master. A partire dall'anno 2014 è passato sotto il controllo di Giovanni Mascellani e Giovanni Paolini, e da lì in poi l'organigramma si è esteso edizione per edizione.

Questo regolamento è stato messo la prima volta per iscritto da Andrea Caleo in occasione della terza edizione del gioco. Da allora in poi è stato di volta in volta aggiornato, con cambiamenti che sono stati discussi e stabiliti principalmente da Luca Ghidelli, Alessandro Iraci, Giovanni Mascellani, Giovanni Paolini, Leonardo Tolomeo, Matteo Migliorini, Giovanni Italiano, Claudio Afeltra, Edoardo Rizzi, Giovanni Interdonato e Lorenzo Pierro.

Nel seguito di questo regolamento, verrà utilizzata la sigla ``GM'' come abbreviazione di ``Game Master''.

\subsection{Ambientazione}

La storia si svolge in quello che una volta era un tranquillo villaggio di campagna. A turbare la quiete del posto è un evento quantomeno atipico, ossia un'invasione di lupi mannari. Nascosti sotto le spoglie di normalissimi contadini, i licantropi hanno intenzione di prendere il controllo del villaggio e sbranare chiunque non sia disposto ad allearsi con loro.

Le autorità nazionali, venendo a sapere dell'accaduto e temendo il diffondersi della licantropia in tutto il paese, hanno disposto un blocco forzato delle vie d'accesso al villaggio, rendendo virtualmente impossibile a chiunque entrare o uscire dalla frazione.

Per tentare di ristabilire la normalità viene imposta una durissima legge marziale: ogni sera, il villaggio convocherà un'assemblea composta da tutti gli abitanti, al termine della quale un sospetto lupo mannaro verrà arso sul rogo. La sfiducia nei confronti degli altri abitanti è totale, chiunque potrebbe essere un licantropo sotto mentite spoglie. La propria vicina di casa, l'insospettabile anziano signore dell'ortofrutta, o anche quel nuovo arrivato peloso e con le orecchie a punta, così normali di giorno, potrebbero col calar della notte assumere le sembianze di un feroce lupo assetato di sangue.

Come se non bastasse, da qualche giorno corre voce che nel villaggio si nasconda una setta di oscuri negromanti. Persone in grado di adoperare la magia nera, un'abilità tenuta nascosta per paura di essere scoperti. La confusione creatasi nel piccolo borgo, il numero di morti in probabile aumento, e la protezione offerta dai lupi mannari, potrebbero fornire loro un'occasione irrinunciabile per cominciare a praticare la propria arte, richiamando alla vita le vittime sotto forma di spettri, e utilizzando i loro poteri per rendere il villaggio schiavo della setta.

Fra i cittadini c'è paura, e tanta indecisione su quale sia la cosa giusta da fare. Meglio combattere lupi mannari e negromanti e cercare di liberare il villaggio, oppure allearsi con loro nella speranza di avere salva la vita, e magari anche di ottenere una posizione di potere? Meglio rischiare la pelle andando ad origliare dietro la porta del proprio vicino, cercando di capire se questi è un lupo, oppure stare tranquilli nel proprio letto sperando di non ricevere visite sgradite? Molti abitanti hanno deciso di attivarsi per riprendere il controllo del villaggio; altri hanno ceduto, per paura della morte o per sete di potere, alle lusinghe dei licantropi o dei negromanti, aiutandoli a portare a termine i loro progetti.

Ciascuno ha tutto l'interesse a tenere nascosta la propria scelta: per i traditori, venire scoperti potrebbe significare morte sul rogo; per i villici che hanno deciso di combattere l'invasione, esporsi troppo potrebbe voler dire essere la prossima vittima dei lupi mannari, oppure ricevere terrificanti visite da parte di uno spettro. Ma non sempre tacere per la propria incolumità è la scelta giusta. In determinate circostanze rivelare preziose informazioni, anche a costo di rischiare la vita, potrebbe essere fondamentale: veder trionfare i lupi mannari, in ogni caso significa morte.

Qualunque sia stata la propria decisione, tuttavia, la maggior parte dei cittadini si attiva durante la notte per raccogliere informazioni o infastidire i nemici: per esempio appostandosi fuori da una casa per cercare di capire cosa succede all'interno, utilizzando le proprie capacità divinatorie per scoprire l'identità degli altri, girando armati per proteggere i concittadini dall'attacco dei licantropi, o compiendo rituali per tenere lontani gli spettri. Ciascuno fa il possibile, ma anche gli alleati di lupi mannari si mobilitano per intralciare le indagini, diffondere false informazioni e manipolare le votazioni durante l'assemblea.

Nel villaggio, ormai, la vita è una questione di doppio gioco; ogni passo falso rischia di costare la pelle non solamente a sé stessi, ma anche a tutti i propri alleati. Un delicatissimo equilibrio coinvolge le due parti, cercare di spezzarlo in proprio favore è l'obiettivo di tutti. Solo i più abili manipolatori riusciranno a prendere in mano la situazione e condurre la propria squadra alla vittoria. Lo scopo da raggiungere è chiaro, ma quale sia la fazione che riuscirà a trionfare dipende da voi...

\subsection{Partecipanti}

Possono partecipare:

\begin{itemize}
	\item gli studenti della Scuola;
	\item altre persone che frequentano spesso gli ambienti della Scuola, in particolare il collegio Faedo (solitamente) e la mensa, dal lunedì al venerdì.
\end{itemize}

Tutti i partecipanti si impegnano a farsi vedere un po' in giro negli ambienti della Scuola nei giorni della partita. Nessuno vi chiede di non partire venerdì mattina per tornare a casa se non avete lezione, ma se vivete da reclusi in camera vostra e non mangiate mai in mensa potrebbe essere il caso di non giocare.

% 
% \subsection{Quando si giocherà}
% 
% La partita inizierà la sera di martedì 18 novembre 2014, con ritrovo al collegio Carducci. % TODO

\clearpage

% \subsection{Principali cambiamenti nella quarta edizione}
% 
% Questa sezione è pensata per dare a chi ha giocato alla terza edizione di Lupus
% un'idea delle modifiche più importanti che sono state apportate. Tuttavia non
% sostituisce un'attenta lettura del resto del regolamento.
% 
% \begin{itemize}
%  \item La Fazione dei Vampiri è stata sostituita dalla nuova Fazione dei
% Negromanti, che ha meccaniche di gioco molto diverse.
%  \item È stato eliminato il suicidio.
%  \item È stato eliminato il rituale dei fantasmi.
%  \item I possibili colori dell'aura sono tornati ad essere solamente il bianco e
% il nero.
%  \item Molti ruoli sono stati creati, eliminati o modificati. A chi è abituato
% al regolamento della terza edizione raccomandiamo di controllare con particolare
% attenzione i seguenti ruoli, che hanno subito i cambiamenti più rilevanti:
% Cacciatore, Custode del cimitero, Esorcista, Investigatore, Rinnegato, Profanatore di tombe, Negromante, Fantasma, Ipnotista, Medium, Spettro.
%  \item Le comunicazioni di gioco (votazione, utilizzo dei poteri, resoconto
% dell'alba e del tramonto) avvengono attraverso un'interfaccia web e non più via
% e-mail.
% \end{itemize}
% 
% \subsection{Principali cambiamenti nella quinta edizione}
% 
% Le seguenti sono le modifiche apportate rispetto alla quarta edizione di Lupus in tempo reale.
% 
% \begin{itemize}
% \item È stato introdotto lo Scrutatore.
% \item È stato eliminato lo Spettro con il potere della Duplicazione.
% \item L'ultimo Ipnotista, quando muore, diventa uno Spettro con il potere dell'Ipnosi.
% \item Il Profanatore di Tombe è stato trasferito alla Fazione dei
%   Popolani ed è stato rinominato in Sciamano. 
%   Rimane con aura nera, ma diventa mistico.
% \item Il Necrofilo è stato rinominato in Trasformista.
% \item Il Medium scopre il ruolo del personaggio scelto, e non più l'aura.
% \end{itemize}

% 
% 
% \subsection{Principali cambiamenti nella sesta edizione}
% 
% Questa sezione è pensata per dare a chi ha giocato alla quinta edizione di Lupus un'idea delle modifiche più importanti che sono state apportate. Tuttavia non sostituisce un'attenta lettura del resto del regolamento.
% 
% \begin{itemize}
%   \item Il Trasformista e il Fantasma hanno ora aura nera.
%   \item Nessun personaggio può utilizzare il proprio potere su sé stesso.
%   \item Sono state rimosse le restrizioni sull'utilizzo di un potere due notti consecutive sullo stesso personaggio.
%   \item Sono stati eliminati il Custode del Cimitero, il Cacciatore, e il potere soprannaturale della Mistificazione.
%   \item Sono stati introdotti l'Assassino, lo Stregone, e il potere soprannaturale della Confusione.
%   \item I personaggi della Fazione dei Lupi non possono diventare Spettri.
%   \item Le conoscenze iniziali dei personaggi della Fazione dei Lupi sono state modificate.
%   \item Il potere soprannaturale dell'Ipnosi può ora essere scelto dai Negromanti; il primo Ipnotista che muore diventa uno Spettro con tale potere (se disponibile).
%   \item Il Messia fallisce se utilizza il proprio potere su uno Spettro.
%   \item L'Avvocato del Diavolo è stato rinominato in Avvocato.
% \end{itemize}
%
%\subsection{Principali cambiamenti nella settima edizione}
% 
%Questa sezione è pensata per dare a chi ha giocato alla sesta edizione di Lupus un'idea delle modifiche più importanti che sono state apportate. Tuttavia non sostituisce un'attenta lettura del resto del regolamento.
% 
%\begin{itemize}
%	\item Il poteri speciali del Trasformista, dei Lupi, ed il potere soprannaturale della Morte hanno effetto solo sui componenti della Fazione dei Popolani.
% 	\item Il potere speciale del Diavolo non ha effetto sui componenti della Fazione dei Negromanti.
% 	\item Il potere speciale dei Negromanti può essere usato ogni notte.
% 	\item L'Ipnotista non diviene uno Spettro dopo la sua morte.
% 	\item I seguenti poteri soprannaturali sono stati modificati: Confusione, Ipnosi, Morte, Visione. È stato aggiunto un nuovo potere soprannaturale: Corruzione.
% 	\item Le restrizioni sulla spettrificazione sono state modificate.
% 	\item È stato reintrodotto il ruolo di Custode del cimitero. Oltre al vecchio potere, Custode del cimitero e Guardia del corpo scoprono anche quanti personaggi hanno agito sul loro stesso bersaglio.
% 	\item È stato introdotto un nuovo ruolo: il Necrofilo. Appartiene alla Fazione dei Lupi.
% 	\item La Fattucchiera ha aura bianca.
% 	\item Il potere speciale dello Scrutatore è stato modificato.
%\end{itemize}

%     \subsection{Principali cambiamenti nell'ottava edizione} % TODO: Finire la lista
% 
%     Questa sezione è pensata per dare a chi ha giocato alla settima edizione di Lupus un'idea delle modifiche più importanti che sono state apportate. Tuttavia non sostituisce un'attenta lettura del resto del regolamento.
% 
%     \begin{itemize}
% 	    \item Le meccaniche di spettrificazione sono state modificate, così come il funzionamento degli Spettri.
% 	    \item Gli Spettri sono ora associati ad Incantesimi e non a poteri soprannaturali.
% 	    \item I poteri speciali si chiamano ora \emph{abilità}. I poteri soprannaturali si chiamano ora semplicemente \emph{poteri}.
% 	    \item I fallimenti non contano più come utilizzo di abilità o potere se il suo utilizzo è permesso ogni due notti, oppure una sola volta a partita.
% 	    \item Per uccidere qualcuno sul rogo, occorre che questi riceva più del 50\% dei voti disponibili. Inoltre, i voti espressi per il rogo adesso sono segreti; il totale dei voti ricevuti da ciascuno rimane pubblico.
% 	    \item La carica di Sindaco è stata eliminata.
% 	    \item L'alba è stata spostata alle ore 9:00.
% 	    \item Tutti i componenti della Fazione dei Lupi si conoscono fra di loro. Lo stesso vale per la Fazione dei Negromanti.
% 	    \item Sono state eliminate le restrizioni sull'uccisione dei componenti delle Fazioni dei Lupi e dei Negromanti relative all'altra Fazione.
% 	    \item Sono stati eliminati i seguenti ruoli: Custode del cimitero, Avvocato, Necrofilo, Rinnegato, Ipnotista, Medium, Scrutatore.
% 	    \item Sono stati modificati i seguenti ruoli: Divinatore, Esorcista, Investigatore, Sciamano, Trasformista, Diavolo, Fattucchiera, Sequestratore, Negromante.
% 	    \item Sono stati introdotti i seguenti ruoli: Cacciatore, Spia, Alcolista.
% 	    \item Sono stati eliminati i seguenti Incantesimi: Ipnosi, Visione.
% 	    \item Sono stati modificati i seguenti Incantesimi: Amnesia, Confusione, Morte, Occultamento.
% 	    \item Sono stati introdotti i seguenti Incantesimi: Assoluzione, Diffamazione, Telepatia, Vita.
%     \end{itemize}

\subsection{Principali cambiamenti nella decima edizione} % TODO: FARE la lista

Questa sezione è pensata per dare a chi ha giocato alla ottava o alla nona edizione di Lupus un'idea delle modifiche più importanti che sono state apportate. Tuttavia non sostituisce un'attenta lettura del resto del regolamento.

\begin{itemize}
	\item La Fazione dei Negromanti è stata eliminata.
	\item I ruoli di Negromante e Fantasma fanno ora parte della Fazione dei Lupi.
	\item È stata introdotta la possibilità che alcuni giocatori ricevano informazioni iniziali aggiuntive oltre a quelle pubbliche.
	\item Sono stati eliminati i seguenti ruoli: Spia, Massone, Alcolista.
	\item Sono stati modificati i seguenti ruoli: Esorcista, Sciamano, Stregone, Negromante, Fantasma.
	\item Sono stati introdotti i seguenti ruoli: Apprendista.
	\item Gli Incantesimi sono stati modificati per riadattarli alla fazione dei Lupi.
\end{itemize}

\pagebreak

\section{Metagioco}

\subsection{Giorni e notti di gioco}

I giorni di gioco si estendono (salvo eccezioni concordate a inizio partita) dalle 9:00 (circa) alle 22:00 di lunedì, martedì, mercoledì e giovedì, e dalle 9:00 (circa) di venerdì alle 22:00 di domenica. Le notti di gioco si estendono dalle 22:00 (circa) di domenica, lunedì, martedì, mercoledì e giovedì, alle 9:00 del giorno successivo. 

\subsection{Interfaccia web}

Prima dell'inizio della partita, ciascun giocatore può crearsi un account per accedere all'interfaccia web di Lupus, il cui indirizzo è \verb|http://lupus.uz.sns.it/|.
Le votazioni durante il giorno e le attivazioni di abilità e poteri avvengono tutte tramite l'interfaccia web. Le informazioni sulle votazioni del giorno e sugli avvenimenti della notte compaiono a loro volta sull'interfaccia web (le informazioni pubbliche possono essere visualizzate senza bisogno di autenticazione, mentre quelle private sono accessibili solo dopo il login). In caso di impossibilità di accedere a internet, i giocatori possono eccezionalmente contattare i GM per comunicare loro le proprie intenzioni di voto e/o il modo in cui desiderano utilizzare il proprio potere speciale.

\subsection{Restrizioni}

Nel rispetto dello spirito del gioco, è opportuno introdurre alcune restrizioni.

Le comunicazioni fra GM e giocatori (siano esse verbali, scritte, digitali, telepatiche, o in qualsiasi altra forma) sono inviolabili. Lo stesso vale per l'utilizzo dell'area privata dell'interfaccia web, le lettere cartacee consegnate all'inizio della partita, l'account email ufficiale, e l'account del forum che essi designano come ufficiale.
Con \emph{inviolabili} si intende che tali mezzi non possono essere letti, mostrati, registrati, né utilizzati in qualsiasi modo durante discussioni pubbliche o private con parti terze o in modi che parimenti violino lo spirito del gioco.

Gli altri mezzi di comunicazione dei giocatori, quali i loro cellulari, personal computer, o account di messaggistica di sorta, sono considerati privati.
Con \emph{privati} si intende che i giocatori non possono essere incoraggiati, minacciati, o costretti in qualsiasi modo a mostrare o concedere l'accesso a tali mezzi a parti terze, o utilizzarli in modi che parimenti violino lo spirito del gioco. Gli account anonimi del forum sono parimenti considerati privati, e le restrizioni sopra citate si applicano anche al dimostrare la proprietà di un account o mostrarne i contenuti.

In sintesi, è vietato cercare di ottenere informazioni su di un giocatore violando o cercando di violare uno qualsiasi dei mezzi citati, e più in generale non è consentito effettuare metagioco sulle comunicazioni private dei giocatori.

L'utilizzo di qualsiasi forma di crittografia non è consentito. Sono parimenti vietate altre forme di controllo o registrazione delle azioni dei giocatori in modi che violino lo spirito del gioco.

Ci possono essere altri comportamenti che non ricadono in quelli descritti in precedenza ma che comunque violano lo spirito del gioco. In caso di dubbi, chiedete preventivamente ai GM.

In ciascuno di questi casi i GM hanno il diritto di squalificare un giocatore o un'intera fazione, e il loro giudizio è inappellabile.

\subsection{Comunicazioni tra i giocatori}

I giocatori possono comunicare tra di loro con qualsiasi mezzo. Possono dirsi qualsiasi cosa.

All'inizio della partita, verrà pubblicata una lista degli indirizzi e-mail dei partecipanti. Ai giocatori è vietato usare un indirizzo e-mail di quella lista diverso dal proprio: in particolare, è vietato mandare e-mail da quell'account se vi si ha accesso in qualche modo; è vietato usare qualsiasi tipo di abilità informatica per avere accesso a quella casella e-mail; è vietato inviare e-mail usando quell'indirizzo pur non avendone accesso.

La violazione di queste regole può portare alla squalifica del giocatore o, nei casi più gravi, dell'intera fazione. La violazione delle leggi dello Stato in cui vi trovate è sconsigliata dai GM, ma non è proibita a priori dal regolamento. Questo non vi darà comunque alcun tipo di immunità, civile o penale.

È lecito (anzi, spesso consigliato) imbrogliare gli altri giocatori, mentire, usare indirizzi e-mail fasulli o anonimi per comunicare con gli altri giocatori o per cercare di ingannarli, e qualsiasi altro mezzo vi venga in mente per vincere la partita, purché sia nel rispetto delle regole di cui sopra.

È anche lecito spiare il comportamento di altri giocatori, sia di persona che con mezzi informatici, con la seguente eccezione: è vietato spiare un giocatore all'interno della sua camera o abitazione personale. In particolare è vietato eseguire monitoraggio delle reti dei collegi, allo scopo per esempio di determinare in che istanti i giocatori sono connessi ad Internet.

Anche i morti possono parlare con gli altri membri del villaggio, e continuano a giocare. Si incoraggiano i giocatori a non comunicare solo via e-mail ma anche e soprattutto di persona.

Naturalmente le regole sopra indicate si possono prestare a diverse interpretazioni. I giocatori sono invitati a chiedere ai GM conferma della liceità di comportamenti che potrebbero essere valutati \emph{border line}.
\pagebreak

\section{Gioco}

\subsection{Fazioni e condizioni di vittoria}

I giocatori sono divisi in due fazioni: la Fazione dei Popolani e la Fazione dei Lupi. I componenti della Fazione dei Lupi si conoscono fra di loro sin dall'inizio della partita. 

Una fazione vince se, subito dopo l'alba oppure subito dopo la votazione del tramonto, tutti i personaggi vivi appartengono a quella fazione. Inoltre, la Fazione dei Lupi perde immediatamente se muoiono tutti i Lupi: quando questa eventualità accade, tutti i membri della fazione dei Lupi vengono esiliati e pertanto la fazione dei Popolani viene dichiarata vincitrice.

Se una fazione ha chiaramente vinto prima che le condizioni di vittoria precedenti siano rispettate, i GM possono porre fine alla partita in anticipo (ma non devono farlo per forza).

\subsection{Inizio del gioco}

All'inizio della partita ciascun giocatore riceve le informazioni necessarie per partecipare, tra cui il ruolo ed eventualmente le identità di alcuni altri giocatori. La partita comincia con la notte.


\subsection{Svolgimento del giorno e votazioni}

L'inizio del giorno è annunciato sul sito web, insieme a tutte le informazioni ottenute da ciascun giocatore in seguito agli avvenimenti della notte appena trascorsa. Ogni giorno, il villaggio convoca un'assemblea per stabilire chi condannare a morte sul rogo. I personaggi vivi hanno il diritto di votare entro le ore 22:00 un abitante del villaggio da condannare a morte. Dopo le ore 22:00, viene pubblicata la lista delle persone votate, insieme al numero di preferenze ricevute da ciascun personaggio.

Se un personaggio riceve strettamente più del 50\% dei voti disponibili, viene condannato e muore. Altrimenti, nessuno viene condannato.

\subsection{Svolgimento della notte e uso di abilità e poteri}

La notte inizia nel momento in cui viene pubblicato l'esito della votazione del giorno, e termina alle 9:00 della mattina seguente. Con \emph{abilità} si intende la capacità di compiere un'azione speciale da parte di un personaggio vivo; con \emph{potere} si intende la stessa cosa, ma da parte di un personaggio morto. Durante la notte, i personaggi vivi che hanno delle abilità hanno la facoltà di attivarle, ed i personaggi morti che hanno dei poteri possono fare lo stesso.

Alcuni personaggi hanno un'abilità o un potere attivabile ogni due notti: si intende che un personaggio che utilizza una tale abilità o potere con successo non può utilizzare alcuna abilità o potere la notte successiva. Alcune abilità non possono essere utilizzate la prima notte. Nessun personaggio può utilizzare la propria abilità o potere su sé stesso, eccetto dove diversamente specificato.

%Un personaggio che usa la propria abilità o il proprio potere su un altro personaggio, 

\subsection{Movimenti}

Durante la notte, se un personaggio (vivo) utilizza la propria abilità su un altro personaggio, genera un proprio movimento verso la casa del personaggio su cui tale abilità viene utilizzata. L'uso di un potere (da parte di un personaggio morto), invece, non genera alcun movimento.

Alcuni ruoli possono ottenere informazioni sui movimenti, cancellarli, o modificarli. Un movimento generato in modo diverso da quello sopra descritto viene denominato fittizio; i movimenti fittizi sono movimenti a tutti gli effetti, salvo diversamente specificato.

Come specificato nella sezione \ref{fallimento}, il fallimento nell'utilizzo di un'abilità genera comunque un movimento, a meno che non sia dovuto ad un Sequestratore.

\subsection{Morte}

Molti dei personaggi non resteranno vivi sino alla fine della partita. Un personaggio può morire in diversi modi: condannato al rogo, sbranato dai Lupi, oppure ucciso da un Cacciatore o da un Assassino.

I personaggi morti possono comunque continuare a giocare e aiutare la propria fazione. Se una fazione vince, tutti i componenti di quella fazione vincono, indipendentemente dal fatto che siano vivi o morti.

Un personaggio morto non può votare, né usare la propria eventuale abilità; se ha un potere, può invece utilizzarlo. Può anche interagire con gli altri personaggi in tutti i modi non vietati dal regolamento, per esempio rivelando informazioni in suo possesso. Può inoltre essere riportato in vita dal Messia o divenire uno Spettro, se si verificano le opportune condizioni.

Solo i personaggi appartenenti alla Fazione dei Lupi possono avere un potere, come spiegato nella Sezione~\ref{incantesimi}.

\subsection{Composizione del villaggio}
 
La composizione del villaggio è scelta dai GM e non è nota ai partecipanti. I ruoli sono assegnati casualmente ai giocatori.

I GM sono liberi, all'inizio della partita, di dare alcune indicazioni sulla composizione. Ad esempio, potrebbero dire: ``Su $40$ giocatori presenti, i Lupi sono più di $4$, i componenti della Fazione dei Lupi non sono più di $11$ e ci sono almeno un Veggente ed un Investigatore''. Ciascuna di queste indicazioni può essere rivelata pubblicamente, oppure soltanto ad alcuni giocatori.

\subsection{Ruoli}
\label{ruoli}

% Non è necessario che tutti i giocatori sappiano cosa fa ogni ruolo; se siete particolarmente pigri, è sufficiente che sappiate quale potere speciale avete voi (vi verrà ricordato nella lettera in cui vi si assegnerà il ruolo). Tuttavia, per giocare in modo più efficace, è certamente onveniente sapere anche cosa possono fare gli altri.

I seguenti sono i ruoli che \emph{possono} comparire nel gioco, divisi per fazione. Per ciascuno di essi è indicata tra parentesi l'aura, che può essere bianca oppure nera, ed è eventualmente indicato l'attributo ``mistico''.

\subsection*{Fazione dei Popolani}

\begin{itemize}
	        
	\item {\bf Apprendista} (aura bianca). Prima della prima notte e alla fine di ogni alba, all'Apprendista viene assegnata un'abilità di un altro ruolo da usare per la notte successiva. Il ruolo per ciascuna notte è deciso segretamente dai GM all'inizio della partita. Dopo un certo numero di notti, l'Apprendista perde l'ultima abilità ricevuta e smette di riceverne altre.
	
	\item {\bf Cacciatore} (aura nera). Ogni notte dopo la prima, il Cacciatore può scegliere un personaggio vivo e sparargli. Tale personaggio muore.

    Dopo che utilizza la sua abilità con successo, il Cacciatore non può più utilizzare abilità per il resto della partita.
	
	\item {\bf Contadino} (aura bianca). Il Contadino non ha alcuna abilità.

	\item {\bf Divinatore} (aura bianca, mistico). Ogni due notti, il Divinatore può scegliere un personaggio vivo ed un ruolo, ed effettuare una divinazione su di lui. Scopre se il personaggio scelto ha tale ruolo.
	
	All'inizio della partita, il Divinatore è a conoscenza di quattro proposizioni nella forma ``Il personaggio X ha il ruolo Y''. Di queste quattro frasi almeno una è vera ed almeno una è falsa.

	\item {\bf Esorcista} (aura bianca, mistico). Ogni notte, l'Esorcista può scegliere un personaggio, vivo o morto, e benedire la sua casa. Per quella notte i poteri utilizzati su tale personaggio non hanno effetto. Le abilità funzionano invece normalmente.

        Inoltre, l'Esorcista scopre se ha annullato l'effetto di almeno un potere. In tal caso, la notte successiva non può agire.

 
	\item {\bf Espansivo} (aura bianca). Ogni due notti, l'Espansivo può scegliere un personaggio vivo e andare a trovarlo. Tale personaggio scopre l'identità dell'Espansivo.

	\item {\bf Guardia del corpo} (aura bianca). Ogni notte, la Guardia del corpo può scegliere un personaggio vivo e proteggerlo. Per quella notte, se uno o più Lupi tentano di uccidere il personaggio scelto dalla Guardia del corpo, la loro abilità fallisce.

	Inoltre, la Guardia del corpo scopre anche quanti altri personaggi hanno utilizzato la propria abilità su tale personaggio (ma non quali).
 
	\item {\bf Investigatore} (aura bianca). Ogni due notti, l'Investigatore può scegliere un personaggio morto e indagare su di esso. Scopre il ruolo di tale personaggio.

	\item {\bf Mago} (aura bianca, mistico). Ogni notte, il Mago può scegliere un personaggio, vivo o morto, e percepirne la magia. Scopre se tale personaggio è un mistico oppure no.
 
	\item {\bf Messia} (aura bianca, mistico). Ogni notte, il Messia può scegliere un personaggio morto e resuscitarlo. Tale personaggio ritornerà in vita all'inizio del giorno seguente. Se il personaggio scelto è uno Spettro, l'abilità del Messia fallisce.

    Dopo che utilizza la sua abilità con successo, il Messia non può più utilizzare abilità per il resto della partita.

	\item{\bf Sciamano} (aura nera, mistico). Ogni due notti, lo Sciamano può scegliere un personaggio morto ed effettuare su di lui un rito arcano. Se il personaggio scelto è uno Spettro che sta cercando di usare un potere, quel potere non ha effetto.
	
	\item {\bf Stalker} (aura bianca). Ogni due notti, lo Stalker può scegliere un personaggio vivo e pedinarlo. Scopre un eventuale movimento generato da tale personaggio.
	
	\item {\bf Trasformista} (aura nera). Ogni notte, il Trasformista può scegliere un personaggio morto e tentare di prenderne il ruolo. Se il personaggio scelto ha un ruolo appartenente alla Fazione dei Popolani, il Trasformista lo scopre ed ottiene tale ruolo (includendo aura e misticità, e perdendo definitivamente ruolo e abilità precedenti); altrimenti, l'abilità del Trasformista fallisce.
	
	\item {\bf Veggente} (aura bianca, mistico). Ogni notte, il Veggente può scegliere un personaggio vivo e scrutarlo nella sua sfera di cristallo. Scopre il colore dell'aura di tale personaggio.

	\item {\bf Voyeur} (aura bianca). Ogni due notti, il Voyeur può scegliere un personaggio vivo e spiarlo. Scopre tutti i movimenti diretti verso la casa di tale personaggio.
	
\end{itemize}


\subsection*{Fazione dei Lupi}

\begin{itemize}
	
	\item {\bf Lupo} (aura nera). Ogni notte, eccetto la prima, ciascun Lupo può scegliere un personaggio vivo e tentare di ucciderlo. Se tutti i Lupi che decidono di usare la propria abilità scelgono lo stesso personaggio, questi muore. Se almeno due Lupi indicano personaggi diversi, la loro abilità fallisce.
	
	\item {\bf Negromante} (aura bianca, mistico). Ogni notte, ciascun Negromante può scegliere un personaggio morto ed un Incantesimo non già attivo.

        Se il personaggio scelto è uno Spettro, l'Incantesimo selezionato viene attivato sullo Spettro scelto. Un eventuale Incantesimo precedentemente attivo su tale Spettro viene sostituito dall'Incantesimo selezionato.
        Il Negromante può anche disattivare un Incantesimo attivo su tale Spettro senza attivarne uno nuovo.

        Se il personaggio scelto non è uno Spettro e appartiene alla fazione dei Popolani, questi comincia a giocare per la fazione dei Lupi, diventa uno Spettro e l'Incantesimo selezionato viene attivato su di esso. In tal caso però il Negromante muore.
	
	\item {\bf Assassino} (aura nera). Ogni due notti, eccetto la prima, l'Assassino può scegliere un personaggio vivo e puntare il suo fucile di precisione verso casa sua. Uccide uno degli altri personaggi vivi il cui movimento è diretto verso la casa del personaggio scelto dall'Assassino, selezionato in modo casuale. Se il movimento selezionato è fittizio, nessuno viene ucciso.

	\item {\bf Diavolo} (aura nera, mistico). Ogni notte, il Diavolo può scegliere un personaggio vivo e un sottoinsieme dei ruoli, e scrutare fra le fiamme. Scopre se il ruolo del personaggio scelto appartiene a tale sottoinsieme.
 
	\item {\bf Fattucchiera} (aura bianca, mistico). Ogni notte la Fattucchiera può scegliere un personaggio, vivo o morto, ed un ruolo, e colpire il personaggio scelto con una fattura. Per quella notte, ruolo, aura, e misticità di tale personaggio risultano come quelle del ruolo scelto. Eventuali abilità o poteri di tale personaggio non sono influenzati.

	\item {\bf Sequestratore} (aura nera). Ogni notte, il Sequestratore può scegliere un personaggio vivo e rapirlo. Per quella notte, se il personaggio scelto utilizza la propria abilità, tale abilità fallisce. Il Sequestratore scopre se tale personaggio stava generando un movimento (ma non scopre dove era diretto), ed il tal caso cancella il movimento generato.
	
	\item {\bf Stregone} (aura nera, mistico). Ogni notte, lo Stregone può scegliere un personaggio, vivo o morto, e lanciare un sortilegio sulla sua casa. Per quella notte, se un qualsiasi altro personaggio vivo utilizza la propria abilità sul personaggio scelto dallo Stregone, tale abilità fallisce.
 
	\item {\bf Fantasma} (aura nera). Ogni notte, il Fantasma può scegliere un Incantesimo (diverso da quello della Vita), ed effettuare un rituale per legarsi a tale incantesimo.

        Se il Fantasma muore, diventa immediatamente uno Spettro. Se ha effettuato rituali dall' inizio della partita, ottiene l'Incantesimo scelto nell'ultimo rituale, se questo è disponibile; altrimenti, viene risvegliato come Spettro senza alcun Incantesimo attivo.

	\item {\bf Spettro}. Lo Spettro non è un ruolo, ma uno status che può o meno essere assegnato ad un personaggio morto. In particolare, all'inizio della partita non vi è alcuno Spettro.
	
	Ci sono tre modi in cui un personaggio morto può divenire uno Spettro.
	
	\begin{itemize}
		\item Quando un personaggio appartenente alla Fazione dei Popolani muore, c'è la possibilità che questi divenga automaticamente uno Spettro; ciò avviene o meno a seconda di quanti altri Popolani sono morti fino a quel momento, in un modo segretamente deciso dai GM prima dell'inizio della partita. In questo caso, nessun Incantesimo è attivo su tale Spettro; i Negromanti potranno attivarne uno a partire dalla notte successiva.
		
		\item Un personaggio morto appartenente alla Fazione dei Popolani può divenire uno Spettro successivamente, in seguito all'uso dell'abilità di un Negromante. In questo caso un Incantesimo viene immediatamente attivato su tale Spettro, come riportato nella descrizione del Negromante.
		
		\item Quando un Fantasma muore, questi diviene automaticamente uno Spettro. In questo caso un Incantesimo potrebbe essere immediatamente attivato su tale Spettro, nei casi indicati nella sezione \ref{ruoli}; i Negromanti potranno modificarlo o attivarne uno a partire dalla notte successiva.
	\end{itemize}
 	
 	Nel momento in cui un personaggio diviene uno Spettro, mantiene il suo ruolo (includendo aura e misticità). Se apparteneva alla Fazione dei Popolani (ovvero, se non era un Fantasma), inizia a giocare per la Fazione dei Lupi; in tal caso gli viene comunicata l'identità dei membri della Fazione dei Lupi, e ai membri della Fazione dei Lupi viene comunicata la sua identità.
 	
	Quando un personaggio riceve lo status di Spettro, questo rimane morto, a meno che su di esso non venga contemporaneamente assegnato l'Incantesimo della Vita.
	In particolare uno Spettro morto non può essere ucciso, e il villaggio non riceve alcuna informazione sull'assegnazione dello status di Spettro.
	
	Un personaggio appartenente alla Fazione dei Popolani non può diventare uno Spettro durante la stessa notte in cui la Fazione dei Lupi viene sconfitta.

    Uno Spettro che non abbia un Incantesimo attivo ha un potere: ogni notte può agire su se stesso e selezionare un Incantesimo. Allo Spettro viene applicato l'Incantesimo selezionato, ma lo Spettro non potrà usare alcun potere la notte successiva.
	
\end{itemize}


\subsubsection*{Incantesimi}\label{incantesimi}

La maggior parte degli Incantesimi assegna un potere ad uno Spettro. Se non diversamente riportato, la descrizione dell'Incantesimo coincide con quella del potere ottenuto dallo Spettro.

Ciascun Incantesimo può essere attivo su al più uno Spettro alla volta.

\begin{itemize}

	\item {\bf Assoluzione}. Ogni due notti, lo Spettro può scegliere un personaggio vivo, e infondere un'aura di innocenza intorno a lui. Durante il giorno successivo, tutti i voti diretti verso tale personaggio risultano invece voti nulli.
	
	\item {\bf Confusione}. Ogni notte, lo Spettro può scegliere un personaggio, vivo o morto, e generare un'aura di confusione attorno al personaggio scelto. Per quella notte, il colore dell'aura di tale personaggio viene invertito.
	
	\item {\bf Illusione}. Ogni due notti, lo Spettro può scegliere un personaggio vivo, generarne un'illusione, e dirigerla verso la casa di un altro personaggio, vivo o morto. Uno Stalker vedrà questo movimento invece di quello originale. L'illusione non sarà percepita da nessuna altra abilità o potere.
	 
	\item {\bf Occultamento}. Ogni notte, lo Spettro può scegliere un personaggio, vivo o morto, e creare attorno alla sua casa una fittissima nebbia magica. Per quella notte, se un qualsiasi personaggio, eccetto l'Esorcista, utilizza la propria abilità o il proprio potere sul personaggio scelto dallo Spettro, tale abilità o potere fallisce.
	
	\item {\bf Ombra}. Ogni notte, lo Spettro può scegliere un personaggio, vivo o morto, e generare presso la sua casa un'illusione di un altro personaggio vivo. Un Voyeur che visita tale casa, vedrà anche il personaggio selezionato.

	\item {\bf Telepatia}. Ogni due notti, lo Spettro può scegliere un personaggio vivo e un ruolo, e tentarne leggerne la mente. Scopre se il personaggio scelto ha tale ruolo. Inoltre, in caso di responso positivo, ottiene tutte le informazioni che tale personaggio riceve durante l'alba successiva. Ai fini di questo potere, conta il ruolo reale e non eventuali modifiche (date per esempio da una Fattucchiera).
	
	\item {\bf Tormento}. Ogni due notti, lo Spettro può scegliere un personaggio vivo e tormentarne la mente. All'alba successiva, tutte le informazioni scoperte da tale personaggio mediante l'utilizzo del suo potere saranno sostituite da un messaggio di scherno o una sonora pernacchia.

	\item {\bf Visione}. Ogni due notti, lo Spettro può scegliere un personaggio, vivo o morto, e spiarne la casa. Scopre l'elenco dei ruoli (con eventuali molteplicità) di tutte le persone che hanno visitato tale casa (tuttavia, non scopre le identità di tali persone).

	\item {\bf Vita}. Lo Spettro non ottiene alcun potere. Invece, durante l'alba successiva, ritorna in vita mantenendo ruolo ed abilità, ma continuando a giocare con la Fazione dei Lupi. Tuttavia, egli non può utilizzare abilità che sono utilizzabili una sola volta a partita (Cacciatore, Messia, Trasformista).
	L'Incantesimo della Vita viene disattivato se si verifica una condizione per cui tale personaggio dovrebbe morire (per esempio perché bruciato sul rogo), in aggiunta alle normali condizioni di disattivazione.
	Se l'Incantesimo della Vita viene disattivato (per qualsiasi ragione), lo Spettro muore.
 
\end{itemize}

Quando un Incantesimo viene attivato o disattivato su di uno Spettro, questo riceve una notifica.

\subsection{Utilizzo di abilità, poteri, e fallimento}
\label{fallimento}

\paragraph{Utilizzo di abilità e poteri} 

Se un personaggio con un'abilità o un potere attivabile ogni due notti (oppure una volta a partita) lo utilizza con successo, non potrà farlo nella notte successiva (o nel resto della partita).

I personaggi che ottengono informazioni di qualsiasi tipo su altri personaggi, le ottengono riguardo alla condizione in cui questi si trovavano al termine del giorno precedente, anche se tali informazioni cambiano nel corso della notte.
Ad esempio, se un Trasformista viene visitato dal Veggente durante la notte in cui utilizza la proria abilità con successo su un Mago, il Veggente lo vedrà comunque con aura nera, perché la trasformazione avviene al termine della notte.

\paragraph{Fallimento} Ci sono alcuni modi di utilizzare la propria abilità o il proprio potere che sono a priori non consentiti, e pertanto non risultano nemmeno tra le scelte disponibili sull'interfaccia web. Ad esempio, l'Investigatore non può scegliere di usare la propria abilità su un personaggio vivo; un Negromante morto non può provare ad attivare il proprio potere se l'ha già utilizzato con successo in una notte precedente.

Anche escludendo i casi menzionati sopra, non sempre un personaggio riesce ad usare la propria abilità o potere con successo. Ci sono essenzialmente tre possibili ragioni per cui questo può accadere.

\begin{itemize}
	\item Blocco da parte di un altro personaggio. Ad esempio, si può essere rapiti da un Sequestratore, bloccati da un Esorcista o dallo Spettro su cui è attivo l'Incantesimo dell'Occultamento.
	\item Restrizioni all'utilizzo di abilità o potere. Ad esempio, un Lupo non riesce a uccidere se nella stessa notte un altro Lupo cerca di assassinare un personaggio diverso; l'abilità del Trasformista non può essere usata su un Lupo.
	\item Contraddizioni nell'utilizzo di abilità o poteri. Può succedere che non ci sia alcun modo logico di risolvere le azioni dei personaggi nel corso di una notte (vedi le FAQ per una discussione più approfondita). Questo accade per esempio se tre Sequestratori si sequestrano in modo ciclico.
\end{itemize}

Nel caso in cui un personaggio cerchi di utilizzare la propria abilità o potere ma l'utilizzo non ha successo, il personaggio riceve una notifica di fallimento. Tale notifica non precisa la ragione del fallimento. Un utilizzo a vuoto (ad esempio una Guardia che utilizzi la propria abilità su un personaggio diverso da quello scelto dai Lupi) non costituisce un fallimento.

Un personaggio vivo che cerchi di utilizzare la propria abilità e fallisca genera comunque un movimento, a meno che il fallimento sia dovuto a un Sequestratore (come riportato nella descrizione del Sequestratore, Sezione \ref{ruoli}).

\end{document}
