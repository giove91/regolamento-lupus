\documentclass[a4paper,10pt]{article}


\usepackage[utf8]{inputenc}
\usepackage[italian]{babel}
\usepackage[T1]{fontenc}
\usepackage[dvips]{graphicx}
\usepackage{amsmath}
\usepackage{amsthm}
\usepackage{fancyhdr}
\usepackage{amsfonts}
\usepackage{amssymb}

\topmargin -1cm
\oddsidemargin -0.5cm
%\evensidemargin	-1cm
\textwidth 17cm


\newcommand{\smallspace}{\vskip0.3cm}

% Title Page
\title{Lupus in tempo reale\\ Regolamento della quarta edizione}
\author{Giove}

\begin{document}
\maketitle


\section{Introduzione}

...

\section{Partecipanti}

Possono partecipare:
\begin{itemize}
 \item gli studenti della Scuola;
 \item altre persone che frequentano spesso gli ambienti della Scuola, in particolare il collegio Carducci e la mensa, dal lunedì al venerdì.
\end{itemize}

Non è necessario saper già giocare a Lupus in Tabula, è sufficiente farsi spiegare le regole di base da qualcuno che sa già giocare.

Tutti i partecipanti si impegnano a farsi vedere un po' in giro negli ambienti della Scuola nei giorni della partita. Nessuno vi chiede di non partire venerdì mattina per tornare a casa se non avete lezione, ma se vivete da reclusi in camera vostra e non mangiate mai in mensa potrebbe essere il caso di non giocare.


\section{Quando si giocherà}

...


\section{Modifiche rispetto al regolamento della terza edizione}

...


\section{Giorni e notti di gioco}

I giorni di gioco si estendono dalle 8:00 (circa) alle 22:00 di lunedì, martedì, mercoledì e giovedì, e dalle 8:00 (circa) di venerdì alle 22:00 di domenica.
Le notti di gioco si estendono dalle 22:00 (circa) alle 8:00 di domenica, lunedì, martedì, mercoledì e giovedì.


\section{Svolgimento del giorno e votazioni}

L'inizio del giorno è annunciato sul sito web, insieme a tutte le informazioni ottenute da ciascun giocatore in seguito agli avvenimenti della notte appena trascorsa.
I personaggi vivi hanno il diritto di votare entro le ore 22 una persona del villaggio da uccidere. Dopo le ore 22, viene pubblicata la lista dei votanti insieme alla preferenza espressa da ciascuna persona. Il voto è ritenuto valido se almeno il 50\% dei vivi ha votato; in caso contrario, nessuno muore. La persona che ha ricevuto più voti di tutti muore. Se due o più persone hanno ricevuto lo stesso numero di voti, che è superiore a quello dei voti ricevuti da chiunque altro, muore quella che è stata votata dal Sindaco (vedi Sezione XYZ); se nessuna delle due è stata votata dal Sindaco, ne muore una a caso.

TODO: elezione del Sindaco (?)


\section{Svolgimento della notte e poteri speciali}

La notte inizia nel momento in cui viene pubblicato l'esito della votazione del giorno, e termina alle 8:00 della mattina seguente.
Durante la notte, i personaggi che hanno poteri speciali hanno la facoltà di attivarli.
Alcuni personaggi hanno un potere speciale attivabile ogni due notti: si intende che tale potere può essere usato se e solo se non è stato usato durante la notte precedente (in altre parole, l'unica restrizione è data dal non poterlo usare in due notti consecutive).


\section{Comunicazioni GM-giocatori, votazioni, attivazioni dei poteri}

All'inizio della partita, ciascun giocatore riceve le credenziali per accedere all'interfaccia web XYZ (verosimilmente sarà http://uz.sns.it/lupus/).
Le votazioni durante il giorno e le attivazioni dei poteri notturni avvengono tutte tramite l'interfaccia web. Le informazioni sulle votazioni del giorno e sugli avvenimenti della notte compaiono a loro volta sull'interfaccia web (le informazioni pubbliche possono essere visualizzate senza bisogno di autenticazione, mentre quelle private sono accessibili solo dopo il login).
In caso di impossibilità di accedere a internet, i giocatori possono eccezionalmente contattare il GM (ad esempio via SMS) per comunicargli le proprie intenzioni di voto e/o il modo in cui desiderano utilizzare il proprio potere notturno.

Le comunicazioni tra GM e giocatori sono inviolabili: è vietato origliare le conversazioni del GM con i giocatori, fare pressione sul GM in qualsiasi modo, inviargli e-mail con intenti truffaldini, rifiutarsi di rispondere sinceramente alle sue domande sui messaggi ricevuti e qualsiasi azione che ricordi anche solo vagamente le precedenti.
È inoltre vietato spiare altri giocatori mentre accedono all'area riservata dell'interfaccia web, rubare o hackare account altrui (anche approfittando di eventuali distrazioni), mostrare o dare accesso al proprio account ad altri giocatori, o cercare di violare la sicurezza dell'interfaccia web.
Le medesime regole si applicano alla lettera personale che viene data a ciascun giocatore all'inizio della partita (contenente il ruolo assegnatogli).
La violazione di queste regole può portare alla squalifica del giocatore o, nei casi più gravi, dell'intera fazione.


\section{Comunicazioni giocatori-giocatori}

I giocatori possono comunicare tra di loro con qualsiasi mezzo. Possono dirsi qualsiasi cosa.
L’unica cosa illecita è utilizzare truffaldinamente le e-mail per capire i ruoli delle altre persone: mostrare ad un altro giocatore le informazioni che si sono ricevute dal GM (via interfaccia web, via e-mail o con qualsiasi altro mezzo) è vietato; pressioni come ``fammi vedere la tua casella di posta da lontano, per vedere quante mail ricevi sul gioco'' sono vietate; mandare email che sembrano provenire da un indirizzo di posta diverso dal proprio per imbrogliare il destinatario è vietato; qualunque cosa che assomigli alle precedenti è vietata. La violazione di queste regole può portare alla squalifica del giocatore o, nei casi più gravi, dell'intera fazione.

Anche i morti possono parlare con gli altri membri del villaggio, e continuano a giocare. Si incoraggiano i giocatori a non comunicare solo via e-mail ma anche di persona.



\section{Fazioni e condizioni di vittoria}

I giocatori sono divisi in tre fazioni: i Buoni, i Cattivi e i Non-morti.
Una fazione vince se, subito dopo l'alba oppure subito dopo la votazione del tramonto, tutti i personaggi vivi appartengono a quella fazione.

Se una fazione ha chiaramente vinto prima che le condizioni di vittoria precedenti siano rispettate (ad esempio, se sono rimasti 5 Lupi, altri 5 Cattivi e 7 Contadini, ed i Lupi ed i Cattivi si conoscono), il GM può porre fine alla partita in anticipo (ma non deve farlo per forza).


\section{Giocare dopo la disfatta}

...


\section{Suicidio}

Ogni personaggio ha un potere attivabile di notte: il suicidio.
Se un personaggio si suicida, viene trovato morto la mattina seguente. Quando qualcuno muore di notte, il villaggio sa solo che è morto, non come ciò sia accaduto.
Questa regola è stata introdotta principalmente per permettere a chi decide di uscire dal gioco di farlo quando desidera (ma, ovviamente, ci si può suicidare per fare un bluff di qualche tipo, se si vuole).


\section{Composizione del villaggio}

La composizione del villaggio è scelta dal GM e non è nota ai partecipanti. I ruoli sono assegnati casualmente ai giocatori.
Tutte e tre le fazioni sono sicuramente presenti. All'inizio della partita, i Cattivi e i Non-morti sono in totale circa $1/4$ dei giocatori, ma il loro numero non è noto con precisione ai giocatori.

Il GM è libero, all'inizio della partita, di dare alcune indicazioni sulla composizione (ad esempio, potrebbe dire: Su $40$ giocatori presenti, i lupi sono più di $5$, i Cattivi non sono più di $11$ e ci sono almeno un Veggente ed un Medium). La composizione della fazione del vampiro viene sicuramente comunicata a tutti (?).


\section{Ruoli}

Non è necessario che tutti i giocatori sappiano cosa fa ogni ruolo; se siete particolarmente pigri, è sufficiente che sappiate quale potere avete voi (vi verrà ricordato nella lettera in cui vi si assegnerà il ruolo). Tuttavia, per giocare in modo più efficace, è certamente conveniente sapere anche cosa possono fare gli altri.


I seguenti sono i ruoli che \emph{possono} comparire nel gioco.


\subsection*{Personaggi Buoni}

\begin{itemize}
 \item {\bf Contadino} (aura bianca). Il Contadino non ha alcun potere particolare.
 
 \item {\bf Veggente} (aura bianca, mistico). Ogni notte, il Veggente può scegliere un personaggio vivo per scoprirne il colore dell'aura.

 \item {\bf Guardia del corpo} (aura bianca). Ogni notte, la Guardia del corpo può scegliere un personaggio vivo (ma non sé stessa) e proteggerlo. Durante la notte, tale personaggio non potrà morire per effetto dell'attacco dei Lupi.
 
 \item {\bf Massoni} (aura bianca). I Massoni si conoscono tra loro.
 
 \item {\bf Investigatore} (aura bianca). Ogni notte, l'Investigatore può scegliere un personaggio morto per scoprirne il colore dell'aura.

 \item {\bf Messia} (aura bianca, mistico). Una sola volta in tutta la partita, il Messia può scegliere di resuscitare una persona morta. Quella persona ritornerà in vita il giorno seguente, riacquistando i suoi poteri speciali (ma non la carica di sindaco, qualora l'avesse avuta).

 \item {\bf Voyeur} (aura bianca). Ogni due notti, il Voyeur può scegliere una persona, viva o morta, di cui essere morbosamente infatuato, e può andare a spiare la sua camera. Il Voyeur scopre quali sono le persone che durante la notte sono entrate nella sua casa (ad esempio, il Veggente entra nella casa di qualcuno per scrutarne l'aura).
 Il Voyeur non può usare il suo potere su sé stesso.

 \item {\bf Stalker} (aura bianca). Ogni due notti, lo Stalker può indicare una persona viva. Durante la notte seguirà questa persona, scoprendo se è uscita di casa e dove è andata (ma non cosa è andata a fare).
 Ad esempio, se lo Stalker segue un Lupo e si tratta del Lupo che va ad uccidere, lo vedrà andare a casa della persona che i Lupi hanno scelto.
 Lo Stalker non può usare il suo potere su sé stesso.
 
 \item {\bf Mago} (aura bianca, mistico). Ogni notte, il Mago può sapere se una persona è un mistico oppure no.
 
%  \item {\bf Nipote di Mubarak} (aura nera). Ha un potere passivo: se la Nipote di Mubarak viene messa al rogo dal villaggio, non muore.

 \item {\bf Necrofilo} (aura bianca). Il Necrofilo è buono ed ha un potere attivabile di notte, una sola volta in tutta la partita.
Il Necrofilo, di notte, può indicare un morto. Se il morto ha un potere speciale attivabile una volta ogni una o due notti, il Necrofilo lo scopre ed acquista quel potere d’ora in poi. Se il morto non aveva un potere di questo tipo, il potere del Necrofilo è sprecato. Ad esempio, se il Necrofilo indica il Diavolo, che è morto, ne acquisisce il potere (ma continua ad essere buono e giocare per il villaggio). Se il Diavolo sarà resuscitato avrà ancora il suo potere, ovviamente.
Il Necrofilo non può acquisire il potere dei Lupi (uccidere) o dei Negromanti (creare Spettri).

 \item {\bf Esorcista} (aura bianca, mistico). Ogni notte, l'Esorcista può scegliere una persona (viva o morta, compreso sé stesso) e benedire la sua casa.
 Per tutta la notte, tutti gli Spettri che utilizzano il proprio potere su quella persona falliscono nel loro intento.
 Quando uno Spettro tenta di usare un potere su qualcuno schermato dall'Esorcista, viene a sapere che il suo potere non ha funzionato.
 
%  Ad esempio, se l'Esorcista indica sé stesso ed i Lupi lo attaccano, non muore. Se l'Esorcista indica un morto che viene scelto sia dal Messia che dal Necrofilo, sia il Messia che il Necrofilo perdono i loro poteri. Se l'Esorcista indica una persona scrutata dal Veggente, il Veggente non scopre nulla su quella persona. Se l'Esorcista indica una persona sequestrata dal Sequestratore, la persona non viene sequestrata e il suo potere funziona normalmente.


 \item {\bf Espansivo} (aura bianca). Ogni due notti, l'Espansivo può indicare una persona viva. Il GM rivela a questa persona l'identità dell'Espansivo.

 \item {\bf Cacciatore} (aura bianca). Ogni notte, il Cacciatore può indicare una persona. Se il Cacciatore muore durante la notte, spara e uccide la persona scelta.
 La Guradia del corpo non protegge dal potere del Cacciatore.
 Il potere del Cacciatore non si attiva come conseguenza del proprio suicidio.

 \item {\bf Divinatore} (aura bianca, mistico). Il Divinatore, all'inizio della partita, è a conoscenza di quattro proposizioni nella forma ``il ruolo di X è Y''. Tuttavia esattamente due di esse sono vere ed esattamente due sono false.
 
\end{itemize}


\subsection*{Personaggi Cattivi}

\begin{itemize}
 \item {\bf Lupi} (aura nera). I Lupi hanno il potere uccidere un membro del villaggio.
 Ogni notte, un Lupo può scegliere la persona da uccidere. Questo Lupo sarà quello che va fisicamente a casa del malcapitato per ucciderlo, ai fini dell'uso del potere del Voyeur e di altri personaggi. Se più Lupi indicano la stessa vittima, tutti vanno fisicamente a casa della vittima per ucciderla.
 Se ci sono Lupi che indicano vittime diverse, tutti i Lupi che hanno indicato una vittima escono di casa (andando a casa del personaggio designato) ma nessuno effettua l'uccisione.
 
 I Lupi non possono uccidere i Negromanti.
 
 I Lupi si conoscono tra di loro e con le Fattucchiere.

 \item {\bf Fattucchiera} (aura nera). Ogni notte la Fattucchiera può indicare una persona, viva o morta. Se tale persona viene scrutata durante la notte, l'aura viene percepita in modo invertito (ovvero se è bianca risulta nera, e se è nera risulta bianca).
 
 La Fattucchiera conosce i Lupi e le eventuali altre Fattucchiere.
 
 \item {\bf Indemoniato} (aura bianca). L'Indemoniato non ha alcun potere particolare. Conosce gli Esorcisti e gli eventuali altri Indemoniati.

 \item {\bf Diavolo} (aura nera, mistico). Ogni notte, il Diavolo può scegliere una persona viva scoprendone il ruolo che aveva all'inizio della partita, il ruolo attuale e il colore della sua aura.
 
 \item {\bf Sequestratore} (aura nera). Ogni notte, il Sequestratore può indicare una persona e rapirla. Se questa persona cerca di uscire di casa per utilizzare il proprio potere su qualcuno, fallisce. Se si tratta di un Lupo, ed è l'unico che va ad uccidere, non muore nessuno. Se si tratta di un personaggio che usa un potere speciale (ad esempio l'Espansivo, o il Messia), il potere è sprecato.
 Il Sequestratore non sa se la persona ha provato ad uscire di casa oppure no, e la persona non sa chi l'ha sequestrata (ma viene a sapere che il suo potere non ha funzionato).
 Il Sequestratore non può rapire la stessa persona per due notti consecutive. Se prova a farlo, si sposta come se avesse agito, ma il suo potere non ha effetto.

 \item {\bf Avvocato del diavolo} (aura nera). Ogni due notti, l'Avvocato del diavolo può indicare una persona (diversa da sé stesso). Se questa persona sarà quella che, il giorno dopo, avrà ricevuto più voti, non morirà (grazie ad una legge ad personam), e morirà invece la persona con il secondo numero più alto di voti.


\end{itemize}

\subsection*{Personaggi Non-morti}

\begin{itemize}
 \item {\bf Negromanti} (aura bianca, mistici). Ogni notte, ciascun Negromante può scegliere di effettuare una delle seguenti due azioni: creare uno Spettro, o ricercare un nuovo Potere.

 Per creare uno Spettro, un Negromante deve scegliere un Potere dal proprio Libro degli Spettri, e andare da un morto: quel morto diventa uno Spettro e ottiene il Potere scelto. Da quel momento in poi, quella persona gioca per la fazione dei Non-morti.
 Per ricercare un nuovo Potere, un Negromante deve scegliere uno dei Poteri che in quel momento può ricercare. Il Potere scelto viene aggiunto al proprio Libro degli Spettri.

 Ogni Potere può essere assegnato ad un solo Spettro. Vi è un numero massimo di Spettri che possono essere creati; tale numero viene fissato all'inizio della partita e comunicato ai Negromanti.

 I Lupi, la Fattucchiera e tutti i personaggi che appartengono già alla fazione dei Non-morti non possono essere risvegliati come Spettri (in particolare, questo vale per gli Spettri stessi). Se un Negromante cerca di risvegliare come Spettro uno di questi personaggi, fallisce.
 
 I Negromanti non possono essere uccisi dai Lupi.

 I Negromanti si conoscono tra loro.

 \item {\bf Medium} (aura bianca, mistico). Ogni notte, il Medium può scegliere un personaggio morto per scoprirne il colore dell'aura. Scopre inoltre se tale personaggio è diventato o meno uno Spettro.

 Il Medium conosce gli Ipnotisti e gli eventuali altri Medium.

 \item {\bf Ipnotista} (aura bianca). Ogni due notti, l'Ipnotista può scegliere una persona e assoggettarla al suo volere. Da quel momento in poi il voto di quella persona è considerato uguale a quello dell'Ipnotista, indipendentemente dall'eventuale voto espresso.
 Questo è vero anche se l'Ipnotista non vota.

 Quando l'Ipnotista muore, le persone da lui ipnotizzate ricominciano a votare secondo il proprio volere. Se tuttavia lo stesso Ipnotista viene resuscitato dal Messia, tutti coloro che si trovavano sotto il controllo dell'Ipnotista al momento della sua morte sono nuovamente assoggettati al suo volere (a meno che nel frattempo siano stati ipnotizzati da un altro Ipnotista, vedi oltre).

 Una persona può essere sotto il controllo di un solo Ipnotista per volta, e precisamente l'ultimo ad aver agito su di essa. Se più Ipnotisti agiscono sulla stessa persona durante la stessa notte, viene assoggettata da tutti in un ordine casuale, e perciò risulterà essere sotto il controllo soltanto dell'ultimo ad aver agito. Gli altri Ipnotisti non ricevono alcuna notifica di errore. 

 L'Ipnotista conosce i Medium e gli eventuali altri Ipnotisti.

 \item {\bf Fantasma} (aura bianca). Il Fantasma, finché è vivo, non ha alcun potere. Al momento della sua morte (anche se si suicida), diventa immediatamente uno Spettro e assume uno dei Poteri non ancora assegnati, scelto in modo casuale fra tutti quelli possibili. Gli viene inoltre comunicato chi sono i Negromanti. Il Fantasma non conta ai fini del numero di Spettri che possono essere creati dai Negromanti, ma i Negromanti non possono più creare Spettri con il Potere assegnato al Fantasma.
 
 \item {\bf Spettri}. Gli Spettri non sono presenti all'inizio della partita: i personaggi morti possono essere risvegliati come Spettri in seguito all'azione di un Negromante.
 Nel momento in cui un personaggio diviene uno Spettro, mantiene la sua aura ma inizia a giocare per la fazione dei Non-morti. Gli viene comunicata l'identità del Negromante che lo ha risvegliato, e ottiene un Potere che può iniziare a usare dalla notte successiva. Inoltre, perde per sempre gli eventuali poteri che possedeva.
 
 Uno Spettro rimane morto: in particolare non può essere ucciso, e il villaggio non viene informato del suo risveglio.
 La presenza di uno Spettro che sta usando il suo Potere passa inosservata agli occhi del Voyeur.
 
 Se uno Spettro viene resuscitato dal Messia, continua a giocare per la fazione dei Non-morti, può nuovamente essere ucciso dall'assemblea, perde per sempre (DA DECIDERE) il suo Potere da Spettro, non riottiene i poteri eventualmente posseduti prima di essere stato risvegliato come Spettro, e mantiene la sua vecchia aura.
 
 Gli Spettri perdono la possibilità di utilizzare il proprio Potere se muoiono tutti i Negromanti. Nel caso in cui un Negromante venga resuscitato dal Messia, gli Spettri riacquistano nuovamente la facoltà di utilizzare il proprio Potere.
\end{itemize}

\end{document}
