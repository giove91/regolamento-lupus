\documentclass[a4paper,10pt]{article}


\usepackage[utf8]{inputenc}
\usepackage[italian]{babel}
\usepackage[T1]{fontenc}
\usepackage{amsmath}
\usepackage{amsthm}
\usepackage{fancyhdr}
\usepackage{amsfonts}
\usepackage{amssymb}
\usepackage{makeidx}
\usepackage[parfill]{parskip}

% * define a `\twoidxcolumn` based on `\twocolumn`:
\def\twoidxcolumn{%
%\clearpage
\global\columnwidth\textwidth
\global\advance\columnwidth-\columnsep
\global\divide\columnwidth\tw@
\global\hsize\columnwidth
\global\linewidth\columnwidth
\global\@twocolumntrue
\global\@firstcolumntrue
\col@number \tw@
%\@ifnextchar [\@topnewpage
\@floatplacement
}


\makeatletter
\def\@wrindex#1{%
   \protected@write\@indexfile{}%
      {\string\indexentry{#1}{\theenumi}}
      \endgroup
      \@esphack}

\makeatletter
\renewenvironment{theindex}
               {\twocolumn[\section*{Indice delle parole chiave presenti nella
Sezione \ref{faq}}]%
                \@mkboth{\MakeUppercase\indexname}%
                        {\MakeUppercase\indexname}%
                \thispagestyle{plain}\parindent\z@
                \parskip\z@ \@plus .3\p@\relax
                \columnseprule \z@
                \columnsep 35\p@
                \let\item\@idxitem}
               {}
\makeatother

\makeindex
% [title=Indice dei ruoli presenti nella Sezione \ref{faq},columns=3]


\topmargin -1cm
\oddsidemargin -0.5cm
%\evensidemargin	-1cm
\textwidth 17cm


\newcommand{\smallspace}{\vskip0.3cm}

% Title Page
\title{Lupus in tempo reale\\ Regolamento della quarta edizione}
\author{Alessandro Iraci, Giovanni Paolini, Leonardo Tolomeo}

\begin{document}
\maketitle


\section{Introduzione}

\subsection{Cos'è Lupus in tempo reale?}

\emph{Lupus in tempo reale} è una variante dei più tradizionali \emph{Lupus in
Tabula} e \emph{Mafia}, caratterizzata da un maggiore coinvolgimento e da
interazioni più sofisticate tra i giocatori.
La principale differenza consiste nel ritmo del gioco: la partita si estende
infatti su diversi giorni, in modo aderente all'ambientazione.
Sono premiate dalla dinamica di gioco la logica, la capacità di bluffare (e di
smascherare i bluff altrui), la capacità di saper coordinare un’azione di
gruppo, il carisma e le velleità leaderistiche.
Una versione semplificata del regolamento, per avere un'idea di come funzioni
il gioco, si trova nella Sezione \ref{regolamento}.

\emph{Lupus in tempo reale} è stato inventato nel 2010, ed è stato giocato una
volta all'anno fino al 2012 sotto la supervisione di Francesco Guatieri come
principale Game Master.
Questo regolamento coincide in buona parte con quello della terza edizione,
scritto da Andrea Caleo.
I cambiamenti sono stati discussi e stabiliti da Alessandro Iraci, Luca
Ghidelli, Giovanni Paolini e Leonardo Tolomeo.
I Game Master che gestiranno la partita sono Giovanni Mascellani e Giovanni
Paolini.
Nel seguito di questo regolamento, verrà utilizzata la sigla ``GM'' come
abbreviazione di ``Game Master'' (o anche di ``Giovanni Mascellani'', se
preferite).

\subsection{Ambientazione}

La storia si svolge in quello che una volta era un tranquillo villaggio di
campagna. A turbare la quiete del posto è un evento quantomeno atipico, ossia
un'invasione di lupi mannari. Nascosti sotto le spoglie di normalissimi
contadini, i licantropi hanno intenzione di prendere il controllo del villaggio,
e sbranare chiunque non sia disposto ad allearsi con loro; per portare a termine
la loro missione, hanno cominciato creando quanto più scompiglio e terrore
possibile, ovvero divorando lo sfortunato sindaco della piccola cittadina.

Le autorità nazionali, venendo a sapere dell'accaduto e temendo il diffondersi
della licantropia in tutto il paese, hanno disposto un blocco forzato delle vie
d'accesso al villaggio, rendendo virtualmente impossibile a chiunque entrare o
uscire dalla frazione.

Eletto con urgenza un nuovo primo cittadino, per tentare di ristabilire la
normalità viene imposta una durissima legge marziale: ogni sera, il villaggio
convocherà un'assemblea composta da tutti gli abitanti, al termine della quale
un sospetto lupo mannaro verrà arso sul rogo. La sfiducia nei confronti degli
altri abitanti è totale, chiunque potrebbe essere un licantropo sotto mentite
spoglie. La propria vicina di casa, l'insospettabile anziano signore
dell'ortofrutta, o anche il nuovo sindaco, così normali di giorno, potrebbero
col calar della notte assumere le sembianze di un feroce lupo assetato di
sangue.

Come se non bastasse, da qualche giorno corre la voce che nel villaggio si
nasconda una setta di oscuri negromanti. Persone in grado di adoperare la magia
nera, un'abilità tenuta nascosta per paura di essere scoperti. La confusione
creatasi nel piccolo borgo, e il numero di morti in probabile aumento, potrebbe
fornire loro un'occasione irrinunciabile per cominciare a praticare la loro
arte, richiamando alla vita le vittime sotto forma di spettri e utilizzando i
poteri soprannaturali di questi ultimi per rendere il villaggio schiavo della
setta.

Fra i cittadini c'è paura, e tanta indecisione su quale sia la cosa giusta da
fare. Meglio combattere lupi mannari e negromanti e cercare di liberare il
villaggio, oppure allearsi con una di queste due fazioni nella speranza di avere
salva la vita, e magari anche di ottenere una posizione di potere? Meglio
rischiare la pelle andando ad origliare dietro la porta del proprio vicino,
cercando di capire se questi è un lupo, oppure stare tranquilli nel proprio
letto sperando di non ricevere sgradite visite? Molti abitanti hanno deciso di
attivarsi per riprendere il controllo del villaggio; altri hanno ceduto, per
paura della morte o per sete di potere, alle lusinghe dei licantropi o dei
negromanti, aiutandoli a portare a termine i loro progetti.

Ciascuno ha tutto l'interesse a tenere nascosta la propria scelta: per i
traditori, venire scoperti potrebbe significare morte sul rogo; per i villici
che hanno deciso di combattere l'invasione, esporsi troppo potrebbe voler dire
essere la prossima vittima dei lupi mannari, oppure ricevere terrificanti visite
da parte di uno spettro. Ma non sempre tacere per la propria incolumità è la
scelta giusta. In determinate circostanze rivelare preziose informazioni, anche
a costo di rischiare la vita, potrebbe essere fondamentale; veder trionfare
un'altra fazione in ogni caso significa morte.

Qualunque sia stata la propria decisione, tuttavia, la maggior parte dei
cittadini si attiva durante la notte per raccogliere informazioni o infastidire
i nemici, per esempio appostandosi fuori da una casa per cercare di capire cosa
succede all'interno, utilizzando le proprie capacità divinatorie per scoprire
l'identità degli altri, girando armati per proteggere i concittadini
dall'attacco dei licantropi o compiendo rituali per tenere lontani gli spettri;
ciascuno fa il possibile, ma anche gli alleati di lupi mannari e negromanti si
mobilitano per intralciare le indagini, diffondere false informazioni e
manipolare le votazioni durante l'assemblea.

Nel villaggio, ormai, la vita è una questione di doppio gioco; ogni passo falso
rischia di costare la pelle non solamente a sé stessi, ma anche a tutti i propri
alleati. Un delicatissimo equilibrio coinvolge le tre parti, cercare di
spezzarlo in proprio favore è l'obiettivo di tutti. Solo i più abili
doppiogiochisti riusciranno a prendere in mano la situazione e condurre la
propria squadra alla vittoria. Lo scopo da raggiungere è chiaro, ma quale sia la
fazione che riuscirà a trionfare dipende da voi...


\subsection{Partecipanti}
Possono partecipare:
\begin{itemize}
 \item gli studenti della Scuola;
 \item altre persone che frequentano spesso gli ambienti della Scuola, in
particolare il collegio Carducci e la mensa, dal lunedì al venerdì.
\end{itemize}
Tutti i partecipanti si impegnano a farsi vedere un po' in giro negli ambienti
della Scuola nei giorni della partita. Nessuno vi chiede di non partire venerdì
mattina per tornare a casa se non avete lezione, ma se vivete da reclusi in
camera vostra e non mangiate mai in mensa potrebbe essere il caso di non
giocare.


\subsection{Quando si giocherà}

L'intenzione è di cominciare la partita a metà marzo 2014. È possibile che
invece si debba iniziare a maggio.


\subsection{Principali cambiamenti rispetto alla terza edizione}

Questa sezione è pensata per dare a chi ha giocato alla terza edizione di Lupus
un'idea delle modifiche più importanti che sono state apportate. Tuttavia non
sostituisce un'attenta lettura del resto del regolamento.

\begin{itemize}
 \item La fazione dei Vampiri è stata sostituita dalla nuova fazione dei
Negromanti, che ha meccaniche di gioco molto diverse.
 \item È stato eliminato il suicidio.
 \item È stato eliminato il rituale dei fantasmi.
 \item I possibili colori dell'aura sono tornati ad essere solamente il bianco e
il nero.
 \item Molti ruoli sono stati creati, eliminati o modificati. A chi è abituato
al regolamento della terza edizione raccomandiamo di controllare con particolare
attenzione i seguenti ruoli, che hanno subito i cambiamenti più rilevanti:
Custode del cimitero, Esorcista, Investigatore, Rinnegato, Negromante, Fantasma,
Ipnotista, Medium, Spettro.
 \item Le comunicazioni di gioco (votazione, utilizzo dei poteri, resoconto
dell'alba e del tramonto) avvengono attraverso un'interfaccia web e non più via
e-mail.
\end{itemize}



\pagebreak
\section{Metagioco}

\subsection{Giorni e notti di gioco}

I giorni di gioco si estendono dalle 8:00 (circa) alle 22:00 di lunedì, martedì,
mercoledì e giovedì, e dalle 8:00 (circa) di venerdì alle 22:00 di domenica.
Le notti di gioco si estendono dalle 22:00 (circa) alle 8:00 di domenica,
lunedì, martedì, mercoledì e giovedì.



\subsection{Comunicazioni tra GM e giocatori, votazioni, attivazioni dei poteri}

All'inizio della partita, ciascun giocatore riceve le credenziali per accedere
all'interfaccia web di Lupus 4, il cui indirizzo è
\verb|http://uz.sns.it/lupus/|.
Le votazioni durante il giorno e le attivazioni dei poteri avvengono tutte
tramite l'interfaccia web. Le informazioni sulle votazioni del giorno e sugli
avvenimenti della notte compaiono a loro volta sull'interfaccia web (le
informazioni pubbliche possono essere visualizzate senza bisogno di
autenticazione, mentre quelle private sono accessibili solo dopo il login).
In caso di impossibilità di accedere a internet, i giocatori possono
eccezionalmente contattare i GM (ad esempio via SMS) per comunicare loro le proprie
intenzioni di voto e/o il modo in cui desiderano utilizzare il proprio potere
speciale.

Le comunicazioni tra GM e giocatori sono inviolabili. È vietato origliare le
conversazioni dei GM con i giocatori, fare pressione sui GM in qualsiasi modo,
cercare truffaldinamente di ottenere informazioni sulla partita in corso,
rifiutarsi di rispondere sinceramente alle loro domande ed altre azioni di
questo genere.

È inoltre vietato spiare altri giocatori mentre accedono all'area riservata
dell'interfaccia web, rubare o hackare account altrui (anche approfittando di
eventuali distrazioni), mostrare o dare accesso al proprio account ad altri
giocatori, o cercare di violare la sicurezza dell'interfaccia web.
Le medesime regole si applicano alla lettera personale che viene data a ciascun
giocatore all'inizio della partita (contenente il ruolo assegnatogli).
La violazione di queste regole può portare alla squalifica del giocatore o, nei
casi più gravi, dell'intera fazione.


\subsection{Comunicazioni tra i giocatori}

I giocatori possono comunicare tra di loro con qualsiasi mezzo. Possono dirsi
qualsiasi cosa.

È vietato mostrare ad altri giocatori la propria interfaccia web, o qualsiasi
tipo di informazione ricevuta direttamente dai GM, a prescindere dal mezzo con 
cui è stata comunicata: tramite interfaccia web, e-mail, SMS, lettera cartacea
o qualsiasi altra cosa.

All'inizio della partita, verrà pubblicata una lista degli indirizzi e-mail dei
partecipanti. Ai giocatori è vietato usare un indirizzo e-mail
di quella lista diverso dal proprio: in particolare, è vietato mandare e-mail da
quell'account se ne si ha accesso in qualche modo; è vietato
usare qualsiasi tipo di abilità informatica per avere accesso a quella casella
e-mail; è vietato inviare e-mail usando quell'indirizzo pur non
avendone accesso.

La violazione di queste regole può portare alla squalifica
del giocatore o, nei casi più gravi, dell'intera fazione.
La violazione delle leggi dello Stato in cui vi trovate è sconsigliata dai GM,
ma consentita dal regolamento. Questo non vi darà comunque alcun tipo di
immunità, civile o penale.

È lecito, anzi, spesso consigliato, imbrogliare gli altri, mentire, usare
indirizzi e-mail fasulli o anonimi per comunicare con gli altri giocatori o per
cercare di ingannarli, e qualsiasi altro mezzo vi venga in mente per vincere la
partita, purché sia coerente con i divieti di cui sopra.

Anche i morti possono parlare con gli altri membri del villaggio, e continuano a
giocare. Si incoraggiano i giocatori a non comunicare solo via e-mail ma anche
di persona.


\pagebreak
\section{Gioco}


\subsection{Regolamento in breve}
\label{regolamento}

\paragraph{Regole del gioco}

All'inizio del gioco, ad ogni giocatore viene segretamente assegnato un ruolo,
che determina anche la fazione di cui farà parte.
Ci sono tre fazioni: i Popolani, i Lupi e i Negromanti. Un giocatore vince se
vince la sua fazione, a prescindere dal fatto che sia vivo o morto
al termine della partita.

I turni di gioco sono divisi in giorni e notti. Durante il giorno, i
giocatori hanno il diritto di votare per cercare di uccidere un altro
giocatore. Al termine del giorno, salvo eccezioni, muore il giocatore con più voti.
Durante la notte i giocatori hanno diritto ad usare i loro poteri
speciali, che dipendono dal ruolo. Un elenco dei ruoli e dei rispettivi
poteri speciali è disponibile nella sezione \ref{ruoli}. In linea di
massima, comunque, i Lupi possono accordarsi fra loro e uccidere un personaggio
a loro scelta, mentre gli altri giocatori della loro fazione hanno dei poteri
utili per aiutare i Lupi; i Negromanti possono, ogni due notti, scegliere un
personaggio morto e trasformarlo in uno Spettro, che si aggiungerà alla loro
fazione e otterrà un nuovo potere speciale, mentre gli Spettri e gli altri
giocatori della loro fazione hanno poteri utili per evitare che i Negromanti vengano
scoperti oppure per manipolare le votazioni di giorno; la maggior parte dei Popolani
ha poteri adatti ad ottenere informazioni su chi siano i Lupi e i Negromanti,
ma non solo. La maggior parte dei
personaggi gioca con la fazione dei Popolani.

La partita termina quando tutti i personaggi vivi appartengono alla stessa
fazione, che viene dichiarata vincitrice.

\paragraph{Strategie}

Ogni giocatore è libero di rivelare il proprio ruolo, ma gli altri non hanno
alcun modo di sapere se stia mentendo. In generale non è un'idea saggia, perché
un giocatore della fazione dei Lupi o dei Negromanti che viene scoperto viene,
solitamente, ucciso con la votazione di giorno, dato che i Popolani sono in
maggioranza; d'altra parte, un giocatore che dichiara di essere un Popolano con
un potere particolarmente forte potrebbe indurre i Lupi o i Negromanti ad ucciderlo
durante la notte, o comunque ad impedirgli di usare il proprio potere, ma anche
rivelare di essere un Popolano con un ruolo non molto utile può essere dannoso,
perché fornisce informazioni alle fazioni avversarie.
Nonostante ciò, soprattutto per i Popolani, potrebbe essere necessario uscire allo
scoperto per rivelare le informazioni in proprio possesso, ottenute durante la notte,
oppure per permettere ai propri alleati di organizzare dei controlli su dei giocatori
sospetti.

I Popolani devono puntare a scoprire quante più informazioni possibile, organizzare
controlli a tappeto sugli altri giocatori e uccidere dei nemici con le votazioni ogni
qualvolta ne hanno l'occasione. Spesso conviene uccidere qualcuno anche sulla base di
semplici sospetti, perché non farlo, di fatto, permette ai Lupi di uccidere un Popolano
in più.

I Lupi devono organizzarsi per uccidere quanta più gente possibile, preferibilmente fra
i Popolani con dei ruoli che possono danneggiarli, senza essere scoperti. Può convenire
anche far fuori qualche alleato dei Negromanti, per non facilitare la vittoria a questi
ultimi. Ovviamente devono costruirsi una copertura credibile per far sì che le indagini
dei Popolani risultino inefficaci. Bisogna soprattuto cercare di neutralizzare il
Veggente, un Popolano in grado di distinguere i Lupi e i loro alleati dagli altri giocatori.

I Negromanti devono cercare di creare molti Spettri e rendersi virtualmente indistinguibili
dai Popolani, e cercare di prendere il controllo delle votazioni di giorno grazie agli
Ipnotisti, dei loro alleati. Anche per loro è fondamentale ottenere una copertura credibile
e resistere molto a lungo, perché partono in nettissima minoranza. Per loro, comunque,
mimetizzarsi fra i Popolani è molto più facile che per i Lupi, grazie agli Spettri e al
fatto che il Veggente non è in grado di riconoscerli.

In generale, in questo gioco è fondamentale la capacità di mentire, di crearsi una copertura
perfetta e di coordinare le azioni della propria fazione. L'interazione con gli altri
giocatori è importantissima e rende divertente il gioco.

\subsection{Fazioni e condizioni di vittoria}

I giocatori sono divisi in tre fazioni: la Fazione dei Popolani, la Fazione dei
Lupi e la Fazione dei Negromanti.
Una fazione vince se, subito dopo l'alba oppure subito dopo la votazione del
tramonto, tutti i personaggi vivi appartengono a quella fazione.
Inoltre, la Fazione dei Lupi perde immediatamente se muoiono tutti i Lupi, e la
Fazione dei Negromanti perde immediatamente se muoiono tutti i Negromanti.
Quando una di queste due eventualità accade, viene reso pubblico l'elenco dei
membri della fazione che ha appena perso (vivi e morti), e questi vengono
esiliati dal villaggio: da quel momento in poi, smettono di giocare a tutti gli
effetti.

Se una fazione ha chiaramente vinto prima che le condizioni di vittoria
precedenti siano rispettate, i GM possono porre fine alla partita in anticipo
(ma non devono farlo per forza).

\subsection{Inizio del gioco}

La partita inizia con la consegna delle lettere ai partecipanti. Il primo giorno,
non essendoci indizi per accusare nessuno, non si è tenuto alcun rogo, e così la
partita ha inizio con la notte. Il villaggio è ancora scosso, tutti sono all'erta,
e per questo motivo i Lupi, per non correre rischi, durante la prima notte non
uccidono nessuno; piuttosto approfittano della confusione per incontrarsi
segretamente e decidere come agire nelle notti successive. Diversi gruppi di
abitanti si incontrano, creandosi degli alleati fidati, e molti avviano subito le
indagini.

A livello di gioco questo significa che, con la consegna delle lettere, a ciascuno
vengono comunicate tutte le informazioni necessarie ad iniziare la partita, comprese
eventualmente le identità di altri giocatori (per esempio, ai Lupi viene comunicata
l'identità degli altri Lupi e delle Fattucchiere). Ai lupi non è data la possibilità
di uccidere, in modo che a tutti siano garantiti almeno una notte ed un giorno di
gioco. Agli altri personaggi è invece permesso usare i propri poteri, purché vi siano
bersagli validi (per esempio, non essendoci alcun morto, un eventuale Medium non
potrà agire).

\subsection{Svolgimento del giorno e votazioni}

L'inizio del giorno è annunciato sul sito web, insieme a tutte le informazioni
ottenute da ciascun giocatore in seguito agli avvenimenti della notte appena
trascorsa.
Ogni giorno, il Villaggio convoca un'assemblea per stabilire chi condannare a
morte sul rogo.
I personaggi vivi hanno il diritto di votare entro le ore 22 un abitante del
villaggio da condannare a morte. Dopo le ore 22, viene pubblicata la lista dei
votanti insieme alla preferenza espressa da ciascuna persona.
Il voto è ritenuto valido se almeno il 50\% dei vivi ha votato; in caso
contrario, nessuno viene condannato. Il personaggio che ha ricevuto più voti di
tutti muore.
In caso di parità tra due o più personaggi, fra questi viene condannato quello
eventualmente votato dal Sindaco; se nessuno di questi è stato votato dal
Sindaco, ne muore uno a caso.

Di giorno è anche possibile votare per eleggere un nuovo Sindaco, come è
spiegato più dettagliatamente nella Sezione \ref{sindaco}.


\subsection{Svolgimento della notte e uso dei poteri}

La notte inizia nel momento in cui viene pubblicato l'esito della votazione del
giorno, e termina alle 8:00 della mattina seguente.
Durante la notte, i personaggi vivi che hanno poteri speciali hanno la facoltà
di attivarli.

Alcuni personaggi hanno un potere attivabile ogni due notti: si intende che tale
potere può essere usato se e solo se non è stato usato durante la notte
precedente: in altre parole, l'unica restrizione è data dal non poterlo usare in
due notti consecutive.

\subsection{Morte}

Molti dei personaggi non resteranno vivi sino alla fine della partita. Un
personaggio può morire in diversi modi: condannato al rogo, sbranato dai Lupi,
oppure ucciso da uno Spettro o da un Cacciatore.

I personaggi morti possono continuare a giocare e aiutare la propria fazione, se
vogliono. Se una fazione vince, tutti i componenti di quella fazione vincono,
indipendentemente dal fatto che siano vivi o morti.

Un personaggio morto non può votare, né usare il proprio potere speciale, se ne
ha uno. Può interagire con gli altri personaggi in tutti i modi non vietati dal
regolamento, per esempio rivelando informazioni in suo possesso. Può inoltre
essere riportato in vita dal Messia o risvegliato come Spettro, se si verificano
le opportune condizioni.

Se un personaggio morto viene risvegliato come Spettro, resta morto, ma può
usare il suo nuovo potere soprannaturale. Il funzionamento degli Spettri è
spiegato nella Sezione \ref{spettri}, sotto la voce ``Negromanti''.


\subsection{Composizione del villaggio}
 
La composizione del villaggio è scelta dai GM e non è nota ai partecipanti. I
ruoli sono assegnati casualmente ai giocatori.

I GM sono liberi, all'inizio della partita, di dare alcune indicazioni sulla
composizione (ad esempio, potrebbe dire: ``Su $40$ giocatori presenti, i Lupi
sono più di $4$, i componenti della Fazione dei Lupi non sono più di $11$ e ci
sono almeno un Veggente ed un Investigatore'').


\subsection{Ruoli}
\label{ruoli}

Non è necessario che tutti i giocatori sappiano cosa fa ogni ruolo; se siete
particolarmente pigri, è sufficiente che sappiate quale potere speciale avete
voi (vi verrà ricordato nella lettera in cui vi si assegnerà il ruolo).
Tuttavia, per giocare in modo più efficace, è certamente conveniente sapere
anche cosa possono fare gli altri.

I seguenti sono i ruoli che \emph{possono} comparire nel gioco, divisi per
fazione.
Per ciascuno di essi è indicata tra parentesi l'aura, che può essere bianca
oppure nera, ed è eventualmente indicato l'attributo ``mistico''.


\subsection*{Fazione dei Popolani}

\begin{itemize}
 \item {\bf Contadino} (aura bianca). Il Contadino non ha alcun potere speciale.

 \item {\bf Cacciatore} (aura nera). Una sola volta durante l'arco della
partita, il Cacciatore può scegliere un personaggio vivo e puntarlo col fucile.
 Il Cacciatore uccide il personaggio scelto.
 % Ricontrollare questa cosa
 
 \item {\bf Custode del cimitero} (aura bianca). Ogni notte, il Custode del
cimitero può scegliere un personaggio morto e custodirne la tomba. Per quella
notte, se uno o più Negromanti tentano di risvegliare come Spettro il
personaggio scelto dal Custode del cimitero, il loro potere non ha effetto.
 
 Questo potere speciale non può essere usato per due notti consecutive sullo
stesso personaggio.

 \item {\bf Divinatore} (aura bianca, mistico). Il Divinatore non ha alcun
potere speciale. All'inizio della partita, il Divinatore è a conoscenza di
quattro proposizioni nella forma ``Il personaggio X ha il ruolo Y''. Esattamente
due sono vere ed esattamente due sono false.
 
 Il modo in cui queste quattro frasi sono state generate è a discrezione dei GM,
e non è noto ai giocatori.

 \item {\bf Esorcista} (aura bianca, mistico). Ogni due notti, l'Esorcista può
scegliere un personaggio, vivo o morto, e benedire la sua casa.
 Per quella notte, se uno Spettro tenta di utilizzare il proprio potere sul
personaggio scelto dall'Esorcista, il suo potere non ha effetto.
 
 L'Esorcista può usare il suo potere speciale su sé stesso.
 
 \item {\bf Espansivo} (aura bianca). Ogni due notti, l'Espansivo può scegliere
un personaggio vivo, e andare a trovarlo. Questo personaggio scopre l'identità
dell'Espansivo.

 \item {\bf Guardia del corpo} (aura bianca). Ogni notte, la Guardia del corpo
può scegliere un personaggio vivo e proteggerlo. Per quella notte, se uno o più
Lupi tentano di uccidere il personaggio scelto dalla Guardia, il loro potere non
ha effetto.
 
 \item {\bf Investigatore} (aura bianca). Ogni notte, l'Investigatore può
scegliere un personaggio morto e indagare su di esso. Scopre il colore della sua
aura.

 \item {\bf Mago} (aura bianca, mistico). Ogni notte, il Mago può scegliere un
personaggio, vivo o morto, e percepirne la magia. Scopre se quel personaggio è
un mistico oppure no.
 
 \item {\bf Massone} (aura bianca). Il Massone non ha alcun potere speciale. Il
Massone conosce gli altri Massoni.
 
 \item {\bf Messia} (aura bianca, mistico). Una sola volta durante l'arco della
partita, il Messia può scegliere un personaggio morto e resuscitarlo. Quel
personaggio ritornerà in vita all'inizio del giorno seguente, riacquistando i
suoi poteri speciali.

 \item {\bf Necrofilo} (aura bianca). Una sola volta durante l'arco della
partita, il Necrofilo può scegliere un personaggio morto e rubarne il potere. 
 Se il personaggio scelto ha un potere speciale attivabile una volta ogni una o
due notti, il Necrofilo lo scopre ed ottiene tale potere; altrimenti, il potere
del Necrofilo non ha effetto.
 
 Se il personaggio scelto è un Lupo, un Negromante o un Fantasma, il potere del
Necrofilo non ha effetto.
 Se il personaggio scelto è uno Spettro, il Necrofilo ottiene il potere che quel
personaggio aveva prima di diventare Spettro.
 
 Nel momento in cui il Necrofilo ottiene il potere speciale di un altro
personaggio, acquisisce anche il ruolo, l'aura e la proprietà di essere mistico
del personaggio in questione.

 \item {\bf Stalker} (aura bianca). Ogni due notti, lo Stalker può scegliere un
personaggio vivo e pedinarlo. Scopre se quel personaggio ha agito su altri
personaggi durante la notte, ed in tal caso su chi. Non scopre, però, cosa ha
fatto.
 
 Se il personaggio pedinato dallo Stalker utilizza il proprio potere su sé
stesso, lo Stalker riceve informazioni come se tale personaggio non avesse
agito.
 
 \item {\bf Veggente} (aura bianca, mistico). Ogni notte, il Veggente può
scegliere un personaggio vivo e scrutarlo nella sua sfera di cristallo. Scopre
il colore della sua aura.

 \item {\bf Voyeur} (aura bianca). Ogni due notti, il Voyeur può scegliere un
personaggio vivo e spiarlo. Scopre quali sono gli altri personaggi che
durante la notte hanno agito sul personaggio scelto.
 
 Se il personaggio spiato dal Voyeur utilizza il proprio potere su sé stesso, il
Voyeur non riceve tale informazione.

\end{itemize}


\subsection*{Fazione dei Lupi}

\begin{itemize}
 \item {\bf Lupi} (aura nera). Ogni notte, ciascun Lupo può scegliere un
personaggio vivo e tentare di ucciderlo.
 Se tutti i Lupi che decidono di usare il proprio potere speciale scelgono lo
stesso personaggio, questi muore.
 Se almeno due Lupi indicano personaggi diversi, il loro potere speciale non ha
effetto.
 
 I Lupi non possono uccidere i Negromanti. Se i Lupi tentano di uccidere un
Negromante, il loro potere non ha effetto.
 
 I Lupi conoscono le Fattucchiere e gli altri Lupi.

 \item {\bf Avvocato del diavolo} (aura nera). Ogni due notti, l'Avvocato del
diavolo può scegliere un personaggio vivo e scrivere per lui una missiva di
rilascio.
 Durante il giorno successivo, se l'assemblea decide, tramite la votazione, di
uccidere il personaggio scelto dall'Avvocato del diavolo, questi non muore.
 
 L'Avvocato del Diavolo conosce i Diavoli e gli eventuali altri Avvocati del
Diavolo.

 \item {\bf Diavolo} (aura nera, mistico). Ogni notte, il Diavolo può scegliere
un personaggio vivo e leggere nella sua anima. Scopre il ruolo del personaggio
scelto.
 
 Il Diavolo conosce gli Avvocati del Diavolo e gli eventuali altri Diavoli.
 
 \item {\bf Fattucchiera} (aura nera, mistico). Ogni notte la Fattucchiera può scegliere
un personaggio, vivo o morto, e stregarlo. Per quella notte, il colore dell'aura
del personaggio scelto risulta diverso da quello effettivo.
 
 La Fattucchiera può usare il suo potere speciale su sé stessa.
 
 La Fattucchiera conosce i Lupi e le eventuali altre Fattucchiere.
 
 \item{\bf Profanatore di tombe} (aura nera). Ogni due notti, il Profanatore di tombe
può scegliere un personaggio morto e profanarne la tomba. Se il personaggio scelto è
uno Spettro, e questi agisce, il suo potere non ha effetto.
 
 \item {\bf Rinnegato} (aura bianca). Il Rinnegato non ha alcun potere speciale.
Il Rinnegato conosce i Sequestratori e gli eventuali altri Rinnegati.

 \item {\bf Sequestratore} (aura nera). Ogni notte, il Sequestratore può
scegliere un personaggio vivo e rapirlo. Per quella notte, se il personaggio
scelto agisce, il suo potere non ha effetto.
 Stalker e Voyeur ricevono informazioni come se quel personaggio non avesse
agito.
 
 Questo potere speciale non può essere usato per due notti consecutive sullo
stesso personaggio.
 
 Il Sequestratore conosce i Rinnegati e gli eventuali altri Sequestratori.


\end{itemize}

\subsection*{Fazione dei Negromanti}
\label{spettri}
\begin{itemize}

 \item {\bf Negromanti} (aura bianca, mistici).
 Se la notte precedente nessun personaggio è stato risvegliato come Spettro,
ciascun Negromante può scegliere un personaggio morto e selezionare un potere
soprannaturale (per la lista dei possibili poteri, vedere oltre).
 Se tutti i Negromanti che decidono di usare il proprio potere speciale scelgono
lo stesso personaggio e selezionano lo stesso potere soprannaturale, il
personaggio scelto diventa uno Spettro e ottiene il potere selezionato.
 Da quel momento in poi, egli appartiene alla Fazione dei Negromanti. Gli viene
inoltre comunicata l'identità dei Negromanti che lo hanno risvegliato come
Spettro.
 Se almeno due Negromanti scelgono personaggi diversi o selezionano poteri
soprannaturali diversi, il loro potere speciale non ha effetto e nessun
personaggio viene risvegliato come Spettro.
 
 Una volta che un potere soprannaturale viene assegnato ad uno Spettro (compreso
il Fantasma), questo non può più essere scelto per i nuovi Spettri. Appena viene
creato uno Spettro con il potere della Morte, tutti i Negromanti perdono il
potere di creare Spettri.
 
 I Lupi, le Fattucchiere e tutti i personaggi che appartengono già alla Fazione
dei Negromanti non possono essere risvegliati come Spettri.

 I Negromanti non possono essere uccisi dai Lupi. Se i Lupi tentano di uccidere
un Negromante, il loro potere non ha effetto.

 I Negromanti conoscono gli altri Negromanti.
 
 \item {\bf Fantasma} (aura bianca). Il Fantasma non ha alcun potere speciale.
Se il Fantasma muore, diventa immediatamente uno Spettro ed ottiene uno dei
poteri soprannaturali non ancora assegnati, scelto in modo casuale fra quelli
possibili. Gli viene inoltre comunicata l'identità dei Negromanti, e ai
Negromanti viene comunicata l'identità del Fantasma.
 
 Al Fantasma non può venire assegnato il potere soprannaturale della Morte.
 
 \item {\bf Ipnotista} (aura bianca). Ogni due notti, l'Ipnotista può scegliere
un personaggio vivo e controllarne la mente.
 Da quel momento in poi il voto per il rogo di quel personaggio è considerato
uguale a quello espresso dall'Ipnotista; questo è vero anche se l'Ipnotista non
vota.
 Il personaggio scelto non viene informato di essere sotto il controllo
dell'Ipnotista.
 
 L'Ipnotista è immune al potere degli altri Ipnotisti. Se un Ipnotista tenta di
controllare la mente di un altro Ipnotista, il suo potere non ha effetto.
 L'Ipnotista è immune al potere dell'Amnesia. Se tale Spettro tenta di
ottenebrare i ricordi dell'Ipnotista, il suo potere non ha effetto.

 Se un Ipnotista è morto, i personaggi sotto il suo controllo votano secondo il
proprio volere.
 Un personaggio può essere sotto il controllo di un solo Ipnotista per volta, e
precisamente l'ultimo ad aver agito su di esso.

 L'Ipnotista conosce i Medium e gli eventuali altri Ipnotisti.

 \item {\bf Medium} (aura bianca, mistico). Ogni notte, il Medium può scegliere
un personaggio morto e contattarne lo spirito. Scopre il colore della sua aura e
se tale personaggio è diventato o meno uno Spettro.

 Il Medium conosce gli Ipnotisti e gli eventuali altri Medium.

 \item {\bf Spettri}. Gli Spettri non sono presenti all'inizio della partita: i
personaggi morti possono essere risvegliati come Spettri in seguito all'azione
di un Negromante.
Nel momento in cui un personaggio diviene uno Spettro, mantiene la sua aura ed
eventualmente la proprietà di essere mistico, ma inizia a giocare per la
Fazione dei Negromanti. Gli viene comunicata l'identità dei Negromanti che lo
hanno risvegliato, e ottiene un potere soprannaturale che può iniziare a usare
dalla notte successiva. Inoltre, perde per sempre gli eventuali poteri speciali
che possedeva.
 
 Uno Spettro rimane morto: in particolare non può essere ucciso, e il villaggio
non viene informato del suo risveglio.
 La presenza di uno Spettro che sta usando il suo potere passa inosservata agli
occhi del Voyeur.
 
 Se uno Spettro viene resuscitato dal Messia, continua a giocare per la Fazione
dei Negromanti, mantiene il suo potere soprannaturale da Spettro, ma non può 
utilizzarlo finché è in vita. Non riottiene i poteri speciali eventualmente
posseduti prima di essere stato risvegliato come Spettro, mantiene la sua aura
ed eventualmente la proprietà di essere mistico.
 
\end{itemize}


\paragraph{Poteri soprannaturali degli Spettri}

\begin{itemize}
 \item {\bf Amnesia}. Ogni notte, lo Spettro può scegliere un personaggio vivo e
ottenebrarne i ricordi. Il giorno successivo, il suo eventuale voto per il rogo
viene ignorato: a tutti gli effetti è come se non avesse votato. Ciò accade
anche se il personaggio scelto è sotto il controllo dell'Ipnotista: invece di
votare come l'Ipnotista, non vota affatto.
 
 L'Ipnotista è immune al potere dell'Amnesia. Se lo Spettro tenta di ottenebrare
i ricordi dell'Ipnotista, il suo potere non ha effetto.
 
 Questo potere non può essere usato per due notti consecutive sullo stesso
personaggio.

 \item {\bf Duplicazione}. Ogni notte, lo Spettro può scegliere un personaggio
vivo e assumerne l'aspetto. Il giorno successivo, il suo voto per il rogo conta
doppio. Non risulta, però, che il personaggio abbia votato due volte;
semplicemente, il personaggio per cui vota riceve un voto in più.

 \item {\bf Illusione}. Ogni due notti, lo Spettro può scegliere un personaggio,
vivo o morto, quindi generare un'illusione di un personaggio vivo. L'illusione compare
nella casa del personaggio scelto.
 Agli occhi dello Stalker e del Voyeur, è come se il personaggio di cui è stata
generata l'illusione avesse agito sul personaggio scelto.

 \item {\bf Mistificazione}. Ogni notte, lo Spettro può scegliere un personaggio,
vivo o morto, e creare un'aura magica nella sua casa. Per quella notte, il
personaggio scelto è considerato mistico. 
 
 \item {\bf Morte}. Ogni due notti, lo Spettro può scegliere un personaggio vivo
e ucciderlo.
 Lo Spettro non può uccidere i Lupi. Se lo Spettro tenta di uccidere un Lupo,
il suo potere non ha effetto.
 
 \item {\bf Occultamento}. Ogni notte, lo Spettro può scegliere un personaggio
vivo o morto, e creare attorno alla sua casa una fittissima nebbia magica. Per
quella notte, se un qualsiasi altro personaggio, eccetto l'Esorcista, agisce sul
personaggio scelto dallo Spettro, il suo potere non ha effetto.
 
 \item {\bf Visione}. Ogni notte, lo Spettro può scegliere un personaggio vivo e
scrutare nei suoi pensieri. Ne scopre la fazione di appartenenza.
 
\end{itemize}


\subsection{Utilizzo dei poteri e fallimento}
\label{fallimento}

\paragraph{Utilizzo dei poteri} 
Se un personaggio con un potere attivabile ogni due notti, oppure una volta a
partita, agisce, a prescindere dal fatto che agisca con successo, non potrà
farlo nella notte successiva, o nel resto della partita.

I personaggi che ottengono informazioni di qualsiasi tipo su altri personaggi,
le ottengono riguardo alla condizione in cui questi si trovavano al termine del
giorno precedente, anche se tali informazioni cambiano nel corso della notte.
Ad esempio, se un morto viene contemporaneamente risvegliato come Spettro e
visitato dal Medium, quest'ultimo scopre che il morto non è uno Spettro (perché
al termine del giorno precedente non lo era ancora diventato).

Se non è diversamente specificato, i personaggi non possono utilizzare i propri
poteri su loro stessi.

\paragraph{Fallimento} Ci sono alcuni modi di utilizzare il proprio potere che
sono a priori proibiti, e pertanto non risultano nemmeno tra le scelte
disponibili sull'interfaccia web. Ad esempio, l'Investigatore non può scegliere
di agire su un personaggio vivo; un Negromante non può provare ad
attivare il proprio potere se durante la notte precedente è stato creato uno
Spettro (anche se non è stato lui a crearlo).

Anche escludendo i casi menzionati sopra, non sempre un personaggio riesce ad
usare il proprio potere con successo.
Ci sono essenzialmente tre possibili ragioni per cui il potere può non avere
effetto.
\begin{itemize}
 \item Blocco da parte di un altro personaggio. Ad esempio, si può essere rapiti
da un Sequestratore, bloccati da un Esorcista o dallo Spettro con il potere
dell'Occultamento.
 \item Restrizioni sull'utilizzo del potere. Ad esempio, un Lupo non riesce a
uccidere se nella stessa notte un altro Lupo cerca di assassinare un personaggio
diverso; il potere del Necrofilo non ha alcun effetto su un Lupo.
 \item Contraddizioni nell'utilizzo dei poteri. Può succedere che non ci sia
alcun modo logico di risolvere le azioni dei personaggi nel corso di una notte
(vedi Sezione \ref{faq} per una discussione più approfondita). Questo accade per
esempio se tre Sequestratori si sequestrano in modo ciclico.
\end{itemize}
Nel caso in cui un personaggio cerchi di utilizzare il proprio potere e
fallisca, riceve una notifica di fallimento. Tale notifica non precisa la
ragione del fallimento.


\subsection{Lo status di Sindaco}
\label{sindaco}

Se durante l'assemblea si scopre che non vi è un personaggio che ha ricevuto
strettamente più voti di ogni altro, tra i personaggi che hanno ricevuto il
maggior numero di voti, viene condannato quello eventualmente votato dal
Sindaco.
Il Sindaco è scelto casualmente all'inizio della partita dai GM (in modo
scorrelato dalla fazione di appartenenza).

In qualsiasi momento il Sindaco può designare un successore: se il Sindaco
muore, il successore designato diventa il nuovo Sindaco. Se il Sindaco muore
senza aver mai designato un successore, o se il successore designato è morto, i
GM determinano casualmente chi riceverà la carica.

Ogni giorno, oltre a votare per mettere qualcuno al rogo, ogni abitante del
villaggio può votare per l'elezione di un nuovo Sindaco. Se alla fine della
giornata un personaggio ha ricevuto il voto di più del 50\% dei personaggi vivi,
riceve la carica di Sindaco (e il Sindaco precedente la perde).

L'eventuale elezione di un nuovo Sindaco ha effetto prima della condanna a morte
sul rogo: in caso di parità conta il voto del nuovo Sindaco, non di quello
vecchio.



% \pagebreak
% \section{L'interfaccia web}
% (TODO: QUANDO CI SARÀ, METTEREMO QUI UNA BREVE GUIDA)


\pagebreak
\section{Le cose che vorreste sapere ma non avete mai osato chiedere}
\label{faq}

Le regole spiegate fino a questo punto sono più che sufficienti per poter
giocare. Tuttavia possono accadere eventi piuttosto strani (ad esempio, cosa
succede se due Sequestratori si sequestrano a vicenda?). In tutti questi casi, i
GM decidono in modo inappellabile cosa accade; ai giocatori è sempre permesso
chiedere loro cosa accade in una di queste combinazioni.

Segue una lista (verosimilmente non esaustiva) di eventualità bizzarre con le
corrispettive spiegazioni.
Per rendere questo elenco più facilmente consultabile, più avanti vi è un indice
analitico con la lista dei ruoli (o altre parole chiave) e dei punti in cui essi
compaiono.

\begin{enumerate}
 
 \item In caso di contraddizione nell'utilizzo dei poteri, viene scelto
casualmente un personaggio (o più di uno, se necessario), che non utilizza il
proprio potere e riceve la consueta notifica di fallimento (questo è il
fallimento ``di terza categoria'', vedi Sezione \ref{fallimento}).
 
 Ad esempio, se un Esorcista, un Sequestratore e uno Spettro con il potere
dell'Occultamento agiscono simultaneamente sull'Esorcista, si verifica
facilmente (esercizio) che non c'è alcun modo coerente di risolvere le azioni.
 \index{Esorcista}
 \index{Sequestratore}
 \index{Spettro!Occultamento}
 \index{Fallimento dei poteri@\emph{Fallimento dei poteri}}
  
 \item Se nella stessa notte un Messia e un Negromante cercano di agire sullo
stesso personaggio (morto), quel personaggio viene resuscitato dal Messia e non
risvegliato come Spettro. In particolare, il Negromante riceve una notifica di
fallimento.
 \index{Messia}
 \index{Negromante}

 \item Se un Necrofilo usa il proprio potere su uno Spettro, copia il potere
speciale che quel personaggio aveva prima di essere risvegliato come Spettro.
 Se un Necrofilo usa il proprio potere su un Necrofilo morto che aveva già
acquisito un nuovo potere, egli acquisisce a sua volta quello stesso potere.
 \index{Necrofilo}
 \index{Spettro}
 
 \item I Negromanti non possono nemmeno provare a usare il proprio potere se
durante la notte precedente è stato creato uno Spettro.
 \index{Negromante}
 
 \item Se più Negromanti cercano di creare degli Spettri, falliscono come
succede nel caso dei Lupi, a meno che scelgano tutti lo stesso personaggio e
selezionino tutti lo stesso potere soprannaturale.
 \index{Negromante}

 \item Se un Ipnotista morto viene resuscitato dal Messia, tutti coloro che si
trovavano sotto il controllo dell'Ipnotista al momento della sua morte sono
nuovamente sotto il suo controllo, a meno che nel frattempo la loro mente sia
stata controllata da un altro Ipnotista.
 
 Se un personaggio sotto il controllo di un Ipnotista muore e viene
successivamente resuscitato, torna ad essere sotto il controllo dell'ultimo
Ipnotista ad aver agito su di esso.
 \index{Ipnotista}
 \index{Messia}
 
 \item Se un personaggio smette di essere sotto il controllo di un Ipnotista (ad
esempio perché un altro Ipnotista usa il suo potere su di esso), l'Ipnotista non
riceve alcuna notifica.
 \index{Ipnotista}
 
 \item Se più Ipnotisti agiscono sullo stesso personaggio durante la stessa
notte, la mente di quest'ultimo viene controllata da tutti gli Ipnotisti in un
ordine casuale, e perciò risulterà infine essere sotto il controllo soltanto
dell'ultimo ad aver agito. Gli altri Ipnotisti non ricevono alcuna notifica di
fallimento.
 \index{Ipnotista}
 
 \item Se uno Spettro con il potere dell'Illusione fa comparire l'illusione di un
personaggio in una casa dove agiscono anche quel personaggio e il Voyeur, il
Voyeur vede comunque solo una volta quel personaggio. Se tuttavia il personaggio
scelto coincide con il Voyeur, questi compare nella lista dei personaggi visti.
 \index{Spettro!Illusione}
 \index{Voyeur}

 \item Se l'Esorcista e lo Spettro con il potere dell'Occultamento agiscono
sulla stessa persona, è il potere dello Spettro a non avere effetto, mentre
quello dell'Esorcista funziona regolarmente.
 \index{Esorcista}
 \index{Spettro!Occultamento}
 
 \item La Guardia del corpo protegge il personaggio scelto solamente
dall'attacco dei Lupi. Il Cacciatore, lo Spettro con il potere della Morte ed
eventuali altri effetti che uccidono il personaggio scelto non sono influenzati
dal potere della Guardia del corpo.
 \index{Guardia del corpo}
 \index{Lupo}
 \index{Cacciatore}
 \index{Spettro!Morte}
 
 \item Se in notti diverse un personaggio muore, viene risvegliato come Spettro,
resuscitato dal Messia e la sua anima viene letta dal Diavolo, il Diavolo scopre
che il personaggio è uno Spettro, ma non scopre che potere soprannaturale ha.
 \index{Spettro}
 \index{Messia}
 \index{Diavolo}
 
 \item Se più Fattucchiere agiscono sulla stessa persona, ciascuna cambia il
colore dell'aura. Pertanto, il colore dell'aura percepito durante quella notte è
quello corretto se le Fattucchiere in questione sono in numero pari, mentre è
quello sbagliato se le Fattucchiere sono in numero dispari.
 \index{Fattucchiera}
 
 \item Se al momento della morte del Fantasma non sono più disponibili poteri
soprannaturali, il Fantasma non diventa uno Spettro.
 I Negromanti hanno sempre la priorità sui Fantasmi morti di notte: prima
avviene l'eventuale risveglio di un morto come Spettro ad opera dei Negromanti,
e poi i Fantasmi appena morti ottengono casualmente uno dei poteri
soprannaturali ancora disponibili (se ve ne sono).
 Se più Fantasmi muoiono contemporaneamente e non sono disponibili abbastanza
poteri soprannaturali per assegnarne a ciascuno di essi, vengono scelti
casualmente i Fantasmi che diventano effettivamente degli Spettri e ottengono un
potere soprannaturale.
 \index{Fantasma}
 \index{Negromante}
 
 \item Un personaggio può votare sé stesso, sia per la condanna a morte sul rogo
che per l'elezione del Sindaco.
 \index{Votazione per il rogo@\emph{Votazione per il rogo}}
 \index{Elezione del Sindaco@\emph{Elezione del Sindaco}}
 \vspace{-4 mm}
 \item Se più personaggi ottengono il massimo numero di voti per il rogo e
quello scelto casualmente per essere ucciso è stato protetto dall'Avvocato del
Diavolo, nessuno muore e non viene resa pubblica l'identità del personaggio
protetto dall'Avvocato.
 \index{Avvocato del Diavolo}
 \index{Votazione per il rogo@\emph{Votazione per il rogo}}
 
 \item Se durante lo stesso giorno un personaggio viene eletto Sindaco e
condannato a morte sul rogo, immediatamente dopo viene nominato casualmente un
nuovo Sindaco (non vi è stato tempo per la nomina di un successore).
 \index{Elezione del Sindaco@\emph{Elezione del Sindaco}}
 
 \item Il potere dell'Ipnotista e i poteri soprannaturali dell'Amnesia e della
Duplicazione non incidono sull'elezione del sindaco, solo sul voto per il rogo.
 \index{Ipnotista}
 \index{Spettro!Amnesia}
 \index{Spettro!Duplicazione}
 \index{Elezione del Sindaco@\emph{Elezione del Sindaco}}
 
 \item Se lo Spettro con il potere dell'Amnesia e lo Spettro con il potere della
Duplicazione agiscono contemporaneamente su di un personaggio, il voto di quel
personaggio è comunque annullato. In altre parole, il potere dell'Amnesia
annulla il potere della Duplicazione.
 \index{Spettro!Amnesia}
 \index{Spettro!Duplicazione}
 
 \item Quando lo Spettro con il potere della Duplicazione agisce, il villaggio
non viene messo al corrente dell'identità del personaggio che ha beneficiato del
potere. Per esempio, se il personaggio X viene votato da A, B e C, e durante la
notte precedente lo Spettro aveva usato il suo potere su C, allora il server
riporterà un messaggio simile a ``X è stato votato da A, B e C, ricevendo un
totale di 4 voti''.
 \index{Votazione per il rogo@\emph{Votazione per il rogo}}
 \index{Spettro!Duplicazione}
 
 \item Lo Spettro con il potere dell'Illusione che agisce su un personaggio può
dirigere l'illusione dal personaggio stesso. In questo caso, un eventuale Stalker
che utilizzi a sua volta il suo potere sullo stesso personaggio verrebbe a
sapere che tale personaggio non ha agito.
 \index{Spettro!Illusione}
 \index{Stalker}
 
 \item  Se lo Spettro con il potere dell'Illusione decide di dirigere l'illusione 
nella casa del personaggio su cui ha agito anche un Esorcista, il suo potere
non ha effetto.
 Se invece lo Spettro decide di generare l'illusione di un personaggio su cui ha
agito anche un Esorcista, il suo potere funziona normalmente.
 \index{Esorcista}
 \index{Spettro!Illusione}
 
 \item Cercare di uccidere un personaggio che nella stessa notte muore anche per
altri motivi non costituisce di per sé una causa di fallimento.
 Per esempio, se contemporaneamente un Lupo e lo Spettro con il potere della
Morte utilizzano il proprio potere su di uno stesso personaggio, hanno entrambi
successo (a meno che non intervengano cause di fallimento, come spiegato nella
Sezione \ref{fallimento}).
 \index{Morte@\emph{Morte}}
 \index{Fallimento dei poteri@\emph{Fallimento dei poteri}}
 
 \item Al fine di determinare la validità della votazione, la percentuale di
 votanti per il rogo è calcolata dopo aver applicato gli effetti dell'Ipnotista
 e dello Spettro con il potere dell'Amnesia, ma prima di aver applicato gli effetti
 dello Spettro con il potere della Duplicazione.
 Per esempio, se su 10 personaggi vivi ve ne sono 5 sotto il controllo dell'Ipnotista,
 e l'Ipnotista non vota, allora il quorum del 50\% non è raggiunto perché l'Ipnotista
 e i personaggi sotto il suo controllo non votano.
 \index{Votazione per il rogo@\emph{Votazione per il rogo}}
 \index{Ipnotista}
 \index{Spettro!Amnesia}
 \index{Spettro!Duplicazione}

 
\end{enumerate}

\printindex




\end{document}
