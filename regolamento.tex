\documentclass[a4paper,10pt]{article}


\usepackage[utf8]{inputenc}
\usepackage[italian]{babel}
\usepackage[T1]{fontenc}
\usepackage[dvips]{graphicx}
\usepackage{amsmath}
\usepackage{amsthm}
\usepackage{fancyhdr}
\usepackage{amsfonts}
\usepackage{amssymb}

\topmargin -1cm
\oddsidemargin -0.5cm
%\evensidemargin	-1cm
\textwidth 17cm


\newcommand{\smallspace}{\vskip0.3cm}

% Title Page
\title{Lupus in tempo reale\\ Regolamento della quarta edizione}
\author{Giove}

\begin{document}
\maketitle


\section{Introduzione}

...

\section{Partecipanti}

Possono partecipare:
\begin{itemize}
 \item gli studenti della Scuola;
 \item altre persone che frequentano spesso gli ambienti della Scuola, in particolare il collegio Carducci e la mensa, dal lunedì al venerdì.
\end{itemize}

Non è necessario saper già giocare a Lupus in Tabula, è sufficiente farsi spiegare le regole di base da qualcuno che sa già giocare.

Tutti i partecipanti si impegnano a farsi vedere un po' in giro negli ambienti della Scuola nei giorni della partita. Nessuno vi chiede di non partire venerdì mattina per tornare a casa se non avete lezione, ma se vivete da reclusi in camera vostra e non mangiate mai in mensa potrebbe essere il caso di non giocare.


\section{Quando si giocherà}

...


\section{Modifiche rispetto al regolamento della terza edizione}

...


\section{Giorni e notti di gioco}

I giorni di gioco si estendono dalle 8:00 (circa) alle 22:00 di lunedì, martedì, mercoledì e giovedì, e dalle 8:00 (circa) di venerdì alle 22:00 di domenica.
Le notti di gioco si estendono dalle 22:00 (circa) alle 8:00 di domenica, lunedì, martedì, mercoledì e giovedì.


\section{Svolgimento del giorno e votazioni}

L'inizio del giorno è annunciato sul sito web, insieme a tutte le informazioni ottenute da ciascun giocatore in seguito agli avvenimenti della notte appena trascorsa.
I personaggi vivi hanno il diritto di votare entro le ore 22 una persona del villaggio da uccidere. Dopo le ore 22, viene pubblicata la lista dei votanti insieme alla preferenza espressa da ciascuna persona. Il voto è ritenuto valido se almeno il 50\% dei vivi ha votato; in caso contrario, nessuno muore. La persona che ha ricevuto più voti di tutti muore se ha ricevuto almeno il 30\% dei voti totali espressi; in caso contrario, nessuno muore. Se due o più persone hanno ricevuto lo stesso numero di voti, che è superiore a quello dei voti ricevuti da chiunque altro, ed ognuna ha ricevuto almeno il 30\% dei voti totali espressi, muore quella che è stata votata dal Sindaco (vedi Sezione XYZ); se nessuna delle due è stata votata dal sindaco, ne muore una a caso.

TODO: elezione del Sindaco (?)


\section{Svolgimento della notte e poteri speciali}

La notte inizia nel momento in cui viene pubblicato l'esito della votazione del giorno, e termina alle 8:00 della mattina seguente.
Durante la notte, i personaggi che hanno poteri speciali hanno la facoltà di attivarli.
Alcuni personaggi hanno un potere speciale attivabile ``ogni due notti'': si intende che tale potere può essere usato se e solo se non è stato usato durante la notte precedente (in altre parole, l'unica restrizione è data dal non poterlo usare in due notti consecutive).


\section{Comunicazioni master-giocatori, votazioni, attivazioni dei poteri}




\end{document}
