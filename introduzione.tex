\documentclass[a4paper,10pt]{article}


\usepackage[utf8]{inputenc}
\usepackage[italian]{babel}
\usepackage[T1]{fontenc}
\usepackage{amsmath}
\usepackage{amsthm}
\usepackage{fancyhdr}
\usepackage{amsfonts}
\usepackage{amssymb}
\usepackage[parfill]{parskip}

\topmargin -1cm
\oddsidemargin -0.5cm
%\evensidemargin	-1cm
\textwidth 17cm


\newcommand{\smallspace}{\vskip0.3cm}

% Title Page
\title{Introduzione a Lupus in tempo reale}
\author{Alessandro Iraci, Giovanni Paolini, Leonardo Tolomeo}

\begin{document}
\maketitle

Quella che segue è una descrizione molto sommaria delle regole di \emph{Lupus in tempo reale}, che serve
a chi non ha mai partecipato per capire le meccaniche di questo gioco. Per partecipare, comunque, è molto
importante leggere il regolamento completo.

All'inizio del gioco, ad ogni giocatore viene segretamente assegnato un ruolo,
che determina anche la fazione di cui farà parte.
Ci sono tre fazioni: i Popolani, i Lupi e i Negromanti. La partita termina
quando rimane una sola fazione in gioco, che viene dichiarata vincitrice.
Un giocatore vince se la sua fazione vince, a prescindere dal fatto che
questi sia vivo o morto al termine della partita.

I turni di gioco sono divisi in giorni e notti. Durante il giorno, ciascuno
ha il diritto di votare un giocatore da mandare al rogo. 
Al termine del giorno, salvo eccezioni, muore il giocatore con più voti.
Durante la notte i giocatori hanno diritto ad usare i loro poteri
speciali, che dipendono dal ruolo. Un elenco dei ruoli e dei rispettivi
poteri speciali è disponibile nel regolamento.

Quella che segue è una descrizione sommaria delle fazioni.
\begin{itemize}
 \item I Lupi si accordano fra loro per uccidere un personaggio a loro scelta. 
    I loro obiettivi principali sono i Popolani con poteri che potrebbero farli scoprire,
    ma qualunque avversario in meno li avvicina alla vittoria.
    Gli altri giocatori della loro fazione li aiutano ad
    ottenere informazioni, costruirsi una copertura credibile e rallentare le indagini dei Popolani.
 \item I Negromanti possono, ogni due notti, scegliere un personaggio morto e trasformarlo in uno Spettro.
    Questi si aggiungerà alla loro fazione e otterrà un nuovo potere speciale.
    Gli Spettri, così come gli altri membri della fazione, aiutano i Negromanti a mescolarsi nel villaggio
    e a manipolare le votazioni del giorno.
    I Negromanti non avranno abbastanza tempo per creare tutti gli Spettri disponibili, e devono valutare
    con attenzione il momento in cui risvegliare l'ultimo Spettro, che ha il potere di uccidere.
 \item I Popolani sono la fazione più numerosa, ma anche la più disorganizzata. All'inizio della 
    partita non si conoscono fra loro, e molti dei loro poteri servono a ottenere informazioni sugli 
    altri giocatori. Altri personaggi, invece, proteggono il villaggio dalla furia omicida dei Lupi
    e dai poteri degli Spettri e dei Negromanti.
    I Popolani devono puntare a scoprire quante più informazioni possibile, organizzare
    controlli a tappeto sugli altri giocatori e uccidere dei nemici con le votazioni ogni
    qualvolta ne hanno l'occasione. Spesso conviene mandare al rogo qualcuno anche sulla base di
    semplici sospetti, perché non farlo, di fatto, permette ai Lupi di uccidere un Popolano
    in più.
\end{itemize}

In questo gioco è fondamentale la capacità di mentire, di crearsi una copertura
perfetta e di coordinare le azioni della propria fazione. L'interazione con gli altri
giocatori è importantissima, ed è ciò che rende divertente \emph{Lupus in tempo reale}.

In generale conviene tenere segreto il proprio ruolo. Se si gioca per i Lupi o per i Negromanti,
ovviamente, venire scoperti comporta quasi certamente la morte sul rogo; se si parteggia per i
Popolani, invece, esporsi significa diventare un bersaglio appetibile per gli avversari.
Nonostante ciò, a volte ai Popolani conviene rivelare il proprio ruolo a persone fidate, per
comunicare preziose informazioni in proprio possesso o per aiutare la fazione a coordinarsi.
Bisogna ricordarsi, però, che mentire è fondamentale in questo gioco, e fidarsi degli altri è
sempre una scelta da fare con molta cautela.

\end{document}
